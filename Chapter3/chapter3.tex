\section{整数规划}


\subsubsection{割平面法求解}

\begin{problem}{1.1$\bigstar$}
    \begin{maxi*}|s|
        {}
        {x_1 + x_2}
        {}
        {}
        \addConstraint{2x_1 + x_2}{\leq 6}
        \addConstraint{4x_1 + 5x_2}{\leq 20}
        \addConstraint{x_1,x_2}{\geq 0}{,\text{且为整数}.}
    \end{maxi*}
\end{problem}
\begin{solution}
    第一步,单纯形法求解原问题
    \begin{center}
        \begin{tblr}{
                hline{5,6,8,9} = {0.08em},
                cell{5,8}{1} = {c=3,r=1}{c},
            }
            c_j \rightarrow &&& 1   & 1   & 0   & 0   \\
            C_B  & X_B  &b    & x_1 & x_2 & x_3 & x_4 \\
            0    & x_3  &6    & [2] & 1   & 1   & 0   \\
            0    & x_4  &20   & 4   & 5   & 0   & 1   \\
            c_j - z_j       &&& 1   & 1   & 0   & 0   \\
            1    & x_1  &3    & 1   & 1/2 & 1/2 & 0   \\
            0    & x_4  &8    & 0   & [3] & -2  & 1   \\
            c_j - z_j       &&& 0   & 1/2 & -1/2& 0   \\
            1    & x_1  &5/3  & 1   & 0   & 5/6 & -1/6\\
            1    & x_2  &8/3  & 0   & 1   & -2/3& 1/3 \\
            c_j - z_j       &&& 0   & 0   & -1/6& -1/6\\
        \end{tblr}
    \end{center}
    得点(5/3,8/3),非整数,选择$x_1$小数部分2/3进行割平面。
    形成约束:
    $$-5/6x_3 + 1/6x_4 \leq -2/3$$
    第二步,引入松弛变量$x_5$,继续用单纯形法求解
    \begin{center}
        \begin{tblr}{
                hline{6,7,10,11} = {0.08em},
                cell{6,10}{1} = {c=3,r=1}{c},
            }
            c_j \rightarrow &&& 1   & 1   & 0   & 0   & 0   \\
            C_B  & X_B  &b    & x_1 & x_2 & x_3 & x_4 & x_5 \\
            1    & x_1  &5/3  & 1   & 0   & 5/6 & -1/6& 0   \\
            1    & x_2  &8/3  & 0   & 1   & -2/3& 1/3 & 0   \\
            0    & x_5  &-2/3 & 0   & 0   &[-5/6]& 1/6& 1   \\
            c_j - z_j       &&& 0   & 0   & -1/6& -1/6& 0   \\
            1    & x_1  &1    & 1   & 0   & 0   & 0   & 1   \\
            1    & x_2  &16/5 & 0   & 1   & 0   & 1/5 & -4/5\\
            0    & x_3  &4/5  & 0   & 0   & 1   & -1/5& -6/5\\
            c_j - z_j       &&& 0   & 0   & 0   & -1/5& -1/5\\
        \end{tblr}
    \end{center}
    得点(1,16/5),$x_2$非整数,选择$x_2$小数部分1/5进行割平面。
    形成约束:
    $$-1/5x_4 + 4/5x_5 \leq -1/5$$
    第三步,引入松弛变量$x_6$,继续用单纯形法求解
    \begin{center}
        \begin{tblr}{
                hline{7,8,12,13} = {0.08em},
                cell{7,12}{1} = {c=3,r=1}{c},
            }
            c_j \rightarrow &&& 1   & 1   & 0   & 0   & 0   & 0   \\
            C_B  & X_B  &b    & x_1 & x_2 & x_3 & x_4 & x_5 & x_6 \\
            1    & x_1  &1    & 1   & 0   & 0   & 0   & 1   & 0   \\
            1    & x_2  &16/5 & 0   & 1   & 0   & 1/5 & -4/5& 0   \\
            0    & x_3  &4/5  & 0   & 0   & 1   & -1/5& -6/5& 0   \\
            0    & x_6  &-1/5 & 0   & 0   & 0   &[-1/5]&4/5 & 1   \\
            c_j - z_j       &&& 0   & 0   & 0   & -1/5& -1/5& 0   \\
            1    & x_1  &1    & 1   & 0   & 0   & 0   & 1   & 0   \\
            1    & x_2  &3    & 0   & 1   & 0   & 0   & 0   & 1   \\
            0    & x_3  &1    & 0   & 0   & 1   & 0   & -2  & -1   \\
            0    & x_4  &1    & 0   & 0   & 0   & 1   & -4  & -5   \\
            c_j - z_j       &&& 0   & 0   & 0   & 0   & -1  & -1   \\
        \end{tblr}
    \end{center}
    得点(1,3),整数,目标函数值$f_{max}=1+3=4$
\end{solution}
\begin{problem}{1.2$\bigstar$}
    \begin{mini*}|s|
        {}
        {5x_1 + x_2}
        {}
        {}
        \addConstraint{3x_1 + x_2}{\geq 9}
        \addConstraint{x_1 + x_2}{\geq 5}
        \addConstraint{x_1 + 8x_2}{\geq 8}
        \addConstraint{x_1,x_2}{\geq 0}{,\text{且为整数}.}
    \end{mini*}
\end{problem}

\begin{solution}
    第一步,单纯形法求解原问题
    \begin{center}
        \begin{tblr}{
                hline{6,7} = {0.08em},
                cell{6}{1} = {c=3,r=1}{c},
            }
            c_j \rightarrow &&& -5  & -1  & 0   & 0   & 0   \\
            C_B  & X_B  &b    & x_1 & x_2 & x_3 & x_4 & x_5 \\
            0    & x_3  &-9   & -3  & [-1]& 1   & 0   & 0   \\
            0    & x_4  &-5   & -1  & -1  & 0   & 1   & 0   \\
            0    & x_5  &-8   & -1  & -8  & 0   & 0   & 1   \\
            c_j - z_j       &&& -5  & -1  & 0   & 0   & 0   \\
            1    & x_2  &9    & 3   & 1   & -1  & 0   & 0   \\
            0    & x_4  &4    & 2   & 0   & 1   & 1   & 0   \\
            0    & x_5  &64   & 23  & 0   & -8  & 0   & 1   \\
            c_j - z_j       &&& -2  & 0   & -1  & 0   & 0   \\
        \end{tblr}
    \end{center}
    得点$(0,9)$,整数最优解,目标函数值$f_{min}=0+9=9$。
\end{solution}

\begin{problem}{1.3}
    \begin{mini*}|s|
        {}
        {x_1 - 2x_2}
        {}
        {}
        \addConstraint{x_1 + x_2}{\leq 10}
        \addConstraint{-x_1 + x_2}{\leq 5}
        \addConstraint{x_1,x_2}{\geq 0}{,\text{且为整数}.}
    \end{mini*}
\end{problem}

\begin{problem}{1.4}
    \begin{mini*}|s|
        {}
        {5x_1 + 3x_2}
        {}
        {}
        \addConstraint{2x_1 + x_2}{\geq 10}
        \addConstraint{x_1 + 3x_2}{\geq 9}
        \addConstraint{x_1,x_2}{\geq 0}{,\text{且为整数}.}
    \end{mini*}
\end{problem}

\subsubsection{分支定界法求解}

\begin{problem}{2.1$\bigstar$}
    \begin{maxi*}|s|
        {}
        {2x_1 + x_2}
        {}
        {}
        \addConstraint{x_1 + x_2}{\leq 5}
        \addConstraint{-x_1 + x_2}{\leq 0}
        \addConstraint{6x_1 + 2x_2}{\leq 21}
        \addConstraint{x_1,x_2}{\geq 0}{,\text{且为整数}.}
    \end{maxi*}
\end{problem}
\begin{solution}
    第一步,单纯形法求解原问题
    \begin{center}
        \begin{tblr}{
                hline{6,7,10,11} = {0.08em},
                cell{6,10}{1} = {c=3,r=1}{c},
            }
            c_j \rightarrow &&& 2   & 1   & 0   & 0   & 0   \\
            C_B  & X_B  &b    & x_1 & x_2 & x_3 & x_4 & x_5 \\
            0    & x_3  &5    & 1   & 1   & 1   & 0   & 0   \\
            0    & x_4  &0    & -1  & 1   & 0   & 1   & 0   \\
            0    & x_5  &21   & [6] & 2   & 0   & 0   & 1   \\
            c_j - z_j       &&& 2   & 1   & 0   & 0   & 0   \\
            0    & x_3  &3/2  & 0   &[2/3]& 1   & 0   & -1/6\\
            0    & x_4  &7/2  & 0   & 4/3 & 0   & 1   & 1/6 \\
            2    & x_1  &7/2  & 1   & 1/3 & 0   & 0   & 1/6 \\
            c_j - z_j       &&& 0   & 1/3 & 0   & 0   & -1/3\\
            1    & x_2  &9/4  & 0   & 1   & 3/2 & 0   & -1/4\\
            0    & x_4  &1/2  & 0   & 0   & -2  & 1   & 1/2 \\
            2    & x_1  &11/4 & 1   & 0   & -1/2& 0   & 1/4 \\
            c_j - z_j       &&& 0   & 0   & 0   & 0   & -1/4\\
        \end{tblr}
    \end{center}
    得点$(\frac{11}{4},\frac{9}{4})$,非整数,设下界为$\frac{31}{4}$,对原问题的$x_1$进行分支
    \begin{maxi*}|s|
        {}
        {2x_1 + x_2}
        {}
        {(1)}
        \addConstraint{x_1 + x_2}{\leq 5}
        \addConstraint{-x_1 + x_2}{\leq 0}
        \addConstraint{6x_1 + 2x_2}{\leq 21}
        \addConstraint{x_1}{\geq 3}
        \addConstraint{x_1,x_2}{\geq 0}{,\text{且为整数}.}
    \end{maxi*}
    \begin{maxi*}|s|
        {}
        {2x_1 + x_2}
        {}
        {(2)}
        \addConstraint{x_1 + x_2}{\leq 5}
        \addConstraint{-x_1 + x_2}{\leq 0}
        \addConstraint{6x_1 + 2x_2}{\leq 21}
        \addConstraint{x_1}{\leq 2}
        \addConstraint{x_1,x_2}{\geq 0}{,\text{且为整数}.}
    \end{maxi*}
    第二步,解问题(1)
    \begin{center}
        \begin{tblr}{
                hline{7,8,12,13,17,18} = {0.08em},
                cell{7,12,17}{1} = {c=3,r=1}{c},
            }
            c_j \rightarrow &&& 2   & 1   & 0   & 0   & 0   & 0   \\
            C_B  & X_B  &b    & x_1 & x_2 & x_3 & x_4 & x_5 & x_6 \\
            0    & x_3  &5    & 1   & 1   & 1   & 0   & 0   & 0   \\
            0    & x_4  &0    & -1  & 1   & 0   & 1   & 0   & 0   \\
            0    & x_5  &21   & 6   & 2   & 0   & 0   & 1   & 0   \\
            0    & x_6  &-3   & [-1]& 0   & 0   & 0   & 0   & 1   \\
            c_j - z_j       &&& 2   & 1   & 0   & 0   & 0   & 0   \\
            0    & x_3  &2    & 0   & 1   & 1   & 0   & 0   & 1   \\
            0    & x_4  &3    & 0   & 1   & 0   & 1   & 0   & -1  \\
            0    & x_5  &3    & 0   & 2   & 0   & 0   & 1   & [6] \\
            2    & x_1  &3    & 1   & 0   & 0   & 0   & 0   & -1  \\
            c_j - z_j       &&& 0   & 1   & 0   & 0   & 0   & 2   \\
            0    & x_3  &3/2  & 0   & 2/3 & 1   & 0   & -1/6& 0   \\
            0    & x_4  &7/2  & 0   & 4/3 & 0   & 1   & 1/6 & 0   \\
            0    & x_6  &1/2  & 0   &[1/3]& 0   & 0   & 1/6 & 1   \\
            2    & x_1  &7/2  & 1   & 1/3 & 0   & 0   & 1/6 & 0   \\
            c_j - z_j       &&& 0   & 1/3 & 0   & 0   & -1/3& 0   \\
            0    & x_3  &1/2  & 0   & 0   & 1   & 0   & -1/2& -2  \\
            0    & x_4  &3/2  & 0   & 0   & 0   & 1   & -1/2& -4  \\
            1    & x_2  &3/2  & 0   & 1   & 0   & 0   & 1/2 & 3   \\
            2    & x_1  &3    & 1   & 0   & 0   & 0   & 0   & -1  \\
            c_j - z_j       &&& 0   & 1/3 & 0   & 0   & -1/2& -1  \\
        \end{tblr}
    \end{center}
    得点$(3,\frac{3}{2})$,非整数,继续分支
        \begin{maxi*}|s|
        {}
        {2x_1 + x_2}
        {}
        {(1.1)}
        \addConstraint{x_1 + x_2}{\leq 5}
        \addConstraint{-x_1 + x_2}{\leq 0}
        \addConstraint{6x_1 + 2x_2}{\leq 21}
        \addConstraint{x_1}{\geq 3}
        \addConstraint{x_2}{\geq 2}
        \addConstraint{x_1,x_2}{\geq 0}{,\text{且为整数}.}
    \end{maxi*}
    \begin{maxi*}|s|
        {}
        {2x_1 + x_2}
        {}
        {(1.2)}
        \addConstraint{x_1 + x_2}{\leq 5}
        \addConstraint{-x_1 + x_2}{\leq 0}
        \addConstraint{6x_1 + 2x_2}{\leq 21}
        \addConstraint{x_1}{\geq 3}
        \addConstraint{x_2}{\leq 1}
        \addConstraint{x_1,x_2}{\geq 0}{,\text{且为整数}.}
    \end{maxi*}

    第三步,解问题(1.1),易见第三约束不满足,无可行解;解问题(1.2),
    \begin{center}
        \begin{tblr}{
                hline{8,9,14,15,20,21} = {0.08em},
                cell{8,14,20}{1} = {c=3,r=1}{c},
            }
            c_j \rightarrow &&& 2   & 1   & 0   & 0   & 0   & 0   & 0   \\
            C_B  & X_B  &b    & x_1 & x_2 & x_3 & x_4 & x_5 & x_6 & x_7 \\
            0    & x_3  &3/2  & 0   & 2/3 & 1   & 0   & -1/6& 0   & 0   \\
            0    & x_4  &7/2  & 0   & 4/3 & 0   & 1   & 1/6 & 0   & 0   \\
            0    & x_6  &1/2  & 0   & 1/3 & 0   & 0   & 1/6 & 1   & 0   \\
            2    & x_1  &7/2  & 1   & 1/3 & 0   & 0   & 1/6 & 0   & 0   \\
            0    & x_7  &1    & 0   & [1] & 0   & 0   & 0   & 0   & 1   \\
            c_j - z_j       &&& 0   & 1/3 & 0   & 0   & -1/3& 0   & 0   \\
            0    & x_3  &1/2  & 0   & 0   & 1   & 0   & -1/2& 2   & 0   \\
            0    & x_4  &3/2  & 0   & 0   & 0   & 1   & -1/2& 4   & 0   \\
            0    & x_6  &1/2  & 0   & 0   & 0   & 0   & 1/2 & -3  & -1  \\
            2    & x_1  &3    & 1   & 0   & 0   & 0   & 0   & 1   & 0   \\
            1    & x_7  &1    & 0   & 1   & 0   & 0   & 0   & 0   & 1   \\
            c_j - z_j       &&& 0   & 0   & 0   & 0   & 0   & -2  & 0   \\
        \end{tblr}
    \end{center}
    得解(3,1),整数最优解,目标函数值为7,下界改设为7;\\
    第四步,继续解问题(2)
    \begin{center}
        \begin{tblr}{
                hline{7,8,12,13,17,18} = {0.08em},
                cell{7,12,17}{1} = {c=3,r=1}{c},
            }
            c_j \rightarrow &&& 2   & 1   & 0   & 0   & 0   & 0   \\
            C_B  & X_B  &b    & x_1 & x_2 & x_3 & x_4 & x_5 & x_6 \\
            0    & x_3  &5    & 1   & 1   & 1   & 0   & 0   & 0   \\
            0    & x_4  &0    & -1  & 1   & 0   & 1   & 0   & 0   \\
            0    & x_5  &21   & 6   & 2   & 0   & 0   & 1   & 0   \\
            0    & x_6  &2    & [1] & 0   & 0   & 0   & 0   & 1   \\
            c_j - z_j       &&& 2   & 1   & 0   & 0   & 0   & 0   \\
            0    & x_3  &3    & 0   & 1   & 1   & 0   & 0   & -1  \\
            0    & x_4  &2    & 0   & [1] & 0   & 1   & 0   & 1   \\
            0    & x_5  &9    & 0   & 2   & 0   & 0   & 1   & 6   \\
            2    & x_1  &2    & 1   & 0   & 0   & 0   & 0   & 1   \\
            c_j - z_j       &&& 0   & 1   & 0   & 0   & 0   & -2  \\
            0    & x_3  &1    & 0   & 0   & 1   & -1  & 0   & -2  \\
            1    & x_2  &2    & 0   & 1   & 0   & 1   & 0   & 1   \\
            0    & x_5  &5    & 0   & 0   & 0   & -2  & 1   & 4   \\
            2    & x_1  &2    & 1   & 0   & 0   & 0   & 0   & 1   \\
            c_j - z_j       &&& 0   & 0   & 0   & -1  & 0   & -3  \\
        \end{tblr}
    \end{center}
    得点$(2,2)$,整数最优解,目标函数值7.\\
    综上所述,原问题最优解为(3,1)与(2,2),目标函数值$f_{max}=7$
\end{solution}

\begin{problem}{2.2$\bigstar$}
    \begin{mini*}|s|
        {}
        {5x_1 - x_2}
        {}
        {}
        \addConstraint{3x_1 + 10x_2}{\leq 50}
        \addConstraint{7x_1 - 2x_2}{\leq 28}
        \addConstraint{x_1,x_2}{\geq 0}{,x_2\text{为整数}.}
    \end{mini*}
\end{problem}
\begin{solution}
    第一步,单纯形法求解原问题
    \begin{center}
        \begin{tblr}{
                hline{5,6} = {0.08em},
                cell{5}{1} = {c=3,r=1}{c},
            }
            c_j \rightarrow &&& -5  & 1   & 0   & 0   \\
            C_B  & X_B  &b    & x_1 & x_2 & x_3 & x_4 \\
            0    & x_3  &50   & 3   & [10]& 1   & 0   \\
            0    & x_4  &28   & 7   & -2  & 0   & 1   \\
            c_j - z_j       &&& -5  & 1   & 0   & 0   \\
            1    & x_2  &5    & 3/10& 1   & 1/10& 0   \\
            0    & x_4  &38   & 38/5& 0   & 1/5 & 1   \\
            c_j - z_j       &&&-53/10& 0  &-1/10& 0   \\
        \end{tblr}
    \end{center}
    得点$(0,5)$,整数,原问题得解,目标函数值$f_{min}=5\times0-5=-5$

\end{solution}

\begin{problem}{2.3}
    \begin{mini*}|s|
        {}
        {2x_1 + x_2 - 3x_3}
        {}
        {}
        \addConstraint{x_1 + x_2 + 2x3}{\leq 5}
        \addConstraint{2x_1 + 2x_2 - x_3}{\leq 1}
        \addConstraint{x_1,x_2,x_3}{\geq 0}{,\text{且为整数}.}
    \end{mini*}
\end{problem}
\begin{solution}

\end{solution}

\begin{problem}{2.4}
    \begin{mini*}|s|
        {}
        {4x_1 + 7x_2 + 3x_3}
        {}
        {}
        \addConstraint{x_1 + 3x_2 + x_3}{\geq 5}
        \addConstraint{3x_1 + x_2 + 2x_3}{\geq 8}
        \addConstraint{x_1,x_2,x_3}{\geq 0}{,\text{且为整数}.}
    \end{mini*}
\end{problem}
\begin{solution}

\end{solution}

\subsubsection{最小指派求解}

\begin{problem}{3.1$\bigstar$}
    $$\begin{bmatrix}
        10 & 11 & 4  & 2  & 8 \\
        7  & 11 & 10 & 14 & 12\\
        5  & 6  & 9  & 12 & 14\\
        13 & 15 & 11 & 10 & 7
    \end{bmatrix}$$
\end{problem}
\begin{solution}
    为原问题增加一行,利用匈牙利算法求解
    $$\begin{bmatrix}
        10 & 11 & 4  & 2  & 8 \\
        7  & 11 & 10 & 14 & 12\\
        5  & 6  & 9  & 12 & 14\\
        13 & 15 & 11 & 10 & 7 \\
        0  & 0  & 0  & 0  & 0 \\
    \end{bmatrix}
    \xrightarrow{\text{各行减去最小元素}}
    \begin{bmatrix}
        8  & 9  & 2  & 0  & 6 \\
        0  & 4  & 3  & 7  & 5 \\
        0  & 1  & 4  & 7  & 9 \\
        6  & 8  & 4  & 3  & 0 \\
        0  & 0  & 0  & 0  & 0 \\
    \end{bmatrix}
    \xrightarrow{\text{第2、3列减去最小元素}}
    \begin{bmatrix}
        8  & 8  & 0  & 0  & 6 \\
        0  & 3  & 1  & 7  & 5 \\
        0  & 0  & 2  & 7  & 9 \\
        6  & 7  & 2  & 3  & 0 \\
        0  & -1 & -2 & 0  & 0 \\
    \end{bmatrix}$$
    $
    \xrightarrow{\text{第5行去除负数}}
    \begin{bmatrix}
        8  & 8  & 0  & [0]& 6 \\
        [0]& 3  & 1  & 7  & 5 \\
        0  & [0]& 2  & 7  & 9 \\
        6  & 7  & 2  & 3  & [0]\\
        2  & 1  & [0]& 2  & 2 \\
    \end{bmatrix}$
    故,最优解
    $X=\begin{bmatrix}
        0  & 0  & 0  & 1  & 0 \\
        1  & 0  & 0  & 0  & 0 \\
        0  & 1  & 0  & 0  & 0 \\
        0  & 0  & 0  & 0  & 1 \\
        0  & 0  & 1  & 0  & 0 \\
    \end{bmatrix}$
\end{solution}
\begin{problem}{3.2$\bigstar$}
    $$\begin{bmatrix}
        15 & 18 & 21 & 24\\
        19 & 23 & 22 & 18\\
        26 & 17 & 16 & 19\\
        19 & 21 & 23 & 17
    \end{bmatrix}$$
\end{problem}
\begin{solution}
    用匈牙利算法求解
    $$\begin{bmatrix}
        15 & 18 & 21 & 24\\
        19 & 23 & 22 & 18\\
        26 & 17 & 16 & 19\\
        19 & 21 & 23 & 17
    \end{bmatrix}
    \xrightarrow{\text{各行减去最小元素}}
    \begin{bmatrix}
        0  & 3  & 6  & 9 \\
        1  & 5  & 4  & 0 \\
        10 & 1  & 0  & 3 \\
        2  & 4  & 6  & 0
    \end{bmatrix}
    \xrightarrow{\text{第2、4行减去最小元素}}
    \begin{bmatrix}
        0  & 3  & 6  & 9 \\
        0  & 4  & 3  & -1\\
        10 & 1  & 0  & 3 \\
        0  & 2  & 4  & -2
    \end{bmatrix}$$
    $$
    \xrightarrow{\text{第4列去除负数}}
    \begin{bmatrix}
        0  & 3  & 6  & 11\\
        0  & 4  & 3  & 1 \\
        10 & 1  & 0  & 5 \\
        0  & 2  & 4  & 0
    \end{bmatrix}
    \xrightarrow{\text{第1、2行减去最小元素}}
    \begin{bmatrix}
        -3 & 0  & 3  & 8 \\
        -1 & 3  & 2  & 0 \\
        10 & 1  & 0  & 5 \\
        0  & 2  & 4  & 0
    \end{bmatrix}
    \xrightarrow{\text{第1列去除负数}}
    \begin{bmatrix}
        0  & 0  & 3  & 8 \\
        2  & 3  & 2  & 0 \\
        13 & 1  & 0  & 5 \\
        3  & 2  & 4  & 0
    \end{bmatrix}$$
    $$
    \xrightarrow{\text{第2、4行减去最小元素}}
    \begin{bmatrix}
        0  & 0  & 3  & 8 \\
        0  & 1  & 0  & -2\\
        13 & 1  & 0  & 5 \\
        1  & 0  & 2  & -2
    \end{bmatrix}
    \xrightarrow{\text{第4列去除负数}}
    \begin{bmatrix}
        0  & 0  & 3  & 10\\
        0  & 1  & 0  & 0 \\
        13 & 1  & 0  & 7 \\
        1  & 0  & 2  & 0
    \end{bmatrix}$$
    故,最优解
    $X^{*}=\begin{bmatrix}
        1  & 0  & 0  & 0  \\
        0  & 0  & 0  & 1  \\
        0  & 0  & 1  & 0  \\
        0  & 1  & 0  & 0  \\
    \end{bmatrix}$
    或
    $X^{**}=\begin{bmatrix}
        0  & 1  & 0  & 0  \\
        1  & 0  & 0  & 0  \\
        0  & 0  & 1  & 0  \\
        0  & 0  & 0  & 1  \\
    \end{bmatrix}$
    最小值为70.
\end{solution}

\SetTblrInner {
    cells  = {c, m},
}
\begin{problem}{3.3$\bigstar$}
    公司在各地有四项业务,选定4位业务员去分别处理。由于业务能力、经验和其他情况的不同,4位业务员处理这4项业务的费用(单位:元)各不相同,见下表。应当怎样分派任务,才能使总的业务费最少?
        \begin{center}
        \begin{tblr}{
               hlines = {0.08em},
               vlines = {0.08em},
            }
            人=任务 & 1   & 2   & 3   & 4   \\
            甲      & 1100& 800 & 1000& 700 \\
            乙      & 600 & 500 & 300 & 800 \\
            丙      & 400 & 800 & 1000& 900 \\
            丁      & 1100& 1000& 500 & 700 \\
        \end{tblr}
    \end{center}
\end{problem}

\begin{problem}{3.4$\bigstar$}
    从甲乙丙丁戊五人中挑选四人去完成ABCD四项任务。每人完成各项任务所需的时间如下表所示。规定每项任务只能由一个人单独完成,且每人最多承担一项任务。又假定甲必须保证承担一项任务,丁不能承担任务D。在满足上述条件下,如何分配任务,使完成四项任务所花费的时间最小。
    \begin{center}
        \begin{tblr}{

            }
            人=任务 & A   & B   & C   & D   \\
            甲      & 10  & 5   & 15  & 20  \\
            乙      & 2   & 10  & 5   & 15  \\
            丙      & 3   & 15  & 14  & 13  \\
            丁      & 15  & 2   & 7   & 6   \\
            戊      & 9   & 4   & 15  & 8   \\
        \end{tblr}
    \end{center}
\end{problem}