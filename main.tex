%!TEX program = xelatex
\documentclass[cn,hazy,black,normal]{elegantnote}
\title{《最优化理论与方法》自拟习题及解答}

\author{姜孟冯}

\date{\zhtoday}

%%%%%%%%%%%%%%%%%%%%%%%%%%%%%%%%%%%%%%%%%%%%%%%%%%%%%%%%%%%%%%%%%%%%%
% PACKAGES                                                          %
%%%%%%%%%%%%%%%%%%%%%%%%%%%%%%%%%%%%%%%%%%%%%%%%%%%%%%%%%%%%%%%%%%%%%
\usepackage{amssymb}
\usepackage{optidef}
\usepackage{pgfplots}
\usepackage{tabularray}

%%%%%%%%%%%%%%%%%%%%%%%%%%%%%%%%%%%%%%%%%%%%%%%%%%%%%%%%%%%%%%%%%%%%%%
% STYLE ENVIRONMENT                                                %
%%%%%%%%%%%%%%%%%%%%%%%%%%%%%%%%%%%%%%%%%%%%%%%%%%%%%%%%%%%%%%%%%%%%%%
\usepgfplotslibrary{fillbetween}
\pgfplotsset{
    my axis style/.style={
        axis lines=middle, % 将坐标轴置于图形中心
        unit vector ratio=1 1 1, % 设置 x 轴和 y 轴的单位长度比例
        %        xmin=0,  % x 轴范围
        %        ymin=0,  % y 轴范围
        axis on top,
        xlabel={$x_1$},
        ylabel={$x_2$},
        legend pos=outer north east,
    }
}
\SetTblrInner {
    cells  = {c, m},
    row{2} = {mode = math},
    column{1,2} = {mode = math},
    hline{1,Z} = {0.15em},
    hline{2,3,Y} = {0.08em},
    vline{4} = {0.08em},
    cell{1}{1} = {c=3,r=1}{c},
    cell{Z}{1} = {c=3,r=1}{c},
}
%%%%%%%%%%%%%%%%%%%%%%%%%%%%%%%%%%%%%%%%%%%%%%%%%%%%%%%%%%%%%%%%%%%%%%
% PROBLEM ENVIRONMENT                                                %
%%%%%%%%%%%%%%%%%%%%%%%%%%%%%%%%%%%%%%%%%%%%%%%%%%%%%%%%%%%%%%%%%%%%%%
\usepackage{tcolorbox}
\tcbuselibrary{theorems, breakable, skins}

\newtcbtheorem[number within=subsection]{prob}% environment name
{问题}% Title text
{enhanced, % tcolorbox styles
    attach boxed title to top left={xshift = 4mm, yshift=-2mm},
    colback=blue!5, colframe=black, colbacktitle=blue!3, coltitle=black,
    boxed title style={size=small,colframe=gray},
    fonttitle=\bfseries,
    separator sign none
}%
{}

\newenvironment{problem}[1]{\begin{prob*}{#1}{}}{\end{prob*}}

\newtheorem*{solution}{解答.}

%%%%%%%%%%%%%%%%%%%%%%%%%%%%%%%%%%%%%%%%%%%%%%%%%%%%%%%%%%%%%%%%%%%%%%
% MY COMMANDS                                                        %
%%%%%%%%%%%%%%%%%%%%%%%%%%%%%%%%%%%%%%%%%%%%%%%%%%%%%%%%%%%%%%%%%%%%%%
\newcommand{\R}{\mathbf{R}}
\newcommand{\C}{\mathbf{C}}
\newcommand{\F}{\mathbf{F}}
\newcommand{\U}{\mathit{U}}
\newcommand{\V}{\mathit{V}}
\newcommand{\W}{\mathit{W}}
\newcommand{\poly}{\mathcal{P}}
\newcommand{\espace}{\mathcal{L}}
\newcommand{\expect}{\mathcal{E}}
\newcommand{\mat}{\mathcal{M}}
\newcommand{\mtxA}{\mathcal{A}}
\DeclareMathOperator{\Span}{span}
\DeclareMathOperator{\Real}{Re}
\DeclareMathOperator{\Imag}{Im}
\DeclareMathOperator{\Null}{null}
\DeclareMathOperator{\Range}{range}

\newcommand{\ph}{\phantom{+x_0}}
%\newcommand{\bigO}{\mathcal{O}}
%\newcommand{\mat}{\mathcal{M}}
%\newcommand{\defeq}{\vcentcolon=}
%\newcommand{\restr}[1]{|_{#1}}

%%%%%%%%%%%%%%%%%%%%%%%%%%%%%%%%%%%%%%%%%%%%%%%%%%%%%%%%%%%%%%%%%%%%%%
% SECTION NUMBERING                                                  %
%%%%%%%%%%%%%%%%%%%%%%%%%%%%%%%%%%%%%%%%%%%%%%%%%%%%%%%%%%%%%%%%%%%%%%
\renewcommand{\thesubsection}{\thesection\Alph{subsection}}
\renewcommand{\thesubsubsection}{\arabic{subsubsection}}

%%%%%%%%%%%%%%%%%%%%%%%%%%%%%%%%%%%%%%%%%%%%%%%%%%%%%%%%%%%%%%%%%%%%%%
% DOCUMENT                                                           %
%%%%%%%%%%%%%%%%%%%%%%%%%%%%%%%%%%%%%%%%%%%%%%%%%%%%%%%%%%%%%%%%%%%%%%
\begin{document}

    \maketitle



    \section{绪论}

\begin{example}
    对边长为$a$的正方形铁板,在四个角处剪去相等的正方形以制成方形无盖水槽,问如何剪法使水槽的容积最大?
\end{example}
\begin{solution}
    \begin{maxi*}|s|
        {}
        {(a-2x)^2x}
        {}
        {}
        \addConstraint{2x}{< a}
        \addConstraint{x}{\geq 0}{.}
    \end{maxi*}
    \begin{align*}
        f'&=2(a-2x)(-2)x+(a-2x)^2\\
          &=-4(ax-2x^2)+a^2+4x^2-4ax\\
          &=-8ax+12x^2+a^2\\
          &=(2x-a)(6x-a)=0\\
          x=a/6
    \end{align*}
\end{solution}
\begin{example}
    求侧面积为常数$6a^2(a>0)$,体积最大的长方体体积。
\end{example}
\begin{solution}
        \begin{maxi*}|s|
        {}
        {xyz}
        {}
        {}
        \addConstraint{2xy+2xz+2yz}{=6a^2}
        \addConstraint{x,y,z}{\geq 0}{.}
    \end{maxi*}
\end{solution}
\begin{problem}{1}
    一个矩形无盖油箱的外部总面积限定为$S$,怎样设计可使油箱的容积最大?写出该问题的数学模型,并回答该问题是几维最优化问题。
\end{problem}
\begin{solution}
    \begin{maxi*}|s|
        {}
        {xyz}
        {}
        {}
        \addConstraint{xy+2xz+2yz}{=S}
        \addConstraint{x,y,z}{\geq 0}{.}
    \end{maxi*}
\end{solution}
    \section{线性规划}

\subsection{图解法}


\usepgfplotslibrary{fillbetween}
\pgfplotsset{
    my axis style/.style={
        axis lines=middle, % 将坐标轴置于图形中心
        unit vector ratio=1 1 1, % 设置 x 轴和 y 轴的单位长度比例
        xmin=0,  % x 轴范围
        ymin=0,  % y 轴范围
        axis on top,
        xlabel={$x_1$},
        ylabel={$x_2$},
        legend pos=outer north east,
    }
}

\subsubsection{图解法求解}
\begin{problem}{1.1$\bigstar$}
    \begin{mini*}|s|
        {}
        {8x_1 + 5x_2}
        {}
        {}
        \addConstraint{-x_1 + x_2}{\geq 0}
        \addConstraint{6x_1 + 11x_2}{\geq 66}
        \addConstraint{2x_1 + x_2}{\geq 10}
        \addConstraint{x_1, x_2}{\geq 0}{.}
    \end{mini*}
\end{problem}
\begin{solution}
    \begin{tikzpicture}
        \begin{axis}[my axis style]
            \addplot[name path=a,color=red,domain=0:10] {x};
            \addlegendentry{$-x_1 + x_2=0$}
            \addplot[name path=b,color=blue,domain=0:11] {(66-6*x)/11};
            \addlegendentry{$6x_1 + 11x_2=66$}
            \addplot[name path=c,color=green,domain=0:5] {10-2*x};
            \addlegendentry{$2x_1 + x_2=10$}
            \addplot[name path=d,color=black,domain=0:6,dashed] {(44.5-8*x)/5};
            \addplot[geyecolor] fill between [of=a and c, reverse=false,split,every last segment/.style=gray!50];
            \path [name path=axis] (0,0) -- (11,0);
            \addplot[geyecolor] fill between [of=b and axis];
        \end{axis}
    \end{tikzpicture}

    解方程
    $$\left\{
    \begin{aligned}
        6x_1 + 11x_2&=66\\
        2x_1 + x_2&=10
    \end{aligned}\right.$$
    得
    $$\left\{
    \begin{aligned}
        x_1 &=2.75\\
        x_2 2&=4.5
    \end{aligned}\right.$$
    所以最大值为:$8*2.75+5*4.5=44.5$
\end{solution}

\begin{problem}{1.2$\bigstar$}
    \begin{mini*}|s|
        {}
        {13x_1 + 5x_2}
        {}
        {}
        \addConstraint{7x_1 + 3x_2}{\geq 19}
        \addConstraint{10x_1 + 2x_2}{\leq 11}
        \addConstraint{x_1, x_2}{\geq 0}{.}
    \end{mini*}
\end{problem}
\begin{solution}
    \begin{tikzpicture}
        \begin{axis}[
            axis lines = middle,
            unit vector ratio=1 1,
            xlabel = {$x_1$},
            ylabel = {$x_2$},
            xmin=0,
            ymin=0,
            legend pos=north west,
            ]
            \addplot[name path=a,color=red,domain=0:15] {(19-7*x)/3};
            \addlegendentry{$7x_1 + 3x_2=19$}
            \addplot[name path=b,color=blue,domain=0:15] {(11-10*x)/2};
            \addlegendentry{$10x_1 + 2x_2=11$}
            \addplot[name path=d,color=black,dashed] {(5.5*5-13*x)/5};
            \addplot[geyecolor] fill between [of=a and b, split,every segment no 0/.style=gray!50];
        \end{axis}
    \end{tikzpicture}
    由图像易得,无可行解。
\end{solution}

\begin{problem}{1.3}
    \begin{mini*}|s|
        {}
        {5x_1 - 6x_2}
        {}
        {}
        \addConstraint{x_1 + 2x_2}{\leq 10}
        \addConstraint{2x_1 - x_2}{\leq 5}
        \addConstraint{x_1 - 4x_2}{\leq 4}
        \addConstraint{x_1, x_2}{\geq 0}{.}
    \end{mini*}
\end{problem}


\begin{problem}{1.4}
    \begin{mini*}|s|
        {}
        {-x_1 + x_2}
        {}
        {}
        \addConstraint{3x_1 - 7x_2}{\geq 8}
        \addConstraint{x_1 - x_2}{\leq 5}
        \addConstraint{x_1, x_2}{\geq 0}{.}
    \end{mini*}
\end{problem}

\begin{problem}{1.5}
    \begin{maxi*}|s|
        {}
        {-20x_1 + 10x_2}
        {}
        {}
        \addConstraint{x_1 + x_2}{\geq 10}
        \addConstraint{-10x_1 + x_2}{\leq 10}
        \addConstraint{-5x_1 + 5x_2}{\leq 25}
        \addConstraint{x_1 + 4x_2}{\geq 20}
        \addConstraint{x_1, x_2}{\geq 0}{.}
    \end{maxi*}
\end{problem}
\begin{problem}{1.6}
    \begin{mini*}|s|
        {}
        {-3x_1 - 2x_2}
        {}
        {}
        \addConstraint{3x_1 + 2x_2}{\leq 6}
        \addConstraint{x_1 - 2x_2}{\leq 1}
        \addConstraint{x_1 + x_2}{\geq 1}
        \addConstraint{-x_1 + 2x_2}{\leq 1}
        \addConstraint{x_1, x_2}{\geq 0}{.}
    \end{mini*}
\end{problem}
\begin{problem}{1.7}
    \begin{mini*}|s|
        {}
        {5x_1 + 4x_2}
        {}
        {}
        \addConstraint{-2x_1 + x_2}{\geq -4}
        \addConstraint{x_1 + 2x_2}{\leq 6}
        \addConstraint{5x_1 + 3x_2}{\leq 15}
        \addConstraint{x_1, x_2}{\geq 0}{.}
    \end{mini*}
\end{problem}
\begin{problem}{1.8}
    \begin{maxi*}|s|
        {}
        {3x_1 + x_2}
        {}
        {}
        \addConstraint{x_1 - x_2}{\geq 0}
        \addConstraint{x_1 + x_2}{\leq 5}
        \addConstraint{6x_1 + 2x_2}{\leq 21}
        \addConstraint{x_1, x_2}{\geq 0}{.}
    \end{maxi*}
\end{problem}

\subsubsection{试通过求基本可行解来确定各问题的最优解}

\begin{problem}{2.1}
    \begin{maxi*}|s|
        {}
        {2x_1 + 5x_2}
        {}
        {}
        \addConstraint{x_1 + 2x_2 + x_3 \ph }{= 16}
        \addConstraint{2x_1 + x_2 \ph + x_4}{= 12}
        \addConstraint{x_j}{\geq 0}{,\ j=1,\ldots,4.}
    \end{maxi*}
\end{problem}
\begin{problem}{2.2}
    \begin{mini*}|s|
        {}
        {-2x_1 + x_2 + x_3 + 10x_4}
        {}
        {}
        \addConstraint{-x_1 + x_2 + x_3 + x_4}{= 20}
        \addConstraint{2x_1 - x_2 \ph + 2x_4}{= 10}
        \addConstraint{x_j}{\geq 0}{,\ j=1,\ldots,4.}
    \end{mini*}
\end{problem}
\begin{problem}{2.3}
    \begin{mini*}|s|
        {}
        {x_1 - x_2}
        {}
        {}
        \addConstraint{x_1 + x_2 + x_3}{\leq 5}
        \addConstraint{-x_1 + x_2 + 2x_3}{\leq 6}
        \addConstraint{x_1, x_2, x_3}{\geq 0}{.}
    \end{mini*}
\end{problem}


\subsection{单纯形法}

\subsubsection{单纯形法求解}
\begin{problem}{1.1}
    \begin{mini*}|s|
        {}
        {-9x_1 - 16x_2}
        {}
        {}
        \addConstraint{x_1 + 4x_2 + x_3 \ph}{=80 }
        \addConstraint{2x_1 + 3x_2 \ph +x_4}{=90}
        \addConstraint{x_j}{\geq 0}{,\ j=1,\ldots,4.}
    \end{mini*}
\end{problem}
\begin{problem}{1.2}
    \begin{maxi*}|s|
        {}
        {x_1 + 3x_2}
        {}
        {}
        \addConstraint{2x_1 + 3x_2 + x_3 \ph}{=6}
        \addConstraint{-x_1 + x_2 \ph + x_4}{=1}
        \addConstraint{x_j}{\geq 0}{,\ j=1,\ldots,4.}
    \end{maxi*}
\end{problem}
\begin{problem}{1.3}
    \begin{maxi*}|s|
        {}
        {-x_1 + 3x_2 + x_3}
        {}
        {}
        \addConstraint{3x_1 - x_2 + 2x_3}{\leq 7}
        \addConstraint{-2x_1 + 4x_2 \ph}{\leq 12}
        \addConstraint{-4x_1 + 3x_2 + 8x_3}{\leq 10}
        \addConstraint{x_1, x_2, x_3}{\geq 0}{.}
    \end{maxi*}
\end{problem}
\begin{problem}{1.4}
    \begin{mini*}|s|
        {}
        {3x_1 - 5x_2 - 2x_3 - x_4}
        {}
        {}
        \addConstraint{x_1 + x_2 + x_3}{\leq 4}
        \addConstraint{-x_1 + x_2 + 2x_3}{\leq 6}
        \addConstraint{-x_1 + x_2 + 2x_3}{\leq 12}
        \addConstraint{x_j}{\geq 0}{,\ j=1,\ldots,4.}
    \end{mini*}
\end{problem}
\begin{problem}{1.5}
    \begin{mini*}|s|
        {}
        {-3x_1 - x_2}
        {}
        {}
        \addConstraint{3x_1 + 3x_2 + x_3 \ph}{=30}
        \addConstraint{4x_1 - 4x_2 \ph + x_4}{=16}
        \addConstraint{2x_1 - x_2 \ph \ph}{\leq 12}
        \addConstraint{x_j}{\geq 0}{,\ j=1,\ldots,4.}
    \end{mini*}
\end{problem}

\subsubsection{转化为标准形式并列出单纯形表}

\begin{problem}{2.1$\bigstar$}
    \begin{maxi*}|s|
        {}
        {2x_1 - 2x_2 - 3x_3}
        {}
        {}
        \addConstraint{2x_1 - x_2 + 2x_3}{\leq 2}
        \addConstraint{-x_1 + 2x_2 - 3x_3}{\leq -2}
        \addConstraint{x_1, x_2, x_3}{\geq 0}{.}
    \end{maxi*}
\end{problem}
\begin{solution}
    \begin{maxi*}|s|
        {}
        {2x_1 - 2x_2 - 3x_3 + 0x_4 + 0x_5}
        {}
        {}
        \addConstraint{2x_1 - x_2 + 2x_3 + x_4}{= 2}
        \addConstraint{x_1 - 2x_2 + 3x_3 - x_5}{= 2}
        \addConstraint{x_1, x_2, x_3}{\geq 0}{.}
    \end{maxi*}
    \begin{center}
        \begin{tblr}{
                cells  = {c, m},
                row{2} = {mode = math},
                column{1,2} = {mode = math},
                hline{1,Z} = {0.15em},
                hline{2,3,Y} = {0.08em},
                vline{4} = {0.08em},
                cell{1}{1} = {c=3,r=1}{c},
                cell{Z}{1} = {c=3,r=1}{c},
            }
            c_j \rightarrow &&& 2  & -2  & -3  & 0   & 0  \\
            C_B  &X_B   &b    &x_1 & x_2 & x_3 & x_4 & x_5\\
            0    & x_4  &2    & 2  & -1  & 2   & 1   & 0  \\
            0    & x_5  &2    & 1  & -2  & 3   & 0   & -1 \\
            c_j - z_j       &&& 2  & -2  & -3  & 0   & 0  \\
        \end{tblr}
    \end{center}
\end{solution}

\begin{problem}{2.2$\bigstar$}
    \begin{mini*}|s|
        {}
        {-3x_1 + 4x_2 - 2x_3 + 5x_4}
        {}
        {}
        \addConstraint{4x_1 - x_2 + 2x_3 - x_4}{=-2}
        \addConstraint{x_1 + x_2 - x_3 + 2x_4}{\leq 14}
        \addConstraint{-2x_1 + 3x_2 + x_3 - x_4}{\geq 2}
        \addConstraint{x_1, x_2, x_3}{\geq 0}{,x_4\text{无约束}.}
    \end{mini*}
\end{problem}
\begin{solution}
    \begin{mini*}|s|
        {}
        {-3x_1 + 4x_2 - 2x_3 + 5x_5 - 5x_6 + 0x_7 + 0x_8}
        {}
        {}
        \addConstraint{-4x_1 + x_2 - 2x_3 + x_5 - x_6 \ph \ph}{=2}
        \addConstraint{x_1 + x_2 - x_3 + 2x_5 - 2x_6 + x_7 \ph}{= 14}
        \addConstraint{-2x_1 + 3x_2 + x_3 - x_5 + x_6 \ph - x_8}{= 2}
        \addConstraint{x_1, x_2, x_3,x_5,x_6,x_7,x_8}{\geq 0}{.}
    \end{mini*}
    \begin{center}
        \begin{tblr}{
                cells  = {c, m},
                row{2} = {mode = math},
                column{1,2} = {mode = math},
                hline{1,Z} = {0.15em},
                hline{2,3,Y} = {0.08em},
                vline{4} = {0.08em},
                cell{1}{1} = {c=3,r=1}{c},
                cell{Z}{1} = {c=3,r=1}{c},
            }
            c_j \rightarrow &&& -3 & 4   & -2  & 5   & -5  & 0   & 0  \\
            C_B  &X_B   &b    &x_1 & x_2 & x_3 & x_5 & x_6 & x_7 & x_8\\
            0    & x_3  &2    & 2  & 2   & 1   & 0   & 0   & 0   & 0  \\
            2    & x_4  &14   & 3  & -1  & 0   & 1   & 0   & 0   & 0  \\
            1    & x_5  &2    & 0  &  3  & 0   & 0   & 1   & 0   & 0  \\
            c_j - z_j       &&& 8  & 9   & 0   & 0   & 0   & 0   & 0  \\
        \end{tblr}
    \end{center}
\end{solution}

\subsubsection{分别用图解法与单纯形法求解并对比}

\begin{problem}{3.1$\bigstar$}
    \begin{maxi*}|s|
        {}
        {10x_1 + 5x_2}
        {}
        {}
        \addConstraint{3x_1 + 4x_2}{\leq 9}
        \addConstraint{5x_1 + 2x_2}{\leq 8}
        \addConstraint{x_1, x_2}{\geq 0}{.}
    \end{maxi*}
\end{problem}
\begin{problem}{3.2$\bigstar$}
    \begin{mini*}|s|
        {}
        {2x_1 + x_2}
        {}
        {}
        \addConstraint{3x_1 + 5x_2 }{\leq 15}
        \addConstraint{6x_1 + 2x_2 }{\leq 24}
        \addConstraint{x_1, x_2}{\geq 0}{.}
    \end{mini*}
\end{problem}

\subsubsection{分别用大M法与两阶段法求解并对比}

\begin{problem}{4.1$\bigstar$}
    \begin{maxi*}|s|
        {}
        {2x_1 - x_2 - 2x_3}
        {}
        {}
        \addConstraint{x_1 + x_2 + x_3}{\geq 6}
        \addConstraint{-2x_1 \ph + x_3}{\geq 2}
        \addConstraint{ \ph 2x_2 - x_3}{\geq 0}
        \addConstraint{x_1, x_2, x_3}{\geq 0}{.}
    \end{maxi*}
\end{problem}
\begin{problem}{4.2$\bigstar$}
    \begin{mini*}|s|
        {}
        {4x_1 + x_2}
        {}
        {}
        \addConstraint{3x_1 + x_2}{=3}
        \addConstraint{4x_1 + 3x_2}{\geq 6}
        \addConstraint{x_1 + 2x_2}{\leq 4}
        \addConstraint{x_1, x_2, }{\geq 0}{.}
    \end{mini*}
\end{problem}

\subsubsection{不限方法求解}

\begin{problem}{5.1}
    \begin{mini*}|s|
        {}
        {4x_1 + 6x_2 + 18x_3}
        {}
        {}
        \addConstraint{x_1 \ph + 3x_3}{\geq 3}
        \addConstraint{\ph x_2 + 2x_3}{\geq 5}
        \addConstraint{x_1, x_2, x_3}{\geq 0}{.}
    \end{mini*}
\end{problem}
\begin{problem}{5.2}
    \begin{maxi*}|s|
        {}
        {2x_1 + x_2}
        {}
        {}
        \addConstraint{x_1 + x_2}{\leq 5}
        \addConstraint{x_1 - x_2}{\geq 0}
        \addConstraint{6x_1 + 2x_2}{\leq 21}
        \addConstraint{x_1, x_2}{\geq 0}{.}
    \end{maxi*}
\end{problem}
\begin{problem}{5.3}
    \begin{maxi*}|s|
        {}
        {3x_1 - 5x_2}
        {}
        {}
        \addConstraint{-x_1 + 2x_2 + 4x_3}{\leq 4}
        \addConstraint{x_1 + x_2 + 2x_3}{\leq 5}
        \addConstraint{-x_1 + 2x_2 + x_3}{\geq 1}
        \addConstraint{x_1, x_2, x_3}{\geq 0}{.}
    \end{maxi*}
\end{problem}
\begin{problem}{5.4}
    \begin{mini*}|s|
        {}
        {x_1 - 3x_2 + x_3}
        {}
        {}
        \addConstraint{2x_1 - x_2 + x_3}{=8}
        \addConstraint{2x_1 + x_2 \ph}{\geq 2}
        \addConstraint{x_1 + 2x_2 \ph}{\leq 10}
        \addConstraint{x_1, x_2, x_3}{\geq 0}{.}
    \end{mini*}
\end{problem}
\begin{problem}{5.5}
    \begin{maxi*}|s|
        {}
        {-3x_1 + x_2 - x_3}
        {}
        {}
        \addConstraint{2x_1 + x_2 - x_3}{\leq 5}
        \addConstraint{4x_1 + 3x_2 + x_3}{\geq 3}
        \addConstraint{-x_1 + x_2 + x_3}{=2}
        \addConstraint{x_1, x_2, x_3}{\geq 0}{.}
    \end{maxi*}
\end{problem}
\begin{problem}{5.6}
    \begin{mini*}|s|
        {}
        {2x_1 - 3x_2 + 4x_3}
        {}
        {}
        \addConstraint{x_1 + x_2 + x_3}{\leq 9}
        \addConstraint{-x_1 + 2x_2 - x_3}{\geq 5}
        \addConstraint{2x_1 - x_2 \ph}{\leq 7}
        \addConstraint{x_1, x_2, x_3}{\geq 0}{.}
    \end{mini*}
\end{problem}
\begin{problem}{5.7}
    \begin{mini*}|s|
        {}
        {3x_1 - 2x_2 + x_3}
        {}
        {}
        \addConstraint{2x_1 - 3x_2 + x_3}{=1}
        \addConstraint{2x_1 + 3x_2 \ph}{\geq 8}
        \addConstraint{x_1, x_2, x_3}{\geq 0}{.}
    \end{mini*}
\end{problem}
\begin{problem}{5.8}
    \begin{mini*}|s|
        {}
        {2x_1 - 3x_2}
        {}
        {}
        \addConstraint{2x_1 - x_2 - x_3}{\geq 3}
        \addConstraint{x_1 - x_2 + x_3}{\geq 2}
        \addConstraint{x_1, x_2, x_3}{\geq 0}{.}
    \end{mini*}
\end{problem}
\begin{problem}{5.9}
    \begin{mini*}|s|
        {}
        {2x_1 + x_2 - x_3 - x_4}
        {}
        {}
        \addConstraint{x_1 - x_2 + 2x_3 - x_4}{=2}
        \addConstraint{2x_1 + x_2 - 3x_3 + x_4}{=6}
        \addConstraint{x_1 + x_2 + x_3 + x_4}{=7}
        \addConstraint{x_j}{\geq 0}{,\ j=1,\ldots,4.}
    \end{mini*}
\end{problem}
\begin{problem}{5.10}
    \begin{maxi*}|s|
        {}
        {3x_1 - x_2 - 3x_3 + x_4}
        {}
        {}
        \addConstraint{x_1 + 2x_2 - x_3 + x_4}{=0}
        \addConstraint{x_1 - x_2 + 2x_3 - x_4}{=6}
        \addConstraint{2x_1 - 2x_2 + 3x_3 + 3x_4}{=9}
        \addConstraint{x_j}{\geq 0}{,\ j=1,\ldots,4.}
    \end{maxi*}
\end{problem}

\subsection{对偶理论}

\subsubsection{对偶理论基础}

\begin{problem}{1.1$\bigstar$}
    已知线性规划问题
    \begin{mini*}|s|
        {}
        {8x_1 + 6x_2 + 3x_3 + 6x_4}
        {}
        {}
        \addConstraint{x_1 + 2x_2 \ph + x_4}{\geq 3}
        \addConstraint{3x_1 + x_2 + x_3 + x_4}{\geq 6}
        \addConstraint{\ph \ph x_3 + x_4}{=2}
        \addConstraint{x_1 \ph + x_3 \ph}{\geq 2}
        \addConstraint{x_1,x_2,x_3,x_4}{\geq 0}{.}
    \end{mini*}
    \begin{enumerate}
        \item[(1)] 写出其对偶问题;
        \item[(2)] 已知原问题最优解为$X^*=(1,1,2,0)$,试根据对偶理论,直接求出对偶问题的最优解。
    \end{enumerate}
\end{problem}

\subsubsection{对偶单纯形法求解}

\begin{problem}{2.1$\bigstar$}
    \begin{mini*}|s|
        {}
        {4x_1 + 12x_2 + 18x_3}
        {}
        {}
        \addConstraint{x_1  \ph + 3x_3}{\geq 3}
        \addConstraint{\ph 2x_2 + 2x_3}{\geq 5}
        \addConstraint{x_1,x_2,x_3}{\geq 0}{.}
    \end{mini*}
\end{problem}
\begin{problem}{2.2$\bigstar$}
    \begin{mini*}|s|
        {}
        {5x_1 + 2x_2 + 4x_3}
        {}
        {}
        \addConstraint{3x_1 + x_2 + 2x_3}{\geq 4}
        \addConstraint{6x_1 + 3x_2 + 5x_3}{\geq 10}
        \addConstraint{x_1,x_2,x_3}{\geq 0}{.}
    \end{mini*}
\end{problem}
\begin{problem}{2.3}
    \begin{mini*}|s|
        {}
        {4x_1 + 6x_2 + 18x_3}
        {}
        {}
        \addConstraint{x_1  \ph + 3x_3}{\geq 3}
        \addConstraint{\ph x_2 + 2x_3}{\geq 5}
        \addConstraint{x_1,x_2,x_3}{\geq 0}{.}
    \end{mini*}
\end{problem}
\begin{problem}{2.4}
    \begin{maxi*}|s|
        {}
        {-3x_1 - 2x_2 - 4x_3 - 8x_4}
        {}
        {}
        \addConstraint{-2x_1 + 5x_2 + 3x_3 - 5x_4}{\leq 3}
        \addConstraint{x_1 + 2x_2 + 5x_3 + 6x_4}{\geq 8}
        \addConstraint{x_j}{\geq 0}{,\ j=1,\ldots,4.}
    \end{maxi*}
\end{problem}
\begin{problem}{2.5}
    \begin{maxi*}|s|
        {}
        {x_1 + x_2}
        {}
        {}
        \addConstraint{x_1 - x_2 - x_3}{=1}
        \addConstraint{-x_1 + x_2 + 2x_3}{\geq 1}
        \addConstraint{x_1,x_2,x_3}{\geq 0}{.}
    \end{maxi*}
\end{problem}
\begin{problem}{2.6}
    \begin{maxi*}|s|
        {}
        {-4x_1 + 3x_2}
        {}
        {}
        \addConstraint{4x_1 + 3x_2 + x_3 - x_4}{=32}
        \addConstraint{2x_1 + x_2 - x_3 - x_4}{=14}
        \addConstraint{x_j}{\geq 0}{,\ j=1,\ldots,4.}
    \end{maxi*}
\end{problem}
\begin{problem}{2.7}
    \begin{mini*}|s|
        {}
        {4x_1 + 3x_2 + 5x_3 + x_4 + 2x_5}
        {}
        {}
        \addConstraint{-x_1 + 2x_2 - 2x_3 + 3x_4 -3x_5 + x_6 \ph + x_8}{=1}
        \addConstraint{x_1 + x_2 - 3x_3 + 2x_4 - 2x_5 \ph \ph + x_8}{=4}
        \addConstraint{\ph \ph -2x_3 + 3x_4 - 3x_5 \ph + x_7 + x_8}{=2}
        \addConstraint{x_j}{\geq 0}{,\ j=1,\ldots,8.}
    \end{mini*}
\end{problem}




    \section{整数规划}


\subsection{割平面法求解}

\begin{problem}{1.1$\bigstar$}
    \begin{maxi*}|s|
        {}
        {x_1 + x_2}
        {}
        {}
        \addConstraint{2x_1 + x_2}{\leq 6}
        \addConstraint{4x_1 + 5x_2}{\leq 20}
        \addConstraint{x_1,x_2}{\geq 0}{,\text{且为整数}.}
    \end{maxi*}
\end{problem}
\begin{solution}
    第一步,单纯形法求解原问题
    \begin{center}
        \begin{tblr}{
                hline{5,6,8,9} = {0.08em},
                cell{5,8}{1} = {c=3,r=1}{c},
            }
            c_j \rightarrow &&& 1   & 1   & 0   & 0   \\
            C_B  & X_B  &b    & x_1 & x_2 & x_3 & x_4 \\
            0    & x_3  &6    & [2] & 1   & 1   & 0   \\
            0    & x_4  &20   & 4   & 5   & 0   & 1   \\
            c_j - z_j       &&& 1   & 1   & 0   & 0   \\
            1    & x_1  &3    & 1   & 1/2 & 1/2 & 0   \\
            0    & x_4  &8    & 0   & [3] & -2  & 1   \\
            c_j - z_j       &&& 0   & 1/2 & -1/2& 0   \\
            1    & x_1  &5/3  & 1   & 0   & 5/6 & -1/6\\
            1    & x_2  &8/3  & 0   & 1   & -2/3& 1/3 \\
            c_j - z_j       &&& 0   & 0   & -1/6& -1/6\\
        \end{tblr}
    \end{center}
    得点(5/3,8/3),非整数,选择$x_1$小数部分2/3进行割平面。
    形成约束:
    $$-5/6x_3 + 1/6x_4 \leq -2/3$$
    第二步,引入松弛变量$x_5$,继续用单纯形法求解
    \begin{center}
        \begin{tblr}{
                hline{6,7,10,11} = {0.08em},
                cell{6,10}{1} = {c=3,r=1}{c},
            }
            c_j \rightarrow &&& 1   & 1   & 0   & 0   & 0   \\
            C_B  & X_B  &b    & x_1 & x_2 & x_3 & x_4 & x_5 \\
            1    & x_1  &5/3  & 1   & 0   & 5/6 & -1/6& 0   \\
            1    & x_2  &8/3  & 0   & 1   & -2/3& 1/3 & 0   \\
            0    & x_5  &-2/3 & 0   & 0   &[-5/6]& 1/6& 1   \\
            c_j - z_j       &&& 0   & 0   & -1/6& -1/6& 0   \\
            1    & x_1  &1    & 1   & 0   & 0   & 0   & 1   \\
            1    & x_2  &16/5 & 0   & 1   & 0   & 1/5 & -4/5\\
            0    & x_3  &4/5  & 0   & 0   & 1   & -1/5& -6/5\\
            c_j - z_j       &&& 0   & 0   & 0   & -1/5& -1/5\\
        \end{tblr}
    \end{center}
    得点(1,16/5),$x_2$非整数,选择$x_2$小数部分1/5进行割平面。
    形成约束:
    $$-1/5x_4 + 4/5x_5 \leq -1/5$$
    第三步,引入松弛变量$x_6$,继续用单纯形法求解
    \begin{center}
        \begin{tblr}{
                hline{7,8,12,13} = {0.08em},
                cell{7,12}{1} = {c=3,r=1}{c},
            }
            c_j \rightarrow &&& 1   & 1   & 0   & 0   & 0   & 0   \\
            C_B  & X_B  &b    & x_1 & x_2 & x_3 & x_4 & x_5 & x_6 \\
            1    & x_1  &1    & 1   & 0   & 0   & 0   & 1   & 0   \\
            1    & x_2  &16/5 & 0   & 1   & 0   & 1/5 & -4/5& 0   \\
            0    & x_3  &4/5  & 0   & 0   & 1   & -1/5& -6/5& 0   \\
            0    & x_6  &-1/5 & 0   & 0   & 0   &[-1/5]&4/5 & 1   \\
            c_j - z_j       &&& 0   & 0   & 0   & -1/5& -1/5& 0   \\
            1    & x_1  &1    & 1   & 0   & 0   & 0   & 1   & 0   \\
            1    & x_2  &3    & 0   & 1   & 0   & 0   & 0   & 1   \\
            0    & x_3  &1    & 0   & 0   & 1   & 0   & -2  & -1   \\
            0    & x_4  &1    & 0   & 0   & 0   & 1   & -4  & -5   \\
            c_j - z_j       &&& 0   & 0   & 0   & 0   & -1  & -1   \\
        \end{tblr}
    \end{center}
    得点(1,3),整数,目标函数值$f_{max}=1+3=4$
\end{solution}
\begin{problem}{1.2$\bigstar$}
    \begin{mini*}|s|
        {}
        {5x_1 + x_2}
        {}
        {}
        \addConstraint{3x_1 + x_2}{\geq 9}
        \addConstraint{x_1 + x_2}{\geq 5}
        \addConstraint{x_1 + 8x_2}{\geq 8}
        \addConstraint{x_1,x_2}{\geq 0}{,\text{且为整数}.}
    \end{mini*}
\end{problem}

\begin{solution}
    第一步,单纯形法求解原问题
    \begin{center}
        \begin{tblr}{
                hline{6,7} = {0.08em},
                cell{6}{1} = {c=3,r=1}{c},
            }
            c_j \rightarrow &&& -5  & -1  & 0   & 0   & 0   \\
            C_B  & X_B  &b    & x_1 & x_2 & x_3 & x_4 & x_5 \\
            0    & x_3  &-9   & -3  & [-1]& 1   & 0   & 0   \\
            0    & x_4  &-5   & -1  & -1  & 0   & 1   & 0   \\
            0    & x_5  &-8   & -1  & -8  & 0   & 0   & 1   \\
            c_j - z_j       &&& -5  & -1  & 0   & 0   & 0   \\
            1    & x_2  &9    & 3   & 1   & -1  & 0   & 0   \\
            0    & x_4  &4    & 2   & 0   & 1   & 1   & 0   \\
            0    & x_5  &64   & 23  & 0   & -8  & 0   & 1   \\
            c_j - z_j       &&& -2  & 0   & -1  & 0   & 0   \\
        \end{tblr}
    \end{center}
    得点$(0,9)$,整数最优解,目标函数值$f_{min}=0+9=9$。
\end{solution}

\begin{problem}{1.3}
    \begin{mini*}|s|
        {}
        {x_1 - 2x_2}
        {}
        {}
        \addConstraint{x_1 + x_2}{\leq 10}
        \addConstraint{-x_1 + x_2}{\leq 5}
        \addConstraint{x_1,x_2}{\geq 0}{,\text{且为整数}.}
    \end{mini*}
\end{problem}

\begin{problem}{1.4}
    \begin{mini*}|s|
        {}
        {5x_1 + 3x_2}
        {}
        {}
        \addConstraint{2x_1 + x_2}{\geq 10}
        \addConstraint{x_1 + 3x_2}{\geq 9}
        \addConstraint{x_1,x_2}{\geq 0}{,\text{且为整数}.}
    \end{mini*}
\end{problem}

\subsection{分支定界法求解}

\begin{problem}{2.1$\bigstar$}
    \begin{maxi*}|s|
        {}
        {2x_1 + x_2}
        {}
        {}
        \addConstraint{x_1 + x_2}{\leq 5}
        \addConstraint{-x_1 + x_2}{\leq 0}
        \addConstraint{6x_1 + 2x_2}{\leq 21}
        \addConstraint{x_1,x_2}{\geq 0}{,\text{且为整数}.}
    \end{maxi*}
\end{problem}
\begin{solution}
    第一步,单纯形法求解原问题
    \begin{center}
        \begin{tblr}{
                hline{6,7,10,11} = {0.08em},
                cell{6,10}{1} = {c=3,r=1}{c},
            }
            c_j \rightarrow &&& 2   & 1   & 0   & 0   & 0   \\
            C_B  & X_B  &b    & x_1 & x_2 & x_3 & x_4 & x_5 \\
            0    & x_3  &5    & 1   & 1   & 1   & 0   & 0   \\
            0    & x_4  &0    & -1  & 1   & 0   & 1   & 0   \\
            0    & x_5  &21   & [6] & 2   & 0   & 0   & 1   \\
            c_j - z_j       &&& 2   & 1   & 0   & 0   & 0   \\
            0    & x_3  &3/2  & 0   &[2/3]& 1   & 0   & -1/6\\
            0    & x_4  &7/2  & 0   & 4/3 & 0   & 1   & 1/6 \\
            2    & x_1  &7/2  & 1   & 1/3 & 0   & 0   & 1/6 \\
            c_j - z_j       &&& 0   & 1/3 & 0   & 0   & -1/3\\
            1    & x_2  &9/4  & 0   & 1   & 3/2 & 0   & -1/4\\
            0    & x_4  &1/2  & 0   & 0   & -2  & 1   & 1/2 \\
            2    & x_1  &11/4 & 1   & 0   & -1/2& 0   & 1/4 \\
            c_j - z_j       &&& 0   & 0   & 0   & 0   & -1/4\\
        \end{tblr}
    \end{center}
    得点$(\frac{11}{4},\frac{9}{4})$,非整数,设下界为$\frac{31}{4}$,对原问题的$x_1$进行分支
    \begin{maxi*}|s|
        {}
        {2x_1 + x_2}
        {}
        {(1)}
        \addConstraint{x_1 + x_2}{\leq 5}
        \addConstraint{-x_1 + x_2}{\leq 0}
        \addConstraint{6x_1 + 2x_2}{\leq 21}
        \addConstraint{x_1}{\geq 3}
        \addConstraint{x_1,x_2}{\geq 0}{,\text{且为整数}.}
    \end{maxi*}
    \begin{maxi*}|s|
        {}
        {2x_1 + x_2}
        {}
        {(2)}
        \addConstraint{x_1 + x_2}{\leq 5}
        \addConstraint{-x_1 + x_2}{\leq 0}
        \addConstraint{6x_1 + 2x_2}{\leq 21}
        \addConstraint{x_1}{\leq 2}
        \addConstraint{x_1,x_2}{\geq 0}{,\text{且为整数}.}
    \end{maxi*}
    第二步,解问题(1)
    \begin{center}
        \begin{tblr}{
                hline{7,8,12,13,17,18} = {0.08em},
                cell{7,12,17}{1} = {c=3,r=1}{c},
            }
            c_j \rightarrow &&& 2   & 1   & 0   & 0   & 0   & 0   \\
            C_B  & X_B  &b    & x_1 & x_2 & x_3 & x_4 & x_5 & x_6 \\
            0    & x_3  &5    & 1   & 1   & 1   & 0   & 0   & 0   \\
            0    & x_4  &0    & -1  & 1   & 0   & 1   & 0   & 0   \\
            0    & x_5  &21   & 6   & 2   & 0   & 0   & 1   & 0   \\
            0    & x_6  &-3   & [-1]& 0   & 0   & 0   & 0   & 1   \\
            c_j - z_j       &&& 2   & 1   & 0   & 0   & 0   & 0   \\
            0    & x_3  &2    & 0   & 1   & 1   & 0   & 0   & 1   \\
            0    & x_4  &3    & 0   & 1   & 0   & 1   & 0   & -1  \\
            0    & x_5  &3    & 0   & 2   & 0   & 0   & 1   & [6] \\
            2    & x_1  &3    & 1   & 0   & 0   & 0   & 0   & -1  \\
            c_j - z_j       &&& 0   & 1   & 0   & 0   & 0   & 2   \\
            0    & x_3  &3/2  & 0   & 2/3 & 1   & 0   & -1/6& 0   \\
            0    & x_4  &7/2  & 0   & 4/3 & 0   & 1   & 1/6 & 0   \\
            0    & x_6  &1/2  & 0   &[1/3]& 0   & 0   & 1/6 & 1   \\
            2    & x_1  &7/2  & 1   & 1/3 & 0   & 0   & 1/6 & 0   \\
            c_j - z_j       &&& 0   & 1/3 & 0   & 0   & -1/3& 0   \\
            0    & x_3  &1/2  & 0   & 0   & 1   & 0   & -1/2& -2  \\
            0    & x_4  &3/2  & 0   & 0   & 0   & 1   & -1/2& -4  \\
            1    & x_2  &3/2  & 0   & 1   & 0   & 0   & 1/2 & 3   \\
            2    & x_1  &3    & 1   & 0   & 0   & 0   & 0   & -1  \\
            c_j - z_j       &&& 0   & 1/3 & 0   & 0   & -1/2& -1  \\
        \end{tblr}
    \end{center}
    得点$(3,\frac{3}{2})$,非整数,继续分支
        \begin{maxi*}|s|
        {}
        {2x_1 + x_2}
        {}
        {(1.1)}
        \addConstraint{x_1 + x_2}{\leq 5}
        \addConstraint{-x_1 + x_2}{\leq 0}
        \addConstraint{6x_1 + 2x_2}{\leq 21}
        \addConstraint{x_1}{\geq 3}
        \addConstraint{x_2}{\geq 2}
        \addConstraint{x_1,x_2}{\geq 0}{,\text{且为整数}.}
    \end{maxi*}
    \begin{maxi*}|s|
        {}
        {2x_1 + x_2}
        {}
        {(1.2)}
        \addConstraint{x_1 + x_2}{\leq 5}
        \addConstraint{-x_1 + x_2}{\leq 0}
        \addConstraint{6x_1 + 2x_2}{\leq 21}
        \addConstraint{x_1}{\geq 3}
        \addConstraint{x_2}{\leq 1}
        \addConstraint{x_1,x_2}{\geq 0}{,\text{且为整数}.}
    \end{maxi*}

    第三步,解问题(1.1),易见第三约束不满足,无可行解;解问题(1.2),
    \begin{center}
        \begin{tblr}{
                hline{8,9,14,15,20,21} = {0.08em},
                cell{8,14,20}{1} = {c=3,r=1}{c},
            }
            c_j \rightarrow &&& 2   & 1   & 0   & 0   & 0   & 0   & 0   \\
            C_B  & X_B  &b    & x_1 & x_2 & x_3 & x_4 & x_5 & x_6 & x_7 \\
            0    & x_3  &3/2  & 0   & 2/3 & 1   & 0   & -1/6& 0   & 0   \\
            0    & x_4  &7/2  & 0   & 4/3 & 0   & 1   & 1/6 & 0   & 0   \\
            0    & x_6  &1/2  & 0   & 1/3 & 0   & 0   & 1/6 & 1   & 0   \\
            2    & x_1  &7/2  & 1   & 1/3 & 0   & 0   & 1/6 & 0   & 0   \\
            0    & x_7  &1    & 0   & [1] & 0   & 0   & 0   & 0   & 1   \\
            c_j - z_j       &&& 0   & 1/3 & 0   & 0   & -1/3& 0   & 0   \\
            0    & x_3  &1/2  & 0   & 0   & 1   & 0   & -1/2& 2   & 0   \\
            0    & x_4  &3/2  & 0   & 0   & 0   & 1   & -1/2& 4   & 0   \\
            0    & x_6  &1/2  & 0   & 0   & 0   & 0   & 1/2 & -3  & -1  \\
            2    & x_1  &3    & 1   & 0   & 0   & 0   & 0   & 1   & 0   \\
            1    & x_7  &1    & 0   & 1   & 0   & 0   & 0   & 0   & 1   \\
            c_j - z_j       &&& 0   & 0   & 0   & 0   & 0   & -2  & 0   \\
        \end{tblr}
    \end{center}
    得解(3,1),整数最优解,目标函数值为7,下界改设为7;\\
    第四步,继续解问题(2)
    \begin{center}
        \begin{tblr}{
                hline{7,8,12,13,17,18} = {0.08em},
                cell{7,12,17}{1} = {c=3,r=1}{c},
            }
            c_j \rightarrow &&& 2   & 1   & 0   & 0   & 0   & 0   \\
            C_B  & X_B  &b    & x_1 & x_2 & x_3 & x_4 & x_5 & x_6 \\
            0    & x_3  &5    & 1   & 1   & 1   & 0   & 0   & 0   \\
            0    & x_4  &0    & -1  & 1   & 0   & 1   & 0   & 0   \\
            0    & x_5  &21   & 6   & 2   & 0   & 0   & 1   & 0   \\
            0    & x_6  &2    & [1] & 0   & 0   & 0   & 0   & 1   \\
            c_j - z_j       &&& 2   & 1   & 0   & 0   & 0   & 0   \\
            0    & x_3  &3    & 0   & 1   & 1   & 0   & 0   & -1  \\
            0    & x_4  &2    & 0   & [1] & 0   & 1   & 0   & 1   \\
            0    & x_5  &9    & 0   & 2   & 0   & 0   & 1   & 6   \\
            2    & x_1  &2    & 1   & 0   & 0   & 0   & 0   & 1   \\
            c_j - z_j       &&& 0   & 1   & 0   & 0   & 0   & -2  \\
            0    & x_3  &1    & 0   & 0   & 1   & -1  & 0   & -2  \\
            1    & x_2  &2    & 0   & 1   & 0   & 1   & 0   & 1   \\
            0    & x_5  &5    & 0   & 0   & 0   & -2  & 1   & 4   \\
            2    & x_1  &2    & 1   & 0   & 0   & 0   & 0   & 1   \\
            c_j - z_j       &&& 0   & 0   & 0   & -1  & 0   & -3  \\
        \end{tblr}
    \end{center}
    得点$(2,2)$,整数最优解,目标函数值7.\\
    综上所述,原问题最优解为(3,1)与(2,2),目标函数值$f_{max}=7$
\end{solution}

\begin{problem}{2.2$\bigstar$}
    \begin{mini*}|s|
        {}
        {5x_1 - x_2}
        {}
        {}
        \addConstraint{3x_1 + 10x_2}{\leq 50}
        \addConstraint{7x_1 - 2x_2}{\leq 28}
        \addConstraint{x_1,x_2}{\geq 0}{,x_2\text{为整数}.}
    \end{mini*}
\end{problem}
\begin{solution}
    第一步,单纯形法求解原问题
    \begin{center}
        \begin{tblr}{
                hline{5,6} = {0.08em},
                cell{5}{1} = {c=3,r=1}{c},
            }
            c_j \rightarrow &&& -5  & 1   & 0   & 0   \\
            C_B  & X_B  &b    & x_1 & x_2 & x_3 & x_4 \\
            0    & x_3  &50   & 3   & [10]& 1   & 0   \\
            0    & x_4  &28   & 7   & -2  & 0   & 1   \\
            c_j - z_j       &&& -5  & 1   & 0   & 0   \\
            1    & x_2  &5    & 3/10& 1   & 1/10& 0   \\
            0    & x_4  &38   & 38/5& 0   & 1/5 & 1   \\
            c_j - z_j       &&&-53/10& 0  &-1/10& 0   \\
        \end{tblr}
    \end{center}
    得点$(0,5)$,整数,原问题得解,目标函数值$f_{min}=5\times0-5=-5$

\end{solution}

\begin{problem}{2.3}
    \begin{mini*}|s|
        {}
        {2x_1 + x_2 - 3x_3}
        {}
        {}
        \addConstraint{x_1 + x_2 + 2x3}{\leq 5}
        \addConstraint{2x_1 + 2x_2 - x_3}{\leq 1}
        \addConstraint{x_1,x_2,x_3}{\geq 0}{,\text{且为整数}.}
    \end{mini*}
\end{problem}
\begin{solution}

\end{solution}

\begin{problem}{2.4}
    \begin{mini*}|s|
        {}
        {4x_1 + 7x_2 + 3x_3}
        {}
        {}
        \addConstraint{x_1 + 3x_2 + x_3}{\geq 5}
        \addConstraint{3x_1 + x_2 + 2x_3}{\geq 8}
        \addConstraint{x_1,x_2,x_3}{\geq 0}{,\text{且为整数}.}
    \end{mini*}
\end{problem}
\begin{solution}

\end{solution}

\subsection{最小指派求解}

\begin{problem}{3.1$\bigstar$}
    $$\begin{bmatrix}
        10 & 11 & 4  & 2  & 8 \\
        7  & 11 & 10 & 14 & 12\\
        5  & 6  & 9  & 12 & 14\\
        13 & 15 & 11 & 10 & 7
    \end{bmatrix}$$
\end{problem}
\begin{solution}
    为原问题增加一行,利用匈牙利算法求解
    $$\begin{bmatrix}
        10 & 11 & 4  & 2  & 8 \\
        7  & 11 & 10 & 14 & 12\\
        5  & 6  & 9  & 12 & 14\\
        13 & 15 & 11 & 10 & 7 \\
        0  & 0  & 0  & 0  & 0 \\
    \end{bmatrix}
    \xrightarrow{\text{各行减去最小元素}}
    \begin{bmatrix}
        8  & 9  & 2  & 0  & 6 \\
        0  & 4  & 3  & 7  & 5 \\
        0  & 1  & 4  & 7  & 9 \\
        6  & 8  & 4  & 3  & 0 \\
        0  & 0  & 0  & 0  & 0 \\
    \end{bmatrix}
    \xrightarrow{\text{第2、3列减去最小元素}}
    \begin{bmatrix}
        8  & 8  & 0  & 0  & 6 \\
        0  & 3  & 1  & 7  & 5 \\
        0  & 0  & 2  & 7  & 9 \\
        6  & 7  & 2  & 3  & 0 \\
        0  & -1 & -2 & 0  & 0 \\
    \end{bmatrix}$$
    $
    \xrightarrow{\text{第5行去除负数}}
    \begin{bmatrix}
        8  & 8  & 0  & [0]& 6 \\
        [0]& 3  & 1  & 7  & 5 \\
        0  & [0]& 2  & 7  & 9 \\
        6  & 7  & 2  & 3  & [0]\\
        2  & 1  & [0]& 2  & 2 \\
    \end{bmatrix}$
    故,最优解
    $X=\begin{bmatrix}
        0  & 0  & 0  & 1  & 0 \\
        1  & 0  & 0  & 0  & 0 \\
        0  & 1  & 0  & 0  & 0 \\
        0  & 0  & 0  & 0  & 1 \\
        0  & 0  & 1  & 0  & 0 \\
    \end{bmatrix}$
\end{solution}
\begin{problem}{3.2$\bigstar$}
    $$\begin{bmatrix}
        15 & 18 & 21 & 24\\
        19 & 23 & 22 & 18\\
        26 & 17 & 16 & 19\\
        19 & 21 & 23 & 17
    \end{bmatrix}$$
\end{problem}
\begin{solution}
    用匈牙利算法求解
    $$\begin{bmatrix}
        15 & 18 & 21 & 24\\
        19 & 23 & 22 & 18\\
        26 & 17 & 16 & 19\\
        19 & 21 & 23 & 17
    \end{bmatrix}
    \xrightarrow{\text{各行减去最小元素}}
    \begin{bmatrix}
        0  & 3  & 6  & 9 \\
        1  & 5  & 4  & 0 \\
        10 & 1  & 0  & 3 \\
        2  & 4  & 6  & 0
    \end{bmatrix}
    \xrightarrow{\text{第2、4行减去最小元素}}
    \begin{bmatrix}
        0  & 3  & 6  & 9 \\
        0  & 4  & 3  & -1\\
        10 & 1  & 0  & 3 \\
        0  & 2  & 4  & -2
    \end{bmatrix}$$
    $$
    \xrightarrow{\text{第4列去除负数}}
    \begin{bmatrix}
        0  & 3  & 6  & 11\\
        0  & 4  & 3  & 1 \\
        10 & 1  & 0  & 5 \\
        0  & 2  & 4  & 0
    \end{bmatrix}
    \xrightarrow{\text{第1、2行减去最小元素}}
    \begin{bmatrix}
        -3 & 0  & 3  & 8 \\
        -1 & 3  & 2  & 0 \\
        10 & 1  & 0  & 5 \\
        0  & 2  & 4  & 0
    \end{bmatrix}
    \xrightarrow{\text{第1列去除负数}}
    \begin{bmatrix}
        0  & 0  & 3  & 8 \\
        2  & 3  & 2  & 0 \\
        13 & 1  & 0  & 5 \\
        3  & 2  & 4  & 0
    \end{bmatrix}$$
    $$
    \xrightarrow{\text{第2、4行减去最小元素}}
    \begin{bmatrix}
        0  & 0  & 3  & 8 \\
        0  & 1  & 0  & -2\\
        13 & 1  & 0  & 5 \\
        1  & 0  & 2  & -2
    \end{bmatrix}
    \xrightarrow{\text{第4列去除负数}}
    \begin{bmatrix}
        0  & 0  & 3  & 10\\
        0  & 1  & 0  & 0 \\
        13 & 1  & 0  & 7 \\
        1  & 0  & 2  & 0
    \end{bmatrix}$$
    故,最优解
    $X^{*}=\begin{bmatrix}
        1  & 0  & 0  & 0  \\
        0  & 0  & 0  & 1  \\
        0  & 0  & 1  & 0  \\
        0  & 1  & 0  & 0  \\
    \end{bmatrix}$
    或
    $X^{**}=\begin{bmatrix}
        0  & 1  & 0  & 0  \\
        1  & 0  & 0  & 0  \\
        0  & 0  & 1  & 0  \\
        0  & 0  & 0  & 1  \\
    \end{bmatrix}$
    最小值为70.
\end{solution}
%    \section{非线性规划问题}

\subsection{凸函数与凸规划}

\subsubsection{凸规划判定}

\begin{problem}{1.1$\bigstar$}
    \begin{mini*}|s|
        {}
        {f(x) = 2x_1^2 + x_2^2 + x_3^2}
        {}
        {}
        \addConstraint{-x_1^2 - x_2^2 + 4}{\geq 0}
        \addConstraint{5x_1 - 4x_2 \ph}{=8}
        \addConstraint{x_1,x_2,x_3}{\geq 0}{.}
    \end{mini*}
\end{problem}
\begin{solution}
    第二、三条件为线性函数,把它们看成凹函数,\\
    第一约束条件的海赛矩阵是$\nabla^2g_1(x)=\begin{bmatrix}
        -2  & 0  \\
        0  & -2  \\
    \end{bmatrix}$,负定,凹函数,\\
    目标函数的海赛矩阵是$\nabla^2g_1(x)=\begin{bmatrix}
        4  & 0 & 0  \\
        0  & 2 & 0  \\
        0  & 0 & 2  \\
    \end{bmatrix}$,正定,为凸函数,\\
    从而可知该问题是凸规划,有唯一最小值解。
\end{solution}
\begin{problem}{1.2$\bigstar$}
    \begin{mini*}|s|
        {}
        {f(x) = x_1 + 3x_2}
        {}
        {}
        \addConstraint{x_1^2 + x_2^2}{\leq 9}
        \addConstraint{x_2}{\geq 0}{.}
    \end{mini*}
\end{problem}
\begin{solution}

    目标函数是线性看书,把其看为凸函数;\\
    第一约束条件两边同乘-1,改为大于等于号后的海赛矩阵是
    $\nabla^2g_1(x)=\begin{bmatrix}
        -2  & 0  \\
        0  & -2  \\
    \end{bmatrix}$,负定,凹函数,\\
    第二约束条件为线性函数,把其看成凹函数\\
    从而可知该问题是凸规划,有唯一最小值解。
\end{solution}

\subsection{最优性条件}


\subsubsection{Kuhn-Tucker点判定}

\begin{problem}{1.1$\bigstar$}
    \begin{mini*}|s|
        {}
        {f(x) = (x_1 - 3)^2 + (x_2 - 2)^2}
        {}
        {}
        \addConstraint{x_1^2 + x_2^2}{\leq 5}
        \addConstraint{x_1 + 2x_2}{=4}
        \addConstraint{x_1,x_2}{\geq 0}{.}
    \end{mini*}
    $$\bar{x}=[2\ 1]^T$$
\end{problem}
\begin{solution}
    令
    \begin{align*}
        g_1(x)&=-x_1^2 - x_2^2 + 5\geq0,\\
        g_2(x)&=x_1\geq0\\
        g_3(x)&=x_2\geq0\\
        h(x)&=x_1 + 2x_2-4=0
    \end{align*}
    将$\bar{x}=[2\ 1]^T$代入$g(x)$,易见$g_1(x)$等式成立,$g_2(x)、g_3(x)$不等式成立,则有$I(\bar{x})=\{1\}$,\\
    $\nabla f(\bar{x})=[2x_1-6\ 2x_2-4]^T=[-2\ -2]^T$,\\
    $\nabla g_1(\bar{x})=[-2x_1\ -2x_2]^T=[-4\ -2]^T$,\\
    $\nabla g_2(\bar{x})=[1\ 0]^T$,\\
    $\nabla g_3(\bar{x})=[0\ 1]^T$,\\
    $\nabla h(\bar{x})=[1\ 2]^T$,\\
    根据Kuhn-Tucker一阶必要条件,则有
    $$\left\{
    \begin{aligned}
        -2 - (-4)\lambda_1 - \mu_1 &=0\\
        -2 - (-2)\lambda_1 - 2\mu_1 &=0\\
        \lambda_1&\geq0\\
        \lambda_2=\lambda_3&=0
    \end{aligned}\right.$$
    解得$\lambda_1=\frac{1}{3},\mu_1=-\frac{2}{3}$,故点$[2\ 1]^T$是原问题Kuhn-Tucker点。
\end{solution}

\begin{problem}{1.2$\bigstar$}
    \begin{mini*}|s|
        {}
        {f(x) = (x_1 - 2)^2 + x_2^2}
        {}
        {}
        \addConstraint{x_1 - x_2^2 }{\geq 0}
        \addConstraint{-x_1 + x_2}{\geq 0}
    \end{mini*}
    $$\bar{x}^{(1)}=[0\ ,0]^T,\bar{x}^{(2)}=[1\ 1]^T$$
\end{problem}
\begin{solution}
    令
    \begin{align*}
        g_1(x)&=x_1 - x_2^2 \geq 0\\
        g_2(x)&=-x_1 + x_2 \geq 0\\
    \end{align*}
    (1)先验证$\bar{x}^{(1)}=[0\ ,0]^T$\\
    将$\bar{x}^{(1)}=[0\ 0]^T$代入$g(x)$,易见$g_1(x),g_2(x)$等式成立,则有$I(\bar{x}^{(1)})=\{1,2\}$,\\
    $\nabla f(\bar{x}^{(1)})=[2x_1-4\ 2x_2]^T=[-4\ 0]^T$,\\
    $\nabla g_1(\bar{x}^{(1)})=[1\ -2x_2]^T=[1\ 0]^T$,\\
    $\nabla g_2(\bar{x}^{(1)})=[-1\ 1]^T$,\\
    根据Kuhn-Tucker一阶必要条件,则有
    $$\left\{
    \begin{aligned}
        -4 - \lambda_1 - (-1)\lambda_2 &=0\\
        0 \ph - \lambda_2 &=0\\
        \lambda_1&\geq0\\
        \lambda_2&\geq0
    \end{aligned}\right.$$
    解得$\lambda_1=4,\lambda_2=0$,故点$[0\ 0]^T$是原问题Kuhn-Tucker点。\\
    (2)再验证$\bar{x}^{(2)}=[1\ 1]^T$\\
    将$\bar{x}^{(2)}=[1\ 1]^T$代入$g(x)$,易见$g_1(x),g_2(x)$等式成立,则有$I(\bar{x}^{(2)})=\{1,2\}$,\\
    $\nabla f(\bar{x}^{(2)})=[2x_1-4\ 2x_2]^T=[-2\ 2]^T$,\\
    $\nabla g_1(\bar{x}^{(2)})=[1\ -2x_2]^T=[1\ -2]^T$,\\
    $\nabla g_2(\bar{x}^{(2)})=[-1\ 1]^T$,\\
    根据Kuhn-Tucker一阶必要条件,则有
    $$\left\{
    \begin{aligned}
        -2 - \lambda_1 - (-1)\lambda_2 &=0\\
        2 - (-2)\lambda_1 - \lambda_2 &=0\\
        \lambda_1&\geq0\\
        \lambda_2&\geq0
    \end{aligned}\right.$$
    解得$\lambda_1=0,\lambda_2=2$,故点$[1\ 1]^T$是原问题Kuhn-Tucker点。
\end{solution}

\subsubsection{Kuhn-Tucker点求解,并验证是否最优解}

\begin{problem}{2.1$\bigstar$}
    \begin{mini*}|s|
        {}
        {f(x) = (x_1 - 1)^2 + (x_1 - 2)^2}
        {}
        {}
        \addConstraint{-x_1 + x_2}{= 1}
        \addConstraint{x_1 + x_2}{= 2}
        \addConstraint{x_1,x_2}{\geq 0}{.}
    \end{mini*}
\end{problem}
\begin{solution}
    令
    \begin{align*}
        g_1(x)&=x_1\geq0\\
        g_2(x)&=x_2\geq0\\
        h_1(x)&=-x_1 + x_2 - 1 =0\\
        h_2(x)&=x_1 + x_2 - 2 = 0\\
    \end{align*}
    则有
    \begin{align*}
        \nabla f(x)&=[2x_1-2\ 2x_2-4]^T\\
        \nabla g_1(x)&=[1\ 0]^T\\
        \nabla g_2(x)&=[0\ 1]^T\\
        \nabla h_1(x)&=[-1\ 1]^T\\
        \nabla h_2(x)&=[1\ 1]^T\\
    \end{align*}
    根据Kuhn-Tucker一阶必要条件,则有
    $$\left\{
    \begin{aligned}
        2x_1 - 2 - \lambda_1 - (-1)\mu_1 - \mu_2 &=0\\
        2x_2 - 4 - \lambda_2 - \mu_1 - \mu_2 &=0\\
        \lambda_1(x_1)&=0\\
        \lambda_2(x_2)&=0\\
        \lambda_1&\geq0\\
        \lambda_2&\geq0
    \end{aligned}\right.$$
    解得$\lambda_1=4,\lambda_2=0$,故点$[0\ 0]^T$是原问题Kuhn-Tucker点。
\end{solution}

\begin{problem}{2.2$\bigstar$}
    \begin{mini*}|s|
        {}
        {f(x) = x_1^2 - x_2 - 3x_3}
        {}
        {}
        \addConstraint{-x_1 - x_2 - x_3}{\geq 0}
        \addConstraint{x_1^2 + 2x_2 - x_3}{=0}
    \end{mini*}
\end{problem}
\begin{solution}
    令
    \begin{align*}
        g(x)&=-x_1 - x_2 - x_3\geq0\\
        h(x)&=x_1^2 + 2x_2 - x_3 =0\\
    \end{align*}
    则有
    \begin{align*}
        \nabla f(x)&=[2x_1\ -1\ -3]^T\\
        \nabla g(x)&=[-1\ -1\ -1]^T\\
        \nabla h(x)&=[2x_1\ 2\ -1]^T\\
    \end{align*}
    根据Kuhn-Tucker一阶必要条件,则有
    $$\left\{
    \begin{aligned}
        2x_1 - (-1)\lambda - (2x_1)\mu &=0\\
        -1 - (-1)\lambda - 2\mu  &=0\\
        -3 - (-1)\lambda - (-1)\mu  &=0\\
        \lambda(-x_1 - x_2 - x_3)&=0\\
        \lambda&\geq0
    \end{aligned}\right.$$
    解得$\lambda=\frac{7}{3},\mu=\frac{2}{3},x_1=-\frac{7}{2}$。
    再将$x_1$代入下式求$x_2$与$x_3$
    $$\left\{
    \begin{aligned}
        -x_1 - x_2 - x_3 &=0\\
        x_1^2 + 2x_2 - x_3&=0\\
        \lambda_1&\geq0
    \end{aligned}\right.$$
    解得$[x_1\ x_2\ x_3]^T=[-\frac{7}{2}\  -\frac{35}{12}\ \frac{77}{12}]^T$
    考虑拉格朗日函数
    \begin{align*}
        L(x,\lambda,\mu)&=f(x)-\lambda g(x)-\mu h(x)\\
        &=x_1^2 - x_2 - 3x_3 - \frac{7}{3}(-x_1 - x_2 - x_3) - \frac{2}{3}(x_1^2 + 2x_2 - x_3)\\
        &=\frac{1}{3}x_1^2+\frac{7}{3}x_1\\
        \nabla^2 L(x,\lambda,\mu)&=\begin{bmatrix}
            \frac{2}{3}  & 0 & 0  \\
            0  & 0 & 0  \\
            0  & 0 & 0  \\
        \end{bmatrix}\\
    \end{align*}
    海赛矩阵半正定,故点$[-\frac{7}{2}\  -\frac{35}{12}\ \frac{77}{12}]^T$是最优解

\end{solution}

\subsubsection{Fritz-John点判定}

\begin{problem}{3.1$\bigstar$}
    \begin{mini*}|s|
        {}
        {f(x) = -x_2}
        {}
        {}
        \addConstraint{-2x_1 + (2 - x_2)^3}{\geq 0}
        \addConstraint{x_1 \ph\ph}{\geq 0.}
    \end{mini*}
    $$\bar{x}=[0\ 2]^T$$
\end{problem}
\begin{solution}
    令
    \begin{align*}
        g_1(x)&=-2x_1 + (2 - x_2)^3\geq 0\\
        g_2(x)&=x_1\geq0\\
    \end{align*}
    将$\bar{x}=[0\ 2]^T$代入$g(x)$,易见$g_1(x),g_2(x)$等式成立,则有$I(\bar{x})=\{1,2\}$,\\
    $\nabla f(\bar{x})=[0\ -1]^T$,\\
    $\nabla g_1(\bar{x})=[-2\ -3x_2^2+12x_2]^T=[-2\ 12]^T$,\\
    $\nabla g_2(\bar{x})=[1\ 0]^T$,\\
    根据Fritz-John一阶必要条件,则有
    $$\left\{
    \begin{aligned}
        (0)\lambda_0 - (-2)\lambda_1 - \lambda_2 &=0\\
        (-1)\lambda_0 - 12\lambda_1 - (0)\lambda_2 &=0\\
    \end{aligned}\right.$$
    该方程组中$\lambda_0$与$\lambda_1$异号,找不到非负解,故点$[0\ 2]^T$不是原问题Fritz-John点。
\end{solution}
\begin{problem}{3.2$\bigstar$}
    \begin{mini*}|s|
        {}
        {f(x) = -x_1}
        {}
        {}
        \addConstraint{(1 - x_1)^2 + x_2}{\geq 0}
        \addConstraint{-x_2}{\geq 0}{.}
    \end{mini*}
    $$\bar{x}=[1\ 0]^T$$
\end{problem}
\begin{solution}
    令
    \begin{align*}
        g_1(x)&=(1 - x_1)^2 + x_2\geq 0\\
        g_2(x)&=x_2\geq0\\
    \end{align*}
    将$\bar{x}=[1\ 0]^T$代入$g(x)$,易见$g_1(x),g_2(x)$等式成立,则有$I(\bar{x})=\{1,2\}$,\\
    $\nabla f(\bar{x})=[-1\ 0]^T$,\\
    $\nabla g_1(\bar{x})=[2x_1-2\ 1]^T=[0\ 1]^T$,\\
    $\nabla g_2(\bar{x})=[0\ 1]^T$,\\
    根据Fritz-John一阶必要条件,则有
    $$\left\{
    \begin{aligned}
        (-1)\lambda_0 - (0)\lambda_1 - (0)\lambda_2 &=0\\
        (0)\lambda_0 - \lambda_1 - \lambda_2 &=0\\
    \end{aligned}\right.$$
    该方程组中有无穷非负解,如$[0\ 1\ 1]$,故点$[1\ 0]^T$是原问题Fritz-John点。
\end{solution}

\subsection{一维搜索算法}

\subsubsection{黄金分割法}

\begin{problem}{1.1$\bigstar$}
    $f(x)=x^3-3x+2$,区间为$[0,3]$
\end{problem}
\begin{solution}
    $f'(x)=3x^2-3=0$,解得x=1或-1,仅1在[0,3]区间内,故f(x)在[0,3]上仅一个驻点,是[0,3]上的单峰函数。
    采用黄金分割法,迭代计算
    \begin{flalign*}
        &\text{第1次迭代:}\\
        &a_1=0,b_1=3,\lambda_1=0.382\times3=1.146,\mu_1=0.618\times3=1.854\\
        &f(\lambda_1)=(1.146)^3-3\times1.146+2=0.067\\
        &f(\mu_1)=(1.854)^3-3\times1.854+2=2.811\\
        &f(\lambda_1)<f(\mu_1)\\
        &\text{第2次迭代:}\\
        &a_2=0,b_2=\mu_1=1.854,\lambda_2=0.382\times1.854=0.708,\mu_2=\lambda_1=1.146\\
        &f(\lambda_2)=(0.708)^3-3\times0.708+2=0.231\\
        &f(\mu_2)=f(\lambda_1)=0.067\\
        &f(\lambda_2)>f(\mu_2)\\
        &\text{第3次迭代:}\\
        &a_3=\lambda_2=0.708,b_3=\mu_1=1.854,\lambda_3=\mu_2=1.146,\mu_3=0.618\times(1.854-0.708)=1.416\\
        &f(\lambda_3)=f(\mu_2)=0.067\\
        &f(\mu_3)=(1.416)^3-3\times1.416+2=0.591\\
        &f(\lambda_3)<f(\mu_3)\\
        &\text{第4次迭代:}\\
        &a_4=0.708,b_4=\mu_3=1.416,\lambda_4=0.382\times(1.416-0.708)=0.978,\mu_4=\lambda_3=1.146\\
        &f(\lambda_4)=(0.978)^3-3\times0.978+2=0.001\\
        &f(\mu_4)=f(\lambda_3)=0.067\\
        &f(\lambda_4)<f(\mu_4)\\
        &\text{第5次迭代:}\\
        &a_5=0.708,b_5=\mu_4=1.146,\lambda_5=0.382\times(1.146-0.708)=0.875,\mu_5=\lambda_4=0.978\\
        &f(\lambda_5)=(0.875)^3-3\times0.875+2=0.045\\
        &f(\mu_5)=f(\lambda_4)=0.001\\
        &f(\lambda_5)>f(\mu_5)\\
        &\text{第6次迭代:}\\
        &a_6=\lambda_5=0.875,b_6=1.146,\lambda_6=\mu_5=0.978,\mu_6=0.618\times(1.146-0.875)=1.042\\
        &f(\lambda_6)=f(\mu_5)=0.001\\
        &f(\mu_6)=(1.042)^3-3\times1.042+2=0.005\\
        &f(\lambda_6)<f(\mu_6)\\
        &\text{第7次迭代:}\\
        &a_7=0.875,b_7=\mu_6=1.042,\lambda_7=0.382\times(1.042-0.875)=0.939,\mu_7=\lambda_6=0.978\\
        &f(\lambda_7)=(0.939)^3-3\times0.939+2=0.011\\
        &f(\mu_7)=f(\lambda_6)=0.001\\
        &f(\lambda_7)>f(\mu_7)\\
    \end{flalign*}
    \begin{flalign*}
        &\text{第8次迭代:}\\
        &a_8=\lambda_7=0.939,b_8=1.042,\lambda_8=\mu_7=0.978,\mu_8==0.618\times(1.042-0.939)=1.003\\
        &f(\lambda_8)=f(\mu_7)=0.001\\
        &f(\mu_8)=(1.003)^3-3\times1.003+2=0.000\\
        &f(\lambda_8)>f(\mu_8)\\
        &\text{第9次迭代:}\\
        &a_9=\lambda_8=0.978,b_9=1.042,b_9-a_9=0.064<0.1,\text{结束计算}
    \end{flalign*}
    极小值点为1/2*(0.978+1.042)=1.01\\
    列表结果如下
    \begin{center}
        \begin{tblr}{
                hlines,
                vlines,
                row{1} = {mode = math},
            }
            k  & a_k      & b_k    &\lambda_k&\mu_k&f(\lambda_k)&f(\mu_k)& \text{不等式方向} \\
            1  &  0       &  3       & 1.146    &  1.854 &  0.067     & 2.811    &   <    \\
            2  &  0       &  1.854   & 0.708    &  1.146 &  0.231     & 0.067    &   >     \\
            3  &  0.708   &  1.854   & 1.146    &  1.416 &  0.067     & 0.591    &   <     \\
            4  &  0.708   &  1.416   & 0.978    &  1.146 &  0.001     & 0.067    &   <     \\
            5  &  0.708   &  1.146   & 0.875    &  0.978 &  0.045     & 0.001    &   >     \\
            6  &  0.875   &  1.146   & 0.978    &  1.042 &  0.001     & 0.005    &   <     \\
            7  &  0.875   &  1.042   & 0.939    &  0.978 &  0.011     & 0.001    &   >     \\
            8  &  0.939   &  1.042   & 0.978    &  1.003 &  0.001     & 0.000    &   >     \\
            9  &  0.978   &  1.042   &          &        &            &          &       \\
        \end{tblr}
    \end{center}
\end{solution}

\begin{problem}{1.2$\bigstar$}
    $f(x)=e^2-e^{-x}$,区间为$[0,1]$
\end{problem}
\begin{solution}
    在[0,1]区间内,易见f(x)单调,故f(x)是[0,1]上的单峰函数。
    采用黄金分割法,迭代计算
    \begin{flalign*}
        &\text{第1次迭代:}\\
        &a_1=0,b_1=1,\lambda_1=0.382,\mu_1=0.618\\
        &f(\lambda_1)=e^2-e^{(-0.382)}=6.707\\
        &f(\mu_1)=e^2-e^{(-0.618)}=6.850\\
        &f(\lambda_1)<f(\mu_1)\\
        &\text{第2次迭代:}\\
        &a_2=0,b_2=\mu_1=0.618,\lambda_2=0.382\times0.618=0.236,\mu_2=\lambda_1=0.382\\
        &f(\lambda_2)=e^2-e^{(-0.236)}=6.599\\
        &f(\mu_2)=f(\lambda_1)=6.707\\
        &f(\lambda_2)<f(\mu_2)\\
        &\text{第3次迭代:}\\
        &a_3=0,b_3=\mu_2=0.382,\lambda_3=0.382\times0.382=0.146,\mu_3=\lambda_2=0.236\\
        &f(\lambda_3)=e^2-e^{(-0.146)}=6.525\\
        &f(\mu_3)=f(\lambda_2)=6.599\\
        &f(\lambda_3)<f(\mu_3)\\
        &\text{第4次迭代:}\\
        &a_4=0,b_4=\mu_3=0.236,\lambda_4=0.382\times0.236=0.090,\mu_4=\lambda_3=0.146\\
        &f(\lambda_4)=e^2-e^{(-0.090)}=6.475\\
        &f(\mu_4)=f(\lambda_3)=6.525\\
        &f(\lambda_4)<f(\mu_4)\\
        &\text{第5次迭代:}\\
        &a_5=0,b_5=\mu_4=0.146,\lambda_5=0.382\times0.146=0.056,\mu_5=\lambda_4=0.090\\
        &f(\lambda_5)=e^2-e^{(-0.056)}=6.444\\
        &f(\mu_5)=f(\lambda_4)=6.475\\
        &f(\lambda_5)<f(\mu_5)\\
        &\text{第6次迭代:}\\
        &a_6=0,b_6=\mu_5=0.090,b_6-a_6=0.090<0.1,\text{结束计算}
    \end{flalign*}
    极小值点为1/2*(0+0.090)=0.045\\
    列表结果如下
    \begin{center}
        \begin{tblr}{
                hlines,
                vlines,
                row{1} = {mode = math},
            }
            k  & a_k      & b_k    &\lambda_k&\mu_k&f(\lambda_k)&f(\mu_k)& \text{不等式方向} \\
            1  &  0       &  1       & 0.382    &  0.618 &  6.707     & 6.850    &     <    \\
            2  &  0       &  0.618   & 0.236    &  0.382 &  6.599     & 6.707    &     <    \\
            3  &  0       &  0.382   & 0.146    &  0.236 &  6.525     & 6.599    &     <    \\
            4  &  0       &  0.236   & 0.090    &  0.146 &  6.475     & 6.525    &     <    \\
            5  &  0       &  0.146   & 0.056    &  0.090 &  6.444     & 6.475    &     <    \\
            6  &  0       &  0.090   &          &        &            &          &          \\
        \end{tblr}
    \end{center}
\end{solution}

\begin{problem}{1.3$\bigstar$}
    $f(x)=1-xe^{-x^2}$,区间为$[0,1]$
\end{problem}
\begin{solution}
    $f'(x)=e^{-x^2}(2x^2-1)=0$,解得$x=\frac{\sqrt{2}}{2}$或$-\frac{\sqrt{2}}{2}$,仅前者在[0,1]区间内,故f(x)在[0,1]上仅一个驻点,是[0,1]上的单峰函数。
    采用黄金分割法,迭代计算
    \begin{flalign*}
        &\text{第1次迭代:}\\
        &a_1=0,b_1=1,\lambda_1=0.382,\mu_1=0.618\\
        &f(\lambda_1)=1-0.382e^{-0.382^2}=0.670\\
        &f(\mu_1)=1-0.618e^{-0.618^2}=0.578\\
        &f(\lambda_1)>f(\mu_1)\\
        &\text{第2次迭代:}\\
        &a_2=\lambda_1=0.382,b_2=1,\lambda_2=\mu_1=0.618,\mu_2=0.382+0.618(1-0.382)=0.764\\
        &f(\lambda_2)=f(\mu_1)=0.578\\
        &f(\mu_2)=1-0.764e^{-0.764^2}=0.574\\
        &f(\lambda_2)>f(\mu_2)\\
        &\text{第3次迭代:}\\
        &a_3=\lambda_2=0.618,b_3=1,\lambda_3=\mu_2=0.764,\mu_3=0.618+0.618\times(1-0.618)=0.854\\
        &f(\lambda_3)=f(mu_2)=0.574\\
        &f(\mu_3)=1-0.854e^{-0.854^2}=0.588\\
        &f(\lambda_3)<f(\mu_3)\\
        \end{flalign*}
        \begin{flalign*}
        &\text{第4次迭代:}\\
        &a_4=0.618,b_4=\mu_3=0.854,\lambda_4=0.618+0.382\times(0.854-0.618)=0.708,\mu_4=\lambda_3=0.764\\
        &f(\lambda_4)=1-0.708e^{-0.708^2}=0.571\\
        &f(\mu_4)=f(\lambda_3)=0.574\\
        &f(\lambda_4)<f(\mu_4)\\
        &\text{第5次迭代:}\\
        &a_5=0.618,b_5=\mu_4=0.764,\lambda_5=0.618+0.382\times(0.764-0.618)=0.674,\mu_5=\lambda_4=0.708\\
        &f(\lambda_5)=1-0.674e^{-0.674^2}=0.572\\
        &f(\mu_5)=f(\lambda_4)=0.571\\
        &f(\lambda_5)>f(\mu_5)\\
        &\text{第6次迭代:}\\
        &a_6=\lambda_5=0.674,b_6=0.764,b_6-a_6=0.090<0.1,\text{结束计算}
    \end{flalign*}
    极小值点为1/2*(0.674+0.764)=0.719\\
    列表结果如下
    \begin{center}
        \begin{tblr}{
                hlines,
                vlines,
                row{1} = {mode = math},
            }
            k  & a_k      & b_k    &\lambda_k&\mu_k&f(\lambda_k)&f(\mu_k)& \text{不等式方向} \\
            1  &  0       &  1       & 0.382    &  0.618 &  0.670     & 0.578    &     >    \\
            2  &  0.382   &  1       & 0.618    &  0.764 &  0.578     & 0.574    &     >    \\
            3  &  0.618   &  1       & 0.764    &  0.854 &  0.574     & 0.588    &     <    \\
            4  &  0.618   &  0.854   & 0.708    &  0.764 &  0.571     & 0.574    &     <    \\
            5  &  0.618   &  0.764   & 0.674    &  0.708 &  0.572     & 0.571    &     >    \\
            6  &  0.674   &  0.764   &          &        &            &          &          \\
        \end{tblr}
    \end{center}
\end{solution}

\subsubsection{分数法}

\begin{problem}{2.1$\bigstar$}
    $f(x)=x^3-3x+2$,区间为$[0,3]$
\end{problem}
\begin{solution}
    $f'(x)=3x^2-3=0$,解得x=1或-1,仅1在[0,3]区间内,故f(x)在[0,3]上仅一个驻点,是[0,3]上的单峰函数。
    采用分数法,$\varepsilon=0.1$,则$F_n\geq3/0.1=30$,n=8,迭代计算
    \begin{flalign*}
        &\text{第1次迭代:}\\
        &a_1=0,b_1=3,\lambda_1=0+\frac{13}{34}(3-0)=1.147,\mu_1=0+\frac{21}{34}(3-0)=1.853\\
        &f(\lambda_1)=(1.147)^3-3\times1.147+2=0.068\\
        &f(\mu_1)=(1.853)^3-3\times1.853+2=2.803\\
        &f(\lambda_1)<f(\mu_1)\\
        &\text{第2次迭代:}\\
        &a_2=0,b_2=\mu_1=1.853,\lambda_2=0+\frac{8}{21}(1.853-0)=0.706,\mu_2=\lambda_1=1.147\\
        &f(\lambda_2)=(0.706)^3-3\times0.706+2=0.234\\
        &f(\mu_2)=f(\lambda_1)=0.068\\
        &f(\lambda_2)>f(\mu_2)\\
        &\text{第3次迭代:}\\
        &a_3=\lambda_2=0.706,b_3=\mu_1=1.853,\lambda_3=\mu_2=1.147,\mu_3=0.706+\frac{8}{13}(1.853-0.706)=1.412\\
        &f(\lambda_3)=f(\mu_2)=0.068\\
        &f(\mu_3)=(1.412)^3-3\times1.412+2=0.579\\
        &f(\lambda_3)<f(\mu_3)\\
        &\text{第4次迭代:}\\
        &a_4=0.706,b_4=\mu_3=1.412,\lambda_4=0.706+\frac{3}{8}(1.412-0.706)=0.971,\mu_4=\lambda_3=1.147\\
        &f(\lambda_4)=(0.971)^3-3\times0.971+2=0.002\\
        &f(\mu_4)=f(\lambda_3)=0.068\\
        &f(\lambda_4)<f(\mu_4)\\
        &\text{第5次迭代:}\\
        &a_5=0.706,b_5=\mu_4=1.147,\lambda_5=0.706+\frac{2}{5}(1.147-0.706)=0.882,\mu_5=\lambda_4=0.971\\
        &f(\lambda_5)=(0.882)^3-3\times0.882+2=0.040\\
        &f(\mu_5)=f(\lambda_4)=0.002\\
        &f(\lambda_5)>f(\mu_5)\\
        &\text{第6次迭代:}\\
        &a_6=\lambda_5=0.882,b_6=1.147,\lambda_6=\mu_5=0.971,\mu_6=0.882+\frac{2}{3}(1.147-0.882)=1.059\\
        &f(\lambda_6)=f(\mu_5)=0.002\\
        &f(\mu_6)=(1.059)^3-3\times1.059+2=0.011\\
        &f(\lambda_6)<f(\mu_6)\\
        &\text{第7次迭代:}\\
        &a_7=0.882,b_7=\mu_6=1.059,\lambda_7=0.882+\frac{1}{2}(1.059-0.882)=0.971,\mu_7=\lambda_6+\delta=0.971+0.01=0.981\\
        &f(\lambda_7)=(0.971)^3-3\times0.971+2=0.002\\
        &f(\mu_7)=(0.981)^3-3\times0.981+2=0.001\\
        &f(\lambda_7)>f(\mu_7)\\
    \end{flalign*}
    \begin{flalign*}
        &\text{第8次迭代:}\\
        &a_8=\lambda_7=0.971,b_8=1.059,b_6-a_6=0.088<0.1,\text{结束计算}
    \end{flalign*}
    极小值点为1/2*(0.971+1.059)=1.015\\
    列表结果如下
    \begin{center}
        \begin{tblr}{
                hlines,
                vlines,
                row{1} = {mode = math},
            }
            k  & a_k      & b_k    &\lambda_k&\mu_k&f(\lambda_k)&f(\mu_k)& \text{不等式方向} \\
            1  &  0       &  3       & 1.147    &  1.853 &  0.068     & 2.803    &   <    \\
            2  &  0       &  1.853   & 0.706    &  1.147 &  0.234     & 0.068    &   >     \\
            3  &  0.706   &  1.853   & 1.147    &  1.412 &  0.068     & 0.579    &   <     \\
            4  &  0.706   &  1.412   & 0.971    &  1.147 &  0.002     & 0.068    &   <     \\
            5  &  0.706   &  1.147   & 0.882    &  0.971 &  0.040     & 0.002    &   >     \\
            6  &  0.882   &  1.147   & 0.971    &  1.059 &  0.002     & 0.011    &   <     \\
            7  &  0.882   &  1.059   & 0.971    &  0.981 &  0.002     & 0.001    &   >     \\
            8  &  0.971   &  1.059   &          &        &            &          &       \\
        \end{tblr}
    \end{center}
\end{solution}

\begin{problem}{2.2$\bigstar$}
    $f(x)=e^2-e^{-x}$,区间为$[0,1]$
\end{problem}
\begin{solution}
    在[0,1]区间内,易见f(x)单调,故f(x)是[0,1]上的单峰函数。
    采用分数法,$\varepsilon=0.1$,则$F_n\geq1/0.1=10$,n=6,迭代计算
    \begin{flalign*}
        &\text{第1次迭代:}\\
        &a_1=0,b_1=1,\lambda_1=0+\frac{5}{13}(1-0)=0.385,\mu_1=0+\frac{8}{13}(1-0)=0.615\\
        &f(\lambda_1)=e^2-e^{(-0.385)}=6.709\\
        &f(\mu_1)=e^2-e^{(-0.615)}=6.848\\
        &f(\lambda_1)<f(\mu_1)\\
        &\text{第2次迭代:}\\
        &a_2=0,b_2=\mu_1=0.615,\lambda_2=0+\frac{3}{8}(0.615-0)=0.231,\mu_2=\lambda_1=0.385\\
        &f(\lambda_2)=e^2-e^{(-0.231)}=6.595\\
        &f(\mu_2)=f(\lambda_1)=6.709\\
        &f(\lambda_2)<f(\mu_2)\\
        &\text{第3次迭代:}\\
        &a_3=0,b_3=\mu_2=0.385,\lambda_3=0+\frac{2}{5}(0.385-0)=0.154,\mu_3=\lambda_2=0.231\\
        &f(\lambda_3)=e^2-e^{(-0.154)}=6.532\\
        &f(\mu_3)=f(\lambda_2)=6.595\\
        &f(\lambda_3)<f(\mu_3)\\
        &\text{第4次迭代:}\\
        &a_4=0,b_4=\mu_3=0.231,\lambda_4=0+\frac{1}{3}(0.231-0)=0.077,\mu_4=\lambda_3=0.154\\
        &f(\lambda_4)=e^2-e^{(-0.077)}=6.463\\
        &f(\mu_4)=f(\lambda_3)=6.532\\
        &f(\lambda_4)<f(\mu_4)\\
        &\text{第5次迭代:}\\
        &a_5=0,b_5=\mu_4=0.154,\lambda_5=0+\frac{1}{2}(0.154-0)=0.077,\mu_5=\lambda_4+\delta=0.077+0.01=0.087\\
        &f(\lambda_5)=e^2-e^{(-0.077)}=6.463\\
        &f(\mu_5)=e^2-e^{(-0.087)}=6.472\\
        &f(\lambda_5)<f(\mu_5)\\
        &\text{第6次迭代:}\\
        &a_6=0,b_6=\mu_5=0.077,b_6-a_6=0.077<0.1,\text{结束计算}
    \end{flalign*}
    极小值点为1/2*(0+0.077)=0.0385\\
    列表结果如下
    \begin{center}
        \begin{tblr}{
                hlines,
                vlines,
                row{1} = {mode = math},
            }
            k  & a_k      & b_k    &\lambda_k&\mu_k&f(\lambda_k)&f(\mu_k)& \text{不等式方向} \\
            1  &  0       &  1       & 0.385    &  0.615 &  6.709     & 6.848    &     <    \\
            2  &  0       &  0.615   & 0.231    &  0.385 &  6.595     & 6.709    &     <    \\
            3  &  0       &  0.385   & 0.154    &  0.231 &  6.532     & 6.595    &     <    \\
            4  &  0       &  0.231   & 0.077    &  0.154 &  6.463     & 6.532    &     <    \\
            5  &  0       &  0.154   & 0.077    &  0.087 &  6.463     & 6.472    &     <    \\
            6  &  0       &  0.077   &          &        &            &          &          \\
        \end{tblr}
    \end{center}
\end{solution}

\begin{problem}{2.3$\bigstar$}
    $f(x)=1-xe^{-x^2}$,区间为$[0,1]$
\end{problem}
\begin{solution}
    $f'(x)=e^{-x^2}(2x^2-1)=0$,解得$x=\frac{\sqrt{2}}{2}$或$-\frac{\sqrt{2}}{2}$,仅前者在[0,1]区间内,故f(x)在[0,1]上仅一个驻点,是[0,1]上的单峰函数。
    采用分数法,$\varepsilon=0.1$,则$F_n\geq1/0.1=10$,n=6,迭代计算
    \begin{flalign*}
        &\text{第1次迭代:}\\
        &a_1=0,b_1=1,\lambda_1=0+\frac{5}{13}(1-0)=0.385,\mu_1=0+\frac{8}{13}(1-0)=0.615\\
        &f(\lambda_1)=1-0.385e^{-0.385^2}=0.668\\
        &f(\mu_1)=1-0.615e^{-0.615^2}=0.579\\
        &f(\lambda_1)>f(\mu_1)\\
        &\text{第2次迭代:}\\
        &a_2=\lambda_1=0.385,b_2=1,\lambda_2=\mu_1=0.615,\mu_2=0.385+\frac{5}{8}(1-0.385)=0.769\\
        &f(\lambda_2)=f(\mu_1)=0.579\\
        &f(\mu_2)=1-0.769e^{-0.769^2}=0.574\\
        &f(\lambda_2)>f(\mu_2)\\
        &\text{第3次迭代:}\\
        &a_3=\lambda_2=0.615,b_3=1,\lambda_3=\mu_2=0.769,\mu_3=0.615+\frac{3}{5}(1-0.615)=0.846\\
        &f(\lambda_3)=f(mu_2)=0.574\\
        &f(\mu_3)=1-0.846e^{-0.846^2}=0.586\\
        &f(\lambda_3)<f(\mu_3)\\
    \end{flalign*}
    \begin{flalign*}
        &\text{第4次迭代:}\\
        &a_4=0.615,b_4=\mu_3=0.846,\lambda_4=0.615+\frac{1}{3}(0.846-0.615)=0.692,\mu_4=\lambda_3=0.769\\
        &f(\lambda_4)=1-0.692e^{-0.692^2}=0.571\\
        &f(\mu_4)=1-0.769e^{-0.769^2}=0.574\\
        &f(\lambda_4)<f(\mu_4)\\
        &\text{第5次迭代:}\\
        &a_5=0.615,b_5=\mu_4=0.769,\lambda_5=0.615+\frac{1}{2}(0.769-0.615)=0.692,\mu_5=\lambda_4+\delta=0.692+0.01=0.702\\
        &f(\lambda_5)=1-0.692e^{-0.692^2}=0.5713\\
        &f(\mu_5)=1-0.702e^{-0.702^2}=0.5711\\
        &f(\lambda_5)>f(\mu_5)\\
        &\text{第6次迭代:}\\
        &a_6=\lambda_5=0.692,b_6=0.769,b_6-a_6=0.070<0.1,\text{结束计算}
    \end{flalign*}
    极小值点为1/2*(0.692+0.769)=0.719\\
    列表结果如下
    \begin{center}
        \begin{tblr}{
                hlines,
                vlines,
                row{1} = {mode = math},
            }
            k  & a_k      & b_k    &\lambda_k&\mu_k&f(\lambda_k)&f(\mu_k)& \text{不等式方向} \\
            1  &  0       &  1       & 0.385    &  0.615 &  0.668     & 0.579    &     >    \\
            2  &  0.385   &  1       & 0.615    &  0.769 &  0.579     & 0.574    &     >    \\
            3  &  0.615   &  1       & 0.769    &  0.846 &  0.574     & 0.586    &     <    \\
            4  &  0.615   &  0.846   & 0.692    &  0.769 &  0.571     & 0.574    &     <    \\
            5  &  0.615   &  0.769   & 0.692    &  0.702 &  0.5713    & 0.5711   &     >    \\
            6  &  0.692   &  0.769   &          &        &            &          &          \\
        \end{tblr}
    \end{center}
\end{solution}

    %    \include{Chapter15/chapter15}
    %    \include{Chapter6/chapter6}
    %    \include{Chapter7/chapter7}
    %    \include{Chapter8/chapter8}
    %    \include{Chapter9/chapter9}

\end{document}
