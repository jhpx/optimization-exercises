%!TEX program = xelatex
\documentclass[cn,hazy,black,normal]{elegantnote}
\title{《最优化理论与方法》自拟习题及解答}

\author{姜孟冯}

\date{\zhtoday}

%%%%%%%%%%%%%%%%%%%%%%%%%%%%%%%%%%%%%%%%%%%%%%%%%%%%%%%%%%%%%%%%%%%%%
% PACKAGES                                                          %
%%%%%%%%%%%%%%%%%%%%%%%%%%%%%%%%%%%%%%%%%%%%%%%%%%%%%%%%%%%%%%%%%%%%%
\usepackage{amssymb}
\usepackage{optidef}
\usepackage{pgfplots}
\usepackage{tabularray}

%%%%%%%%%%%%%%%%%%%%%%%%%%%%%%%%%%%%%%%%%%%%%%%%%%%%%%%%%%%%%%%%%%%%%%
% STYLE ENVIRONMENT                                                %
%%%%%%%%%%%%%%%%%%%%%%%%%%%%%%%%%%%%%%%%%%%%%%%%%%%%%%%%%%%%%%%%%%%%%%
\usepgfplotslibrary{fillbetween}
\pgfplotsset{
    my axis style/.style={
        axis lines=middle, % 将坐标轴置于图形中心
        unit vector ratio=1 1 1, % 设置 x 轴和 y 轴的单位长度比例
        %        xmin=0,  % x 轴范围
        %        ymin=0,  % y 轴范围
        axis on top,
        xlabel={$x_1$},
        ylabel={$x_2$},
        legend pos=outer north east,
    }
}

\NewTblrEnviron{simplex}
\SetTblrInner[simplex]{
    cells  = {c, m},
    row{2} = {mode = math},
    column{1,2} = {mode = math},
    hline{1,Z} = {0.15em},
    hline{2,3,Y} = {0.08em},
    vline{4} = {0.08em},
    cell{1}{1} = {c=3,r=1}{c},
    cell{Z}{1} = {c=3,r=1}{c},
}
\UseTblrLibrary{diagbox}
%%%%%%%%%%%%%%%%%%%%%%%%%%%%%%%%%%%%%%%%%%%%%%%%%%%%%%%%%%%%%%%%%%%%%%
% PROBLEM ENVIRONMENT                                                %
%%%%%%%%%%%%%%%%%%%%%%%%%%%%%%%%%%%%%%%%%%%%%%%%%%%%%%%%%%%%%%%%%%%%%%
\usepackage{tcolorbox}
\tcbuselibrary{theorems, breakable, skins}

\newtcbtheorem[number within=subsection]{prob}% environment name
{问题}% Title text
{enhanced, % tcolorbox styles
    attach boxed title to top left={xshift = 4mm, yshift=-2mm},
    colback=blue!5, colframe=black, colbacktitle=blue!3, coltitle=black,
    boxed title style={size=small,colframe=gray},
    fonttitle=\bfseries,
    separator sign none
}%
{}

\newenvironment{problem}[1]{\begin{prob*}{#1}{}}{\end{prob*}}

\newtheorem*{solution}{解答.}

%%%%%%%%%%%%%%%%%%%%%%%%%%%%%%%%%%%%%%%%%%%%%%%%%%%%%%%%%%%%%%%%%%%%%%
% MY COMMANDS                                                        %
%%%%%%%%%%%%%%%%%%%%%%%%%%%%%%%%%%%%%%%%%%%%%%%%%%%%%%%%%%%%%%%%%%%%%%
\newcommand{\R}{\mathbf{R}}
\newcommand{\C}{\mathbf{C}}
\newcommand{\F}{\mathbf{F}}
\newcommand{\U}{\mathit{U}}
\newcommand{\V}{\mathit{V}}
\newcommand{\W}{\mathit{W}}
\newcommand{\poly}{\mathcal{P}}
\newcommand{\espace}{\mathcal{L}}
\newcommand{\expect}{\mathcal{E}}
\newcommand{\mat}{\mathcal{M}}
\newcommand{\mtxA}{\mathcal{A}}
\DeclareMathOperator{\Span}{span}
\DeclareMathOperator{\Real}{Re}
\DeclareMathOperator{\Imag}{Im}
\DeclareMathOperator{\Null}{null}
\DeclareMathOperator{\Range}{range}

\newcommand{\ph}{\phantom{+x_0}}
%\newcommand{\bigO}{\mathcal{O}}
%\newcommand{\mat}{\mathcal{M}}
%\newcommand{\defeq}{\vcentcolon=}
%\newcommand{\restr}[1]{|_{#1}}

%%%%%%%%%%%%%%%%%%%%%%%%%%%%%%%%%%%%%%%%%%%%%%%%%%%%%%%%%%%%%%%%%%%%%%
% SECTION NUMBERING                                                  %
%%%%%%%%%%%%%%%%%%%%%%%%%%%%%%%%%%%%%%%%%%%%%%%%%%%%%%%%%%%%%%%%%%%%%%
\renewcommand{\thesubsection}{\thesection\Alph{subsection}}
\renewcommand{\thesubsubsection}{\arabic{subsubsection}}

%%%%%%%%%%%%%%%%%%%%%%%%%%%%%%%%%%%%%%%%%%%%%%%%%%%%%%%%%%%%%%%%%%%%%%
% DOCUMENT                                                           %
%%%%%%%%%%%%%%%%%%%%%%%%%%%%%%%%%%%%%%%%%%%%%%%%%%%%%%%%%%%%%%%%%%%%%%
\begin{document}

    \maketitle



    \section{绪论}

\begin{example}
    对边长为$a$的正方形铁板,在四个角处剪去相等的正方形以制成方形无盖水槽,问如何剪法使水槽的容积最大?
\end{example}

\begin{example}
    求侧面积为常数$6a^2(a>0)$,体积最大的长方体体积。
\end{example}

\begin{problem}{1}
    一个矩形无盖油箱的外部总面积限定为$S$,怎样设计可使油箱的容积最大?写出该问题的数学模型,并回答该问题是几维最优化问题。
\end{problem}

    \section{线性规划}

\subsection{图解法}

\subsubsection{图解法求解}
\begin{problem}{1.1$\bigstar$}
    \begin{mini*}|s|
        {}
        {8x_1 + 5x_2}
        {}
        {}
        \addConstraint{-x_1 + x_2}{\geq 0}
        \addConstraint{6x_1 + 11x_2}{\geq 66}
        \addConstraint{2x_1 + x_2}{\geq 10}
        \addConstraint{x_1, x_2}{\geq 0}{.}
    \end{mini*}
\end{problem}
\begin{solution}
    \begin{tikzpicture}
        \begin{axis}[my axis style,xmax=11.5,ymax=11.5]
            \addplot[name path=a,color=red,domain=0:10] {x};
            \addlegendentry{$-x_1 + x_2=0$}
            \addplot[name path=b,color=blue,domain=0:11] {(66-6*x)/11};
            \addlegendentry{$6x_1 + 11x_2=66$}
            \addplot[name path=c,color=green,domain=0:5] {10-2*x};
            \addlegendentry{$2x_1 + x_2=10$}
            \addplot[name path=func,color=black,domain=0:6,dashed] {(48-8*x)/5};
            \path [name path=ab,intersection segments={of=a and b,sequence=R1 -- L2}];
            \addplot[geyecolor] fill between [of=ab and c, reverse=false,split,every last segment/.style=gray!50];
            \node [anchor=north west] at (axis cs:0,10) {$A$};
            \node [anchor=north east] at (axis cs:2.8,4.8) {$B$};
            \node [anchor=west] at (axis cs:4,4) {$C$};
        \end{axis}
    \end{tikzpicture}

    由图像知,$6x_1 + 11x_2=66$与$2x_1 + x_2=10$的交点B为最优解,解方程
    $$\left\{
    \begin{aligned}
        6x_1 + 11x_2&=66\\
        2x_1 + x_2&=10
    \end{aligned}\right.$$
    得
    $$\left\{
    \begin{aligned}
        x_1 &=2.75\\
        x_2 &=4.5
    \end{aligned}\right.$$
    故(2.75,4.5)为最优解,最大值为:$f_{max}=8\times2.75+5\times4.5=44.5$.
\end{solution}

\begin{problem}{1.2$\bigstar$}
    \begin{mini*}|s|
        {}
        {13x_1 + 5x_2}
        {}
        {}
        \addConstraint{7x_1 + 3x_2}{\geq 19}
        \addConstraint{10x_1 + 2x_2}{\leq 11}
        \addConstraint{x_1, x_2}{\geq 0}{.}
    \end{mini*}
\end{problem}
\begin{solution}
    \begin{tikzpicture}
        \begin{axis}[my axis style]
            \addplot[name path=a,color=red,domain=0:3] {(19-7*x)/3};
            \addlegendentry{$7x_1 + 3x_2=19$}
            \addplot[name path=b,color=blue,domain=0:3] {(11-10*x)/2};
            \addlegendentry{$10x_1 + 2x_2=11$}
            \addplot[name path=func,color=black,domain=0:3,dashed] {(28-13*x)/5};
            \addplot[geyecolor] fill between [of=a and b, split,every segment no 0/.style=gray!50];
        \end{axis}
    \end{tikzpicture}

    由图像易得,无可行解。
\end{solution}

\begin{problem}{1.3}
    \begin{mini*}|s|
        {}
        {5x_1 - 6x_2}
        {}
        {}
        \addConstraint{x_1 + 2x_2}{\leq 10}
        \addConstraint{2x_1 - x_2}{\leq 5}
        \addConstraint{x_1 - 4x_2}{\leq 4}
        \addConstraint{x_1, x_2}{\geq 0}{.}
    \end{mini*}
\end{problem}
\begin{solution}
    \begin{tikzpicture}
        \begin{axis}[my axis style,ymin=0]
            \addplot[name path=a,color=red,domain=0:10] {(10-x)/2};
            \addlegendentry{$x_1 + 2x_2=10$}
            \addplot[name path=b,color=blue,domain=0:10] {2*x-5};
            \addlegendentry{$2x_1 - x_2=5$}
            \addplot[name path=c,color=green,domain=0:10] {(x-4)/4};
            \addlegendentry{$x_1 - 4x_2=4$}
            %            \path[name path=axis,domain=0:10]{0};
            \addplot[name path=func,color=black,domain=0:10,dashed] {4+(5*x)/6};
            %            \path[name path=bx,intersection segments={of=axis and b,sequence= L1 -- R2}];
            \addplot[geyecolor] fill between [of=b and a, split,every segment no 0/.style=gray!50];
            \node [anchor=west] at (axis cs:-0.3,5.3) {$A$};
            \node [anchor=west] at (axis cs:3.8,3) {$B$};
            \node [anchor=south west] at (axis cs:2.5,-0.3) {$C$};
        \end{axis}
    \end{tikzpicture}

    由图像得,点A(0,5)是最优解,最小值为$f_{min}=5\times0-6\times5=-30$.
\end{solution}


\begin{problem}{1.4}
    \begin{mini*}|s|
        {}
        {-x_1 + x_2}
        {}
        {}
        \addConstraint{3x_1 - 7x_2}{\geq 8}
        \addConstraint{x_1 - x_2}{\leq 5}
        \addConstraint{x_1, x_2}{\geq 0}{.}
    \end{mini*}
\end{problem}
\begin{solution}
    \begin{tikzpicture}
        \begin{axis}[my axis style,xmin=0,ymin=0,ymax=4]
            \addplot[name path=a,color=red,domain=0:8] {(3*x-8)/7};
            \addlegendentry{$3x_1 - 7x_2=8$}
            \addplot[name path=b,color=blue,domain=0:8] {x-5};
            \addlegendentry{$x_1 - x_2=5$}
            \addplot[name path=func,color=black,domain=4:8,dashed] {x-4};
            \addplot[geyecolor] fill between [of=a and b, split,every segment no 0/.style=gray!50];
            \node [anchor=south west] at (axis cs:2.5,0) {$A$};
            \node [anchor=west] at (axis cs:27/4,7/4) {$B$};
            \node [anchor=south west] at (axis cs:5,0) {$C$};
        \end{axis}
    \end{tikzpicture}

    由图像得,点B及点C(5,0)及线段BC上的点均是最优解,最小值为$f_{min}=-5+0=-5$.
\end{solution}

\begin{problem}{1.5}
    \begin{maxi*}|s|
        {}
        {-20x_1 + 10x_2}
        {}
        {}
        \addConstraint{x_1 + x_2}{\geq 10}
        \addConstraint{-10x_1 + x_2}{\leq 10}
        \addConstraint{-5x_1 + 5x_2}{\leq 25}
        \addConstraint{x_1 + 4x_2}{\geq 20}
        \addConstraint{x_1, x_2}{\geq 0}{.}
    \end{maxi*}
\end{problem}
\begin{solution}
    \begin{tikzpicture}
        \begin{axis}[my axis style,xmin=0,ymax=20]
            \addplot[name path=a,color=red,domain=0:8] {10-x};
            \addlegendentry{$x_1 + x_2=10$}
            \addplot[name path=b,color=blue,domain=0:5] {10+10*x};
            \addlegendentry{$-10x_1 + x_2=10$}
            \addplot[name path=c,color=green,domain=0:8] {(25+5*x)/5};
            \addlegendentry{$-5x_1 + 5x_2=25$}
            \addplot[name path=d,color=orange,domain=0:8] {(20-x)/4};
            \addlegendentry{$x_1 + 4x_2=20$}
            \addplot[name path=func,color=black,domain=0:8,dashed] {2*x};
            \path [name path=ac,intersection segments={of=a and c,sequence=R1 -- L2}];
            \addplot[geyecolor] fill between [of=ac and d, split,every segment no 0/.style=gray!50];
            \node [anchor=south west] at (axis cs:-0.2,5) {$A$};
            \node [anchor=south] at (axis cs:2.5,7.5) {$B$};
            \node [anchor=south west] at (axis cs:6,3.5) {$C$};
        \end{axis}
    \end{tikzpicture}

    由图像知,$x_1 + x_2=10$与$-5x_1 + 5x_2=25$的交点B为最优解,解方程
    $$\left\{
    \begin{aligned}
        x_1 + x_2&=10\\
        -5x_1 + 5x_2&=25
    \end{aligned}\right.$$
    得
    $$\left\{
    \begin{aligned}
        x_1 &=2.5\\
        x_2 &=7.5
    \end{aligned}\right.$$
    故(2.5,7.5)为最优解,最大值为$f_{max}=-20\times2.5+10\times7.5=25$.
\end{solution}

\begin{problem}{1.6}
    \begin{mini*}|s|
        {}
        {-3x_1 - 2x_2}
        {}
        {}
        \addConstraint{3x_1 + 2x_2}{\leq 6}
        \addConstraint{x_1 - 2x_2}{\leq 1}
        \addConstraint{x_1 + x_2}{\geq 1}
        \addConstraint{-x_1 + 2x_2}{\leq 1}
        \addConstraint{x_1, x_2}{\geq 0}{.}
    \end{mini*}
\end{problem}
\begin{solution}
    \begin{tikzpicture}
        \begin{axis}[my axis style,xmin=0,ymin=0]
            \addplot[name path=a,color=red,domain=0:3] {(6-3*x)/2};
            \addlegendentry{$3x_1 + 2x_2=6$}
            \addplot[name path=b,color=blue,domain=0:3] {(x-1)/2};
            \addlegendentry{$x_1 - 2x_2=1$}
            \addplot[name path=c,color=green,domain=0:3] {1-x};
            \addlegendentry{$x_1 + x_2=1$}
            \addplot[name path=d,color=orange,domain=0:3] {(1+x)/2};
            \addlegendentry{$-x_1 + 2x_2=1$}
            \addplot[name path=func,color=black,domain=0:3,dashed] {2-3*x/2};
            \path [name path=ab,intersection segments={of=a and b,sequence=L1 -- R1}];
            \path [name path=cd,intersection segments={of=c and d,sequence=R2 -- L2}];
            \addplot[geyecolor] fill between [of=ab and cd, split,every segment no 1/.style=gray!50];
            \node [anchor=south] at (axis cs:0.4,0.7) {$A$};
            \node [anchor=south] at (axis cs:1.3,1.1) {$B$};
            \node [anchor=south] at (axis cs:1.8,0.4) {$C$};
            \node [anchor=south] at (axis cs:1,0) {$D$};
        \end{axis}
    \end{tikzpicture}

    由图像得,点B及点C及线段BC上的点均是最优解,任取BC上一点$(1.5,x_2)$,则有
    $$3\times1.5+2x_2=6$$
    得
    $$x_2 = 0.75.$$

    故最优解之一(1.5,0.75),最小值为$f_{min}=-3\times1.5-2\times0.75=-6$.
\end{solution}

\begin{problem}{1.7}
    \begin{maxi*}|s|
        {}
        {5x_1 + 4x_2}
        {}
        {}
        \addConstraint{-2x_1 + x_2}{\geq -4}
        \addConstraint{x_1 + 2x_2}{\leq 6}
        \addConstraint{5x_1 + 3x_2}{\leq 15}
        \addConstraint{x_1, x_2}{\geq 0}{.}
    \end{maxi*}
\end{problem}
\begin{solution}
    \begin{tikzpicture}
        \begin{axis}[my axis style,ymin=0]
            \addplot[name path=a,color=red,domain=0:6] {2*x-4};
            \addlegendentry{$-2x_1 + x_2=-4$}
            \addplot[name path=b,color=blue,domain=0:6] {(6-x)/2};
            \addlegendentry{$x_1 + 2x_2=6$}
            \addplot[name path=c,color=green,domain=0:6] {(15-5*x)/3};
            \addlegendentry{$5x_1 + 3x_2=15$}
            \addplot[name path=func,color=black,domain=0:6,dashed] {2-5*x/4};
            \path [name path=bc,intersection segments={of=b and c,sequence=L1 -- R2}];
            \addplot[geyecolor] fill between [of=a and bc, split,every segment no 0/.style=gray!50];
            \node [anchor=south west] at (axis cs:-0.1,2.8) {$A$};
            \node [anchor=south] at (axis cs:1.9,2) {$B$};
            \node [anchor=west] at (axis cs:2.3,1) {$C$};
            \node [anchor=south] at (axis cs:2,0) {$D$};
        \end{axis}
    \end{tikzpicture}

    由图像知,$x_1 + 2x_2=6$与$5x_1 + 3x_2=15$的交点B为最优解,解方程
    $$\left\{
    \begin{aligned}
        x_1 + 2x_2=6\\
        5x_1 + 3x_2=15
    \end{aligned}\right.$$
    得
    $$\left\{
    \begin{aligned}
        x_1 &=\frac{12}{7}\\
        x_2 &=\frac{15}{7}
    \end{aligned}\right.$$
    故$(\frac{12}{7},\frac{15}{7})$为最优解,最大值为$f_{max}=5\times\frac{12}{7}+4\times\frac{15}{7}=\frac{120}{7}$.
\end{solution}

\begin{problem}{1.8}
    \begin{maxi*}|s|
        {}
        {3x_1 + x_2}
        {}
        {}
        \addConstraint{x_1 - x_2}{\geq 0}
        \addConstraint{x_1 + x_2}{\leq 5}
        \addConstraint{6x_1 + 2x_2}{\leq 21}
        \addConstraint{x_1, x_2}{\geq 0}{.}
    \end{maxi*}
\end{problem}
\begin{solution}
    \begin{tikzpicture}
        \begin{axis}[my axis style,ymin=0,ymax=3]
            \addplot[name path=a,color=red,domain=0:4] {x};
            \addlegendentry{$x_1 - x_2=0$}
            \addplot[name path=b,color=blue,domain=0:4] {5-x};
            \addlegendentry{$x_1 + x_2=5$}            \addplot[name path=c,color=green,domain=0:4] {(21-6*x)/2};
            \addlegendentry{$6x_1 + 2x_2=21$}
            \addplot[name path=func,color=black,domain=0:4,dashed] {4-3*x};
            \path [name path=ab,intersection segments={of=a and b,sequence=L1 -- R2}];
            \addplot[geyecolor] fill between [of=ab and c, split,reverse=false,every segment no 0/.style=gray!50];
            \node [anchor=south west] at (axis cs:-0.1,0) {$A$};
            \node [anchor=south] at (axis cs:2.4,2.5) {$B$};
            \node [anchor=west] at (axis cs:2.7,2.3) {$C$};
            \node [anchor=south] at (axis cs:3.5,0) {$D$};
        \end{axis}
    \end{tikzpicture}

    由图像知,$x_1 + x_2=5$与$6x_1 + 2x_2=21$的交点C为最优解,解方程
    $$\left\{
    \begin{aligned}
        x_1 + x_2=5\\
        6x_1 + 2x_2=21
    \end{aligned}\right.$$
    得
    $$\left\{
    \begin{aligned}
        x_1 &=\frac{11}{4}\\
        x_2 &=\frac{9}{4}
    \end{aligned}\right.$$
    故$(\frac{11}{4},\frac{9}{4})$为最优解,最大值为$f_{max}=3\times\frac{11}{4}+\frac{9}{4}=\frac{21}{2}$.
\end{solution}

\subsection{求基穷举法}

\subsubsection{试通过求基本可行解来确定各问题的最优解}

\begin{problem}{1.1}
    \begin{maxi*}|s|
        {}
        {2x_1 + 5x_2}
        {}
        {}
        \addConstraint{x_1 + 2x_2 + x_3 \ph }{= 16}
        \addConstraint{2x_1 + x_2 \ph + x_4}{= 12}
        \addConstraint{x_j}{\geq 0}{,\ j=1,\ldots,4.}
    \end{maxi*}
\end{problem}
\begin{problem}{1.2}
    \begin{mini*}|s|
        {}
        {-2x_1 + x_2 + x_3 + 10x_4}
        {}
        {}
        \addConstraint{-x_1 + x_2 + x_3 + x_4}{= 20}
        \addConstraint{2x_1 - x_2 \ph + 2x_4}{= 10}
        \addConstraint{x_j}{\geq 0}{,\ j=1,\ldots,4.}
    \end{mini*}
\end{problem}
\begin{problem}{1.3}
    \begin{mini*}|s|
        {}
        {x_1 - x_2}
        {}
        {}
        \addConstraint{x_1 + x_2 + x_3}{\leq 5}
        \addConstraint{-x_1 + x_2 + 2x_3}{\leq 6}
        \addConstraint{x_1, x_2, x_3}{\geq 0}{.}
    \end{mini*}
\end{problem}



    \section{线性规划}

\subsection{单纯形法}

\subsubsection{单纯形法求解}
\begin{problem}{1.1}
    \begin{mini*}|s|
        {}
        {-9x_1 - 16x_2}
        {}
        {}
        \addConstraint{x_1 + 4x_2 + x_3 \ph}{=80 }
        \addConstraint{2x_1 + 3x_2 \ph +x_4}{=90}
        \addConstraint{x_j}{\geq 0}{,\ j=1,\ldots,4.}
    \end{mini*}
\end{problem}
\begin{solution}
    求${-9x_1 - 16x_2}$的最小解,即求${9x_1 + 16x_2}$的最大解
    \begin{center}
        \begin{simplex}{
                hline{5,6,8,9} = {0.08em},
                cell{5,8}{1} = {c=3,r=1}{c},
            }
            c_j \rightarrow &&& 9   & 16  & 0   & 0   \\
            C_B  & X_B  &b    & x_1 & x_2 & x_3 & x_4 \\
            0    & x_3  &80   & 1   & [4] & 1   & 0   \\
            0    & x_4  &90   & 2   & 3   & 0   & 1   \\
            c_j - z_j       &&& 9   & 16  & 0   & 0   \\
            16   & x_2  &20   & 1/4 & 1   & 1/4 & 0   \\
            0    & x_4  &30   &[5/4]& 0   & -3/4& 1   \\
            c_j - z_j       &&& 5   & 0   & -4  & 0   \\
            16   & x_2  &14   & 0   & 1   & 2/5 & -1/5\\
            9    & x_1  &24   & 1   & 0   &-3/5 & 4/5 \\
            c_j - z_j       &&& 0   & 0   & -1  & -4  \\
        \end{simplex}
    \end{center}
    故最优解(24,14,0,0),最小值为$f_{min}=-9\times24-16\times14=-440$.
\end{solution}
\begin{problem}{1.2}
    \begin{maxi*}|s|
        {}
        {x_1 + 3x_2}
        {}
        {}
        \addConstraint{2x_1 + 3x_2 + x_3 \ph}{=6}
        \addConstraint{-x_1 + x_2 \ph + x_4}{=1}
        \addConstraint{x_j}{\geq 0}{,\ j=1,\ldots,4.}
    \end{maxi*}
\end{problem}
\begin{solution}
    \begin{center}
        \begin{simplex}[t]{
                hline{5,6,8,9} = {0.08em},
                cell{5,8}{1} = {c=3,r=1}{c},
            }
            c_j \rightarrow &&& 1   & 3   & 0   & 0   \\
            C_B  & X_B  &b    & x_1 & x_2 & x_3 & x_4 \\
            0    & x_3  &6    & 2   & 3   & 1   & 0   \\
            0    & x_4  &1    & -1  & [1] & 0   & 1   \\
            c_j - z_j       &&& 1   & 3   & 0   & 0   \\
            0    & x_3  &3    & [5] & 0   & 1   & -3  \\
            3    & x_2  &1    & -1  & 1   & 0   & 1   \\
            c_j - z_j       &&& 4   & 0   & 0   & -3  \\
            1    & x_1  &3/5  & 1   & 0   & 1/5 & -3/5\\
            3    & x_2  &8/5  & 0   & 1   & 1/5 & 2/5 \\
            c_j - z_j       &&& 0   & 0   & -4/5& -3/5\\
        \end{simplex}
    \end{center}
    故最优解$(\frac{3}{5},\frac{8}{5},0,0)$,最大值为$f_{max}=1\times\frac{3}{5}+3\times\frac{8}{5}=\frac{27}{5}$.
\end{solution}
\begin{problem}{1.3}
    \begin{maxi*}|s|
        {}
        {-x_1 + 3x_2 + x_3}
        {}
        {}
        \addConstraint{3x_1 - x_2 + 2x_3}{\leq 7}
        \addConstraint{-2x_1 + 4x_2 \ph}{\leq 12}
        \addConstraint{-4x_1 + 3x_2 + 8x_3}{\leq 10}
        \addConstraint{x_1, x_2, x_3}{\geq 0}{.}
    \end{maxi*}
\end{problem}
\begin{solution}
    引入松弛变量$x_4,x_5,x_6$,化为标准形式:
    \begin{maxi*}|s|
        {}
        {-x_1 + 3x_2 + x_3 + 0x_4 + 0x_5 + 0x_6}
        {}
        {}
        \addConstraint{3x_1 - x_2 + 2x_3 + x_4 \ph \ph}{= 7}
        \addConstraint{-2x_1 + 4x_2 \ph \ph + x_5 \ph}{= 12}
        \addConstraint{-4x_1 + 3x_2 + 8x_3 \ph \ph+ x_6}{= 10}
        \addConstraint{x_j}{\geq 0}{,\ j=1,\ldots,6.}
    \end{maxi*}
    \begin{center}
        \begin{simplex}{
                hline{6,7,10,11,14,15} = {0.08em},
                cell{6,10,14}{1} = {c=3,r=1}{c},
            }
            c_j \rightarrow &&& -1  & 3   & 1   & 0   & 0   & 0   \\
            C_B  & X_B  &b    & x_1 & x_2 & x_3 & x_4 & x_5 & x_6 \\
            0    & x_4  &7    & 3   & -1  & 2   & 1   & 0   & 0   \\
            0    & x_5  &12   & -2  & [4] & 0   & 0   & 1   & 0   \\
            0    & x_6  &10   & -4  & 3   & 8   & 0   & 0   & 1   \\
            c_j - z_j       &&& -1  & 3   & 1   & 0   & 0   & 0   \\
            0    & x_4  &10   & 5/2 & 0   & 2   & 1   & 1/4 & 0   \\
            3    & x_2  &3    & -1/2& 1   & 0   & 0   & 1/4 & 0   \\
            0    & x_6  &1    & -5/2& 0   & [8] & 0   & -3/4& 1   \\
            c_j - z_j       &&& 1/2 & 0   & 1   & 0   & -3/4& 0   \\
            0    & x_4  &39/4 &[25/8]& 0  & 0   & 1   & 7/16& -1/4\\
            3    & x_2  &3    & -1/2& 1   & 0   & 0   & 1/4 & 0   \\
            1    & x_3  &1/8  &-5/16& 0   & 1   & 0   &-3/32& 1/8 \\
            c_j - z_j       &&&13/16& 0   & 0   & 0  &-21/32& -1/8\\
            -1   & x_1  &78/25& 1   & 0   & 0   & 8/25& 7/50& -2/25\\
            3    & x_2 &114/25& 0   & 1   & 0   & 4/25& 8/25& -1/25\\
            1    & x_3  &11/10& 0   & 0   & 1   & 1/10&-1/20& 1/10\\
            c_j - z_j       &&& 0   & 0   & 0   &-13/50&-77/100&-3/50\\
        \end{simplex}
    \end{center}
    故最优解$(\frac{78}{25},\frac{114}{25},\frac{11}{10})$,最大值为
    $$f_{max}=-1\times\frac{78}{25}+3\times\frac{114}{25}+1\times\frac{11}{10}=\dfrac{583}{50}$$
\end{solution}
\begin{problem}{1.4}
    \begin{mini*}|s|
        {}
        {3x_1 - 5x_2 - 2x_3 - x_4}
        {}
        {}
        \addConstraint{x_1 + x_2 + x_3}{\leq 4}
        \addConstraint{4x_1 - x_2 + x_3 + 2x_4}{\leq 6}
        \addConstraint{-x_1 + x_2 + 2x_3 + 3x_4}{\leq 12}
        \addConstraint{x_j}{\geq 0}{,\ j=1,\ldots,4.}
    \end{mini*}
\end{problem}
\begin{solution}
    求${3x_1 - 5x_2 - 2x_3 - x_4}$的最小解,即求${-3x_1 + 5x_2 + 2x_3 + x_4}$的最大解.\\
    引入松弛变量$x_5,x_6,x_7$,化为标准形式:
    \begin{maxi*}|s|
        {}
        {-3x_1 + 5x_2 + 2x_3 + x_4 + 0x_5 + 0x_6 + 0x_7}
        {}
        {}
        \addConstraint{x_1 + x_2 + x_3 \ph + x_5 \ph \ph}{= 4}
        \addConstraint{4x_1 - x_2 + x_3 + 2x_4 \ph + x_6 \ph}{= 6}
        \addConstraint{-x_1 + x_2 + 2x_3 + 3x_4 \ph \ph+ x_7}{= 12}
        \addConstraint{x_j}{\geq 0}{,\ j=1,\ldots,7.}
    \end{maxi*}
    \begin{center}
        \begin{simplex}{
                hline{6,7,10,11,14,15} = {0.08em},
                cell{6,10,14}{1} = {c=3,r=1}{c},
            }
            c_j \rightarrow &&& -3  & 5   & 2   & 1   & 0   & 0   & 0   \\
            C_B  & X_B  &b    & x_1 & x_2 & x_3 & x_4 & x_5 & x_6 & x_7 \\
            0    & x_5  &4    & 1   & [1] & 1   & 0   & 1   & 0   & 0   \\
            0    & x_6  &6    & 4   & -1  & 1   & 2   & 0   & 1   & 0   \\
            0    & x_7  &12   & -1  & 1   & 2   & 3   & 0   & 0   & 1   \\
            c_j - z_j       &&& -3  & 5   & 2   & 1   & 0   & 0   & 0   \\
            5    & x_2  &4    & 1   & 1   & 1   & 0   & 1   & 0   & 0   \\
            0    & x_6  &10   & 5   & 0   & 2   & 2   & 1   & 1   & 0   \\
            0    & x_7  &8    & -2  & 0   & 1   & [3] & -1  & 0   & 1   \\
            c_j - z_j       &&& -8  & 0   & -3  & 1   & -5  & 0   & 0   \\
            5    & x_2  &4    & 1   & 1   & 1   & 0   & 1   & 0   & 0   \\
            0    & x_6  &14/3 & 19/3& 0   & 4/3 & 0   & 5/3 & 1   & -2/3\\
            1    & x_4  &8/3  & -2/3& 0   & 1/3 & 1   &-1/3 & 0   & 1/3 \\
            c_j - z_j       &&&-22/3& 0   &-10/3& 0   &-14/3& 0   & -1/3\\
        \end{simplex}
    \end{center}
    故最优解$(0,4,0,\frac{8}{3},0,\frac{14}{3},0)$,最小值为$f_{min}=-5\times4-1\times\frac{8}{3}=-\frac{68}{3}$.
\end{solution}
\begin{problem}{1.5}
    \begin{mini*}|s|
        {}
        {-3x_1 - x_2}
        {}
        {}
        \addConstraint{3x_1 + 3x_2 + x_3 \ph}{=30}
        \addConstraint{4x_1 - 4x_2 \ph + x_4}{=16}
        \addConstraint{2x_1 - x_2 \ph \ph}{\leq 12}
        \addConstraint{x_j}{\geq 0}{,\ j=1,\ldots,4.}
    \end{mini*}
\end{problem}
\begin{solution}
    求${-3x_1 - x_2}$的最小解,即求${3x_1 + x_2}$的最大解.\\
    引入松弛变量$x_5$,化为标准形式:
    \begin{maxi*}|s|
        {}
        {3x_1 + x_2 + 0x_3 + 0x_4 + 0x_5}
        {}
        {}
        \addConstraint{3x_1 + 3x_2 + x_3 \ph \ph}{=30}
        \addConstraint{4x_1 - 4x_2 \ph + x_4 \ph}{=16}
        \addConstraint{2x_1 - x_2 \ph \ph + x_5}{= 12}
        \addConstraint{x_j}{\geq 0}{,\ j=1,\ldots,5.}
    \end{maxi*}
    \begin{center}
        \begin{simplex}{
                hline{6,7,10,11,14,15} = {0.08em},
                cell{6,10,14}{1} = {c=3,r=1}{c},
            }
            c_j \rightarrow &&& 3   & 1   & 0   & 0   & 0   \\
            C_B  & X_B  &b    & x_1 & x_2 & x_3 & x_4 & x_5 \\
            0    & x_3  &30   & 3   & 3   & 1   & 0   & 0   \\
            0    & x_4  &16   & [4] & -4  & 0   & 1   & 0   \\
            0    & x_5  &12   & 2   & -1  & 0   & 0   & 1   \\
            c_j - z_j       &&& 3   & 1   & 0   & 0   & 0   \\
            0    & x_3  &18   & 0   & [6] & 1   & -3/4& 0   \\
            3    & x_1  &4    & 1   & -1  & 0   & 1/4 & 0   \\
            0    & x_5  &4    & 0   & 1   & 0   & -1/2& 1   \\
            c_j - z_j       &&& 0   & 3   & 0   & -3/4& 0   \\
            1    & x_2  &3    & 0   & 1   & 1/6 & -1/8& 0   \\
            3    & x_1  &7    & 1   & 0   & 1/6 & 1/8 & 0   \\
            0    & x_5  &1    & 0   & 0   & -1/6& 3/8 & 1   \\
            c_j - z_j       &&& 0   & 0   & -2/3& -1/4& 0   \\
        \end{simplex}
    \end{center}
    故最优解(7,3,0,0,1),最小值为$f_{min}=-3\times7-1\times3=-24$.
\end{solution}

\subsubsection{转化为标准形式并列出单纯形表}

\begin{problem}{2.1$\bigstar$}
    \begin{maxi*}|s|
        {}
        {2x_1 - 2x_2 - 3x_3}
        {}
        {}
        \addConstraint{2x_1 - x_2 + 2x_3}{\leq 2}
        \addConstraint{-x_1 + 2x_2 - 3x_3}{\leq -2}
        \addConstraint{x_1, x_2, x_3}{\geq 0}{.}
    \end{maxi*}
\end{problem}
\begin{solution}
    标准形式要求$b\geq0$,故约束条件二两边同乘-1,\\
    引入松弛变量$x_4,x_5$,化为标准形式:
    \begin{maxi*}|s|
        {}
        {2x_1 - 2x_2 - 3x_3 + 0x_4 + 0x_5}
        {}
        {}
        \addConstraint{2x_1 - x_2 + 2x_3 + x_4\ph}{= 2}
        \addConstraint{x_1 - 2x_2 + 3x_3 \ph - x_5}{= 2}
        \addConstraint{x_1, x_2, x_3,x_4,x_5}{\geq 0}{.}
    \end{maxi*}
    \begin{center}
        \begin{simplex}{}
            c_j \rightarrow &&& 2   & -2  & -3  & 0   & 0   \\
            C_B  & X_B  &b    & x_1 & x_2 & x_3 & x_4 & x_5 \\
            0    & x_4  &2    & 2   & -1  & 2   & 1   & 0   \\
            0    & x_5  &2    & 1   & -2  & 3   & 0   & -1  \\
            c_j - z_j       &&& 2   & -2  & -3  & 0   & 0   \\
        \end{simplex}
    \end{center}
\end{solution}

\begin{problem}{2.2$\bigstar$}
    \begin{mini*}|s|
        {}
        {-3x_1 + 4x_2 - 2x_3 + 5x_4}
        {}
        {}
        \addConstraint{4x_1 - x_2 + 2x_3 - x_4}{=-2}
        \addConstraint{x_1 + x_2 - x_3 + 2x_4}{\leq 14}
        \addConstraint{-2x_1 + 3x_2 + x_3 - x_4}{\geq 2}
        \addConstraint{x_1, x_2, x_3}{\geq 0}{,x_4\text{无约束}.}
    \end{mini*}
\end{problem}
\begin{solution}
    求原问题的最小解,即求$3x_1 - 4x_2 + 2x_3 - 5x_4$最大解,\\
    标准形式要求$b\geq0$,故约束条件一两边同乘-1,\\
    用非负变量之差$x_5-x_6$代替$x_4$,引入松弛变量$x_7,x_8$,化为标准形式:
    \begin{maxi*}|s|
        {}
        {3x_1 - 4x_2 + 2x_3 - 5x_5 + 5x_6 + 0x_7 + 0x_8}
        {}
        {}
        \addConstraint{-4x_1 + x_2 - 2x_3 + x_5 - x_6 \ph \ph}{=2}
        \addConstraint{x_1 + x_2 - x_3 + 2x_5 - 2x_6 + x_7 \ph}{= 14}
        \addConstraint{-2x_1 + 3x_2 + x_3 - x_5 + x_6 \ph - x_8}{= 2}
        \addConstraint{x_1, x_2, x_3,x_5,x_6,x_7,x_8}{\geq 0}{.}
    \end{maxi*}
    为了列出单纯形表,需要化出一个基,引入人工变量$x_9$,以大M法求解
    \begin{center}
        \begin{simplex}{
                columns={10mm}
                }
            c_j \rightarrow &&& 3   & -4  & 2   & -5  & 5   & 0   & 0  & -M \\
            C_B  & X_B  &b    & x_1 & x_2 & x_3 & x_5 & x_6 & x_7 & x_8& x_9\\
            -M   & x_9  &2    & -4  & 1   & -2  & 1   & -1  & 0   & 0  & 1  \\
            0    & x_7  &14   & 1   & 1   & -1  & 2   & -2  & 1   & 0  & 0  \\
            0    & x_8  &2    & -2  & 3   & 1   & -1  & 1   & 0   & -1 & 0  \\
            c_j - z_j       &&& 3-4M& M-4 & 2-2M& M-5 & 5-M & 0   & 0  & 0  \\
        \end{simplex}
    \end{center}
\end{solution}

\subsubsection{分别用图解法与单纯形法求解并对比}

\begin{problem}{3.1$\bigstar$}
    \begin{maxi*}|s|
        {}
        {10x_1 + 5x_2}
        {}
        {}
        \addConstraint{3x_1 + 4x_2}{\leq 9}
        \addConstraint{5x_1 + 2x_2}{\leq 8}
        \addConstraint{x_1, x_2}{\geq 0}{.}
    \end{maxi*}
\end{problem}
\begin{solution}
    (1)利用图解法求解:
    \begin{tikzpicture}
        \begin{axis}[my axis style,ymin=0,ymax=4]
            \addplot[name path=a,color=red,domain=0:4] {(9-3*x)/4};
            \addlegendentry{$3x_1 + 4x_2=9$}
            \addplot[name path=b,color=blue,domain=0:4] {(8-5*x)/2};
            \addlegendentry{$5x_1 + 2x_2=8$}
            \addplot[name path=func,color=black,domain=0:4,dashed] {(8-10*x)/5};
            \addplot[geyecolor] fill between [of=a and b, split,reverse=false,every segment no 0/.style=gray!50];
            \node [anchor=south west] at (axis cs:-0.1,2.2) {$A$};
            \node [anchor=south] at (axis cs:1.1,1.5) {$B$};
            \node [anchor=south] at (axis cs:1.7,0) {$C$};
            \node [anchor=south west] at (axis cs:0,0) {$O$};
        \end{axis}
    \end{tikzpicture}

    由图像知,$3x_1 + 4x_2=9$与$5x_1 + 2x_2=8$的交点B为最优解,解方程
    $$\left\{
    \begin{aligned}
        3x_1 + 4x_2=9\\
        5x_1 + 2x_2=8
    \end{aligned}\right.$$
    得
    $$\left\{
    \begin{aligned}
        x_1 &=1\\
        x_2 &=1.5
    \end{aligned}\right.$$
    故点$B(1,1.5)$为最优解,最大值为$f_{max}=10\times1+5\times1.5=17.5$.


    (2)利用单纯形法求解:
    \begin{maxi*}|s|
        {}
        {10x_1 + 5x_2 + 0x_3 + 0x_4}
        {}
        {}
        \addConstraint{3x_1 + 4x_2 + x_3 \ph}{= 9}
        \addConstraint{5x_1 + 2x_2 \ph + x_4}{= 8}
        \addConstraint{x_1, x_2, x_3, x_4}{\geq 0}{.}
    \end{maxi*}
    \begin{center}
        \begin{simplex}{
                hline{5,6,8,9} = {0.08em},
                cell{5,8}{1} = {c=3,r=1}{c},
            }
            c_j \rightarrow &&& 10  & 5   & 0   & 0   \\
            C_B  & X_B  &b    & x_1 & x_2 & x_3 & x_4 \\
            0    & x_3  &9    & 3   & 4   & 1   & 0   \\
            0    & x_4  &8    & [5] & 2   & 0   & 1   \\
            c_j - z_j       &&& 10  & 5   & 0   & 0   \\
            0    & x_3  &21/5 & 0  &[14/5]& 1   & -3/5\\
            10   & x_1  &8/5  & 1   & 2/5 & 0   & 1/5 \\
            c_j - z_j       &&& 0   & 1   & 0   & -2  \\
            5    & x_2  &3/2  & 0   & 1   & 5/14&-3/14\\
            10   & x_1  &1    & 1   & 0   & -1/7& 2/7 \\
            c_j - z_j       &&& 0   & 0   &-5/14&-55/14\\
        \end{simplex}
    \end{center}
    故(1,1.5)为最优解,最大值为$f_{max}=10\times1+5\times1.5=17.5$.
    对照可得,计算过程的顶点变化为$O\rightarrow C\rightarrow B$

\end{solution}
\begin{problem}{3.2$\bigstar$}
    \begin{maxi*}|s|
        {}
        {2x_1 + x_2}
        {}
        {}
        \addConstraint{3x_1 + 5x_2 }{\leq 15}
        \addConstraint{6x_1 + 2x_2 }{\leq 24}
        \addConstraint{x_1, x_2}{\geq 0}{.}
    \end{maxi*}
\end{problem}
\begin{solution}
    (1)利用图解法求解:
    \begin{tikzpicture}
        \begin{axis}[my axis style,ymin=0,ymax=4]
            \addplot[name path=a,color=red,domain=0:4] {(15-3*x)/5};
            \addlegendentry{$3x_1 + 5x_2=15$}
            \addplot[name path=b,color=blue,domain=0:4] {(24-6*x)/2};
            \addlegendentry{$6x_1 + 2x_2=24$}
            \addplot[name path=func,color=black,domain=0:4,dashed] {15-2*x};
            \addplot[geyecolor] fill between [of=a and b, split,reverse=false,every segment no 0/.style=gray!50];
            \node [anchor=south west] at (axis cs:-0.1,3) {$A$};
            \node [anchor=south] at (axis cs:3.8,0.8) {$B$};
            \node [anchor=south] at (axis cs:4,0) {$C$};
            \node [anchor=south west] at (axis cs:0,0) {$O$};
        \end{axis}
    \end{tikzpicture}

    由图像知,$3x_1 + 5x_2=15$与$6x_1 + 2x_2=24$的交点B为最优解,解方程
    $$\left\{
    \begin{aligned}
        3x_1 + 5x_2=15\\
        6x_1 + 2x_2=24
    \end{aligned}\right.$$
    得
    $$\left\{
    \begin{aligned}
        x_1 &=\frac{15}{4}\\
        x_2 &=\frac{3}{4}
    \end{aligned}\right.$$
    故点$B(\frac{15}{4},\frac{3}{4})$为最优解,最大值为$f_{max}=2\times\frac{15}{4}+\frac{3}{4}=\frac{33}{4}$.

    (2)利用单纯形法求解:
    \begin{maxi*}|s|
        {}
        {2x_1 + x_2 + 0x_3 + 0x_4}
        {}
        {}
        \addConstraint{3x_1 + 5x_2 + x_3 \ph}{= 15}
        \addConstraint{6x_1 + 2x_2 \ph + x_4}{= 24}
        \addConstraint{x_1, x_2, x_3, x_4}{\geq 0}{.}
    \end{maxi*}
    \begin{center}
        \begin{simplex}{
                hline{5,6,8,9} = {0.08em},
                cell{5,8}{1} = {c=3,r=1}{c},
            }
            c_j \rightarrow &&& 2   & 1   & 0   & 0   \\
            C_B  & X_B  &b    & x_1 & x_2 & x_3 & x_4 \\
            0    & x_3  &15   & 3   & 5   & 1   & 0   \\
            0    & x_4  &24   & [6] & 2   & 0   & 1   \\
            c_j - z_j       &&& 2   & 1   & 0   & 0   \\
            0    & x_3  &3    & 0   & [4] & 1   & -1/2\\
            2    & x_1  &4    & 1   & 1/3 & 0   & 1/6 \\
            c_j - z_j       &&& 0   & 1/3 & 0   & -1/3\\
            1    & x_2  &3/4  & 0   & 1   & 1/4 & -1/8\\
            2    & x_1  &15/4 & 1   & 0   &-1/12& 5/24\\
            c_j - z_j       &&& 0   & 0   &-1/12&-7/24\\
        \end{simplex}
    \end{center}
    故$(\frac{15}{4},\frac{3}{4})$为最优解,最大值为$f_{max}=2\times\frac{15}{4}+\frac{3}{4}=\frac{33}{4}$.
    对照可得,计算过程的顶点变化为$O\rightarrow C\rightarrow B$
\end{solution}

\subsection{大M法与两阶段法}

\subsubsection{分别用大M法与两阶段法求解并对比}

\begin{problem}{1.1$\bigstar$}
    \begin{maxi*}|s|
        {}
        {2x_1 - x_2 - 2x_3}
        {}
        {}
        \addConstraint{x_1 + x_2 + x_3}{\geq 6}
        \addConstraint{-2x_1 \ph + x_3}{\geq 2}
        \addConstraint{ \ph 2x_2 - x_3}{\geq 0}
        \addConstraint{x_1, x_2, x_3}{\geq 0}{.}
    \end{maxi*}
\end{problem}
\begin{solution}
    引入松弛变量$x_4,x_5,x_6$人工变量$x_7,x_8,x_9$,化为标准形式
    \begin{maxi*}|s|
        {}
        {2x_1 - x_2 - 2x_3 + 0x_4 + 0x_5 + 0x_6 - Mx_7 - Mx_8 - Mx_9}
        {}
        {}
        \addConstraint{x_1 + x_2 + x_3 - x_4\ph\ph + x_7\ph\ph}{= 6}
        \addConstraint{-2x_1 \ph + x_3\ph - x_5\ph\ph + x_8\ph}{= 2}
        \addConstraint{ \ph 2x_2 - x_3\ph\ph - x_6\ph\ph + x_9}{= 0}
        \addConstraint{x_1, x_2, x_3}{\geq 0}{.}
    \end{maxi*}
    (1)利用大M法求解:
    \begin{center}
        \begin{simplex}{
                columns={10mm},
                hline{6,7,10,11,14,15} = {0.08em},
                cell{6,10,14}{1} = {c=3,r=1}{c},
            }
            c_j \rightarrow &&& 2   & -1  & -2  & 0   & 0   & 0   & -M  & -M  & -M  \\
            C_B  & X_B  &b    & x_1 & x_2 & x_3 & x_4 & x_5 & x_6 & x_7 & x_8 & x_9 \\
            -M   & x_7  &6    & 1   & [1] & 1   & -1  & 0   & 0   & 1   & 0   & 0   \\
            -M   & x_8  &2    & -2  & 0   & 1   & 0   & -1  & 0   & 0   & 1   & 0   \\
            -M   & x_9  &0    & 0   & 2   & -1  & 0   & 0   & -1  & 0   & 0   & 1   \\
            c_j - z_j       &&& 2-M & 3M-1& M-2 & -M  & -M  & -M  & 0   & 0   & 0   \\
            -1   & x_2  &6    & 1   & 1   & 1   & -1  & 0   & 0   & 1   & 0   & 0   \\
            -M   & x_8  &2    & -2  & 0   & 1   & 0   & -1  & 0   & 0   & 1   & 0   \\
            -M   & x_9  &-12  & -2  & 0   & -3  & 2   & 0   & -1  & -2  & 0   & 1   \\
            c_j - z_j       &&& 3-4M& 0   &-1-2M& 2M-1& -M  & -M  & 1-3M& 0   & 0   \\
        \end{simplex}
    \end{center}
    无上界解。

    (2)利用两阶段法求解:
    \begin{center}
        \begin{simplex}{
                hline{6,7,10,11,14,15} = {0.08em},
                cell{6,10,14}{1} = {c=3,r=1}{c},
            }
            c_j \rightarrow &&& 0   & 0   & 0   & 0   & 0   & 0   & -1  & -1  & -1  \\
            C_B  & X_B  &b    & x_1 & x_2 & x_3 & x_4 & x_5 & x_6 & x_7 & x_8 & x_9 \\
            -1   & x_7  &6    & 1   & [1] & 1   & -1  & 0   & 0   & 1   & 0   & 0   \\
            -1   & x_8  &2    & -2  & 0   & 1   & 0   & -1  & 0   & 0   & 1   & 0   \\
            -1   & x_9  &0    & 0   & 2   & -1  & 0   & 0   & -1  & 0   & 0   & 1   \\
            c_j - z_j       &&& -1  & 3   & 1   & -1  & -1  & -1  & 0   & 0   & 0   \\
            0    & x_2  &6    & 1   & 1   & 1   & -1  & 0   & 0   & 1   & 0   & 0   \\
            -1   & x_8  &2    & -2  & 0   & 1   & 0   & -1  & 0   & 0   & 1   & 0   \\
            -1   & x_9  &-12  & -2  & 0   & -3  & 2   & 0   & -1  & -2  & 0   & 1   \\
            c_j - z_j       &&& -4  & 0   & -2  & 2   & -1  & -1  & -3  & 0   & 0   \\
        \end{simplex}
    \end{center}
    无上界解。
\end{solution}


\begin{problem}{1.2$\bigstar$}
    \begin{mini*}|s|
        {}
        {4x_1 + x_2}
        {}
        {}
        \addConstraint{3x_1 + x_2}{=3}
        \addConstraint{4x_1 + 3x_2}{\geq 6}
        \addConstraint{x_1 + 2x_2}{\leq 4}
        \addConstraint{x_1, x_2}{\geq 0}{.}
    \end{mini*}
\end{problem}
\begin{solution}
    引入松弛变量$x_3,x_4$,人工变量$x_5,x_6$,化为标准形式
    \begin{maxi*}|s|
        {}
        {-4x_1 - x_2 + 0x_3 + 0x_4 - Mx_5 - Mx_6}
        {}
        {}
        \addConstraint{3x_1 + x_2 \ph\ph + x_5\ph}{= 3}
        \addConstraint{4x_1 + 3x_2 - x_3\ph\ph + x_6}{= 6}
        \addConstraint{x_1 + 2x_2 \ph + x_4\ph\ph }{= 4}
        \addConstraint{x_1, x_2, x_3, x_4, x_5, x_6}{\geq 0}{.}
    \end{maxi*}
    (1) 利用大M法求解:
    \begin{center}
        \begin{simplex}{
                columns={15mm},
                hline{6,7,10,11,14,15} = {0.08em},
                cell{6,10,14}{1} = {c=3,r=1}{c},
            }
            c_j \rightarrow &&& -4  & -1  & 0   & 0   & -M  & -M  \\
            C_B  & X_B  &b    & x_1 & x_2 & x_3 & x_4 & x_5 & x_6 \\
            -M   & x_5  &3    & [3] & 1   & 0   & 0   & 1   & 0   \\
            -M   & x_6  &6    & 4   & 3   & -1  & 0   & 0   & 1   \\
            0    & x_4  &4    & 1   & 2   & 0   & 1   & 0   & 0   \\
            c_j - z_j       &&& 7M-4& 4M-1& -M  & 0   & 0   & 0   \\
            -4   & x_1  &1    & 1   & 1/3 & 0   & 0   & 1/3 & 0   \\
            -M   & x_6  &2    & 0   &[5/3]& -1  & 0   & -4/3& 1   \\
            0    & x_4  &3    & 0   & 5/3 & 0   & 1   & -1/3& 0   \\
            c_j - z_j       &&& 0   &(5M+1)/3&-M& 0 &(4-7M)/3&0   \\
            -4   & x_1  &3/5  & 1   & 0   & 1/5 & 0   & 3/5 & -1/5\\
            -1   & x_2  &6/5  & 0   & 1   & -3/5& 0   & -4/5& 3/5 \\
            0    & x_4  &1    & 0   & 0   & [1] & 1   & 1   & -1  \\
            c_j - z_j       &&& 0   & 0   & 1/5 & 0   &4/5-M&-1/5-M\\
            -4   & x_1  &2/5  & 1   & 0   & 0   & -1/5& 2/5 & 0   \\
            -1   & x_2  &9/5  & 0   & 1   & 0   & 3/5 & -1/5& 0   \\
            0    & x_3  &1    & 0   & 0   & 1   & 1   & 1   & -1  \\
            c_j - z_j       &&& 0   & 0   & 0   & -1/5&7/5-M& 0   \\
        \end{simplex}
    \end{center}
    故$(\frac{2}{5},\frac{9}{5})$为最优解,最大值为$f_{min}=4\times\frac{2}{5}+\frac{9}{5}=\frac{17}{5}$.

    (2) 利用两阶段法求解:
    \begin{center}
        \begin{simplex}{
                hline{6,7,10,11} = {0.08em},
                cell{6,10}{1} = {c=3,r=1}{c},
            }
            c_j \rightarrow &&& 0   & 0   & 0   & 0   & -1  & -1  \\
            C_B  & X_B  &b    & x_1 & x_2 & x_3 & x_4 & x_5 & x_6 \\
            -1   & x_5  &3    & [3] & 1   & 0   & 0   & 1   & 0   \\
            -1   & x_6  &6    & 4   & 3   & -1  & 0   & 0   & 1   \\
            0    & x_4  &4    & 1   & 2   & 0   & 1   & 0   & 0   \\
            c_j - z_j       &&& 7   & 4   & -1  & 0   & 0   & 0   \\
            0    & x_1  &1    & 1   & 1/3 & 0   & 0   & 1/3 & 0   \\
            -1   & x_6  &2    & 0   &[5/3]& -1  & 0   & -4/3& 1   \\
            0    & x_4  &3    & 0   & 5/3 & 0   & 1   & -1/3& 0   \\
            c_j - z_j       &&& 0   & 5/3 & -1  & 0   & -4/3& 0   \\
            0    & x_1  &3/5  & 1   & 0   & 1/5 & 0   & 3/5 & -1/5\\
            0    & x_2  &6/5  & 0   & 1   & -3/5& 0   & -4/5& 3/5 \\
            0    & x_4  &1    & 0   & 0   & 1   & 1   & 1   & -1  \\
            c_j - z_j       &&& 0   & 0   & 0   & 0   & -1  & -1  \\
        \end{simplex}
        \begin{simplex}{
                hline{6,7} = {0.08em},
                cell{6}{1} = {c=3,r=1}{c},
            }
            c_j \rightarrow &&& -4  & -1  & 0   & 0   \\
            C_B  & X_B  &b    & x_1 & x_2 & x_3 & x_4 \\
            -4   & x_1  &3/5  & 1   & 0   & 1/5 & 0   \\
            -1   & x_2  &6/5  & 0   & 1   & -3/5& 0   \\
            0    & x_4  &1    & 0   & 0   & [1] & 1   \\
            c_j - z_j       &&& 0   & 0   & 1/5 & 0   \\
            -4   & x_1  &2/5  & 1   & 0   & 0   & -1/5\\
            -1   & x_2  &9/5  & 0   & 1   & 0   & 3/5 \\
            0    & x_3  &1    & 0   & 0   & 1   & 1   \\
            c_j - z_j       &&& 0   & 0   & 0   & -1/5\\
        \end{simplex}
    \end{center}
    故$(\frac{2}{5},\frac{9}{5})$为最优解,最大值为$f_{min}=4\times\frac{2}{5}+\frac{9}{5}=\frac{17}{5}$.
\end{solution}
\subsubsection{不限方法求解}

\begin{problem}{2.1}
    \begin{mini*}|s|
        {}
        {4x_1 + 6x_2 + 18x_3}
        {}
        {}
        \addConstraint{x_1 \ph + 3x_3}{\geq 3}
        \addConstraint{\ph x_2 + 2x_3}{\geq 5}
        \addConstraint{x_1, x_2, x_3}{\geq 0}{.}
    \end{mini*}
\end{problem}
\begin{solution}
    引入松弛变量$x_4,x_5$,化为标准形式:
    \begin{maxi*}|s|
        {}
        {-4x_1 - 6x_2 - 18x_3 + 0x_4 + 0x_5}
        {}
        {}
        \addConstraint{x_1 \ph + 3x_3 - x_4\ph}{= 3}
        \addConstraint{\ph x_2 + 2x_3\ph - x_5}{= 5}
        \addConstraint{x_1, x_2, x_3, x_4, x_5}{\geq 0}{.}
    \end{maxi*}
    \begin{center}
        \begin{simplex}{
                hline{5,6} = {0.08em},
                cell{5}{1} = {c=3,r=1}{c},
            }
            c_j \rightarrow &&& -4  & -6  & -18 & 0   & 0   \\
            C_B  & X_B  &b    & x_1 & x_2 & x_3 & x_4 & x_5 \\
            -4   & x_1  &3    & 1   & 0   & [3] & -1  & 0   \\
            -6   & x_2  &5    & 0   & 1   & 2   & 0   & -1  \\
            c_j - z_j       &&& 0   & 0   & 6   & -4  & -6  \\
            -18  & x_3  &1    & 1/3 & 0   & 1   & -1/3& 0   \\
            -6   & x_2  &3    & -2/3& 1   & 0   & 2/3 & -1  \\
            c_j - z_j       &&& -2  & 0   & 0   & -2  & -6  \\
        \end{simplex}
    \end{center}
    故$(0,3,1)$为最优解,最小值为$f_{min}=4\times0+6\times3+18\times1=36$.
\end{solution}
\begin{problem}{2.2}
    \begin{maxi*}|s|
        {}
        {2x_1 + x_2}
        {}
        {}
        \addConstraint{x_1 + x_2}{\leq 5}
        \addConstraint{x_1 - x_2}{\geq 0}
        \addConstraint{6x_1 + 2x_2}{\leq 21}
        \addConstraint{x_1, x_2}{\geq 0}{.}
    \end{maxi*}
\end{problem}
\begin{problem}{2.3}
    \begin{maxi*}|s|
        {}
        {3x_1 - 5x_2}
        {}
        {}
        \addConstraint{-x_1 + 2x_2 + 4x_3}{\leq 4}
        \addConstraint{x_1 + x_2 + 2x_3}{\leq 5}
        \addConstraint{-x_1 + 2x_2 + x_3}{\geq 1}
        \addConstraint{x_1, x_2, x_3}{\geq 0}{.}
    \end{maxi*}
\end{problem}
\begin{problem}{2.4}
    \begin{mini*}|s|
        {}
        {x_1 - 3x_2 + x_3}
        {}
        {}
        \addConstraint{2x_1 - x_2 + x_3}{=8}
        \addConstraint{2x_1 + x_2 \ph}{\geq 2}
        \addConstraint{x_1 + 2x_2 \ph}{\leq 10}
        \addConstraint{x_1, x_2, x_3}{\geq 0}{.}
    \end{mini*}
\end{problem}
\begin{problem}{2.5}
    \begin{maxi*}|s|
        {}
        {-3x_1 + x_2 - x_3}
        {}
        {}
        \addConstraint{2x_1 + x_2 - x_3}{\leq 5}
        \addConstraint{4x_1 + 3x_2 + x_3}{\geq 3}
        \addConstraint{-x_1 + x_2 + x_3}{=2}
        \addConstraint{x_1, x_2, x_3}{\geq 0}{.}
    \end{maxi*}
\end{problem}
\begin{problem}{2.6}
    \begin{mini*}|s|
        {}
        {2x_1 - 3x_2 + 4x_3}
        {}
        {}
        \addConstraint{x_1 + x_2 + x_3}{\leq 9}
        \addConstraint{-x_1 + 2x_2 - x_3}{\geq 5}
        \addConstraint{2x_1 - x_2 \ph}{\leq 7}
        \addConstraint{x_1, x_2, x_3}{\geq 0}{.}
    \end{mini*}
\end{problem}
\begin{problem}{2.7}
    \begin{mini*}|s|
        {}
        {3x_1 - 2x_2 + x_3}
        {}
        {}
        \addConstraint{2x_1 - 3x_2 + x_3}{=1}
        \addConstraint{2x_1 + 3x_2 \ph}{\geq 8}
        \addConstraint{x_1, x_2, x_3}{\geq 0}{.}
    \end{mini*}
\end{problem}
\begin{problem}{2.8}
    \begin{mini*}|s|
        {}
        {2x_1 - 3x_2}
        {}
        {}
        \addConstraint{2x_1 - x_2 - x_3}{\geq 3}
        \addConstraint{x_1 - x_2 + x_3}{\geq 2}
        \addConstraint{x_1, x_2, x_3}{\geq 0}{.}
    \end{mini*}
\end{problem}
\begin{problem}{2.9}
    \begin{mini*}|s|
        {}
        {2x_1 + x_2 - x_3 - x_4}
        {}
        {}
        \addConstraint{x_1 - x_2 + 2x_3 - x_4}{=2}
        \addConstraint{2x_1 + x_2 - 3x_3 + x_4}{=6}
        \addConstraint{x_1 + x_2 + x_3 + x_4}{=7}
        \addConstraint{x_j}{\geq 0}{,\ j=1,\ldots,4.}
    \end{mini*}
\end{problem}
\begin{problem}{2.10}
    \begin{maxi*}|s|
        {}
        {3x_1 - x_2 - 3x_3 + x_4}
        {}
        {}
        \addConstraint{x_1 + 2x_2 - x_3 + x_4}{=0}
        \addConstraint{x_1 - x_2 + 2x_3 - x_4}{=6}
        \addConstraint{2x_1 - 2x_2 + 3x_3 + 3x_4}{=9}
        \addConstraint{x_j}{\geq 0}{,\ j=1,\ldots,4.}
    \end{maxi*}
\end{problem}

\subsection{对偶理论}

\subsubsection{对偶理论基础}

\begin{problem}{1.1$\bigstar$}
    已知线性规划问题
    \begin{mini*}|s|
        {}
        {z = 8x_1 + 6x_2 + 3x_3 + 6x_4}
        {}
        {}
        \addConstraint{x_1 + 2x_2 \ph + x_4}{\geq 3}
        \addConstraint{3x_1 + x_2 + x_3 + x_4}{\geq 6}
        \addConstraint{\ph \ph x_3 + x_4}{=2}
        \addConstraint{x_1 \ph + x_3 \ph}{\geq 2}
        \addConstraint{x_1,x_2,x_3,x_4}{\geq 0}{.}
    \end{mini*}
    \begin{enumerate}
        \item[(1)] 写出其对偶问题;
        \item[(2)] 已知原问题最优解为$X^*=(1,1,2,0)$,试根据对偶理论,直接求出对偶问题的最优解。
    \end{enumerate}
\end{problem}
\begin{solution}
    (1) 对偶问题为
    \begin{maxi*}|s|
        {}
        {w = 3y_1 + 6y_2 + 2y_3 + 2y_4}
        {}
        {}
        \addConstraint{y_1 + 3y_2 \ph + y_4}{\leq 8}
        \addConstraint{2y_1 + y_2 \ph\ph}{\leq 6}
        \addConstraint{\ph y_2 + y_3 + y_4}{\leq 3}
        \addConstraint{y_1 + y_2 + y_3 \ph}{\leq 6}
        \addConstraint{y_1,y_2,y_4 \geq 0}{,y_3\text{无约束}.}
    \end{maxi*}
    (2) 把$X^*$代入原问题,第1、2、3个约束条件为严格等式,第4个约束条件为严格不等式,\\
    由互补松弛性$(X_s^T\hat{Y} = 0)$得
    $$y_4^*=0,y_1^*,y_2^*,y_3^*>0$$
    又因$x_1^*,x_2^*,x_3^*>0$,由互补松弛性$(\hat{X^T}Y_s=0)$得对偶问题的1、2、3个约束条件应满足严格等式,有
    $$\left\{
    \begin{aligned}
        y_1 + 3y_2\ph &=8\\
        2y_1 + y_2\ph &=6\\
        \ph y_2 + y_3 &=3
    \end{aligned}\right.$$
    解得
    $$\left\{
    \begin{aligned}
        y_1 &=2\\
        y_2 &=2\\
        y_3 &=1
    \end{aligned}\right.$$
    故对偶问题最优解为$(2,2,1,0)^T$.
\end{solution}
\subsubsection{对偶单纯形法求解}

\begin{problem}{2.1$\bigstar$}
    \begin{mini*}|s|
        {}
        {4x_1 + 12x_2 + 18x_3}
        {}
        {}
        \addConstraint{x_1  \ph + 3x_3}{\geq 3}
        \addConstraint{\ph 2x_2 + 2x_3}{\geq 5}
        \addConstraint{x_1,x_2,x_3}{\geq 0}{.}
    \end{mini*}
\end{problem}
\begin{solution}
    \begin{center}
        \begin{simplex}{
                hline{5,6,8,9} = {0.08em},
                cell{5,8}{1} = {c=3,r=1}{c},
            }
            c_j \rightarrow &&& -4  & -12 & -18 & 0   & 0   \\
            C_B  & X_B  &b    & x_1 & x_2 & x_3 & x_4 & x_5 \\
            0    & x_4  &-3   & -1  & 0   & -3  & 1   & 0   \\
            0    & x_5  &-5   & 0   & [-2]& -2  & 0   & 1   \\
            c_j - z_j       &&& -4  & -12 & -18 & 0   & 0   \\
            0    & x_4  &-3   & -1  & 0   & [-3]& 1   & 0   \\
            -12  & x_2  &2/5  & 0   & 1   & 1   & 0   & -1/2\\
            c_j - z_j       &&& -4  & 0   & -6  & 0   & -6  \\
            -18  & x_3  &1    & 1/3 & 0   & 1   & -1/3& 0   \\
            -12  & x_2  &3/2  & -1/3& 1   & 0   & 1/3 & -1/2\\
            c_j - z_j       &&& -2  & 0   & 0   & -2  & -6  \\
        \end{simplex}
    \end{center}

    故$(0,\frac{3}{2},1)$为最优解,最小值为$f_{min}=4\times0+12\times\frac{3}{2}+18\times1=36$.
\end{solution}

\begin{problem}{2.2$\bigstar$}
    \begin{mini*}|s|
        {}
        {5x_1 + 2x_2 + 4x_3}
        {}
        {}
        \addConstraint{3x_1 + x_2 + 2x_3}{\geq 4}
        \addConstraint{6x_1 + 3x_2 + 5x_3}{\geq 10}
        \addConstraint{x_1,x_2,x_3}{\geq 0}{.}
    \end{mini*}
\end{problem}
\begin{solution}
    \begin{center}
        \begin{simplex}{
                hline{5,6,8,9} = {0.08em},
                cell{5,8}{1} = {c=3,r=1}{c},
            }
            c_j \rightarrow &&& -5  & -2  & -4  & 0   & 0   \\
            C_B  & X_B  &b    & x_1 & x_2 & x_3 & x_4 & x_5 \\
            0    & x_4  &-4   & -3  & -1  & -2  & 1   & 0   \\
            0    & x_5  &-10  & -6  & [-3]& -5  & 0   & 1   \\
            c_j - z_j       &&& -5  & -2  & -4  & 0   & 0   \\
            0    & x_4  &-2/3 & [-1]& 0   & -1/3& 1   & -1/3\\
            -2   & x_2  &10/3 & 2   & 1   & 5/3 & 0   & -1/3\\
            c_j - z_j       &&& -1  & 0   & -2/3& 0   & -2/3\\
            -5   & x_1  &2/3  & 1   & 0   & 1/3 & -1  & 1/3 \\
            -2   & x_2  &2    & 0   & 1   & 1   & 2   & -1  \\
            c_j - z_j       &&& 0   & 0   & -1/3& -1  & -1/3\\
        \end{simplex}
    \end{center}

    故$(\frac{2}{3},2,0,0,0)$为最优解,最小值为$f_{min}=5\times\frac{2}{3}+2\times2+4\times0=\frac{22}{3}$.
\end{solution}
\begin{problem}{2.3}
    \begin{mini*}|s|
        {}
        {4x_1 + 6x_2 + 18x_3}
        {}
        {}
        \addConstraint{x_1  \ph + 3x_3}{\geq 3}
        \addConstraint{\ph x_2 + 2x_3}{\geq 5}
        \addConstraint{x_1,x_2,x_3}{\geq 0}{.}
    \end{mini*}
\end{problem}
\begin{solution}
    \begin{center}
        \begin{simplex}{
                hline{5,6,8,9} = {0.08em},
                cell{5,8}{1} = {c=3,r=1}{c},
            }
            c_j \rightarrow &&& -4  & -6  & -18 & 0   & 0   \\
            C_B  & X_B  &b    & x_1 & x_2 & x_3 & x_4 & x_5 \\
            0    & x_4  &-3   & -1  & 0   & -3  & 1   & 0   \\
            0    & x_5  &-5   & 0   & [-1]& -2  & 0   & 1   \\
            c_j - z_j       &&& -4  & -6  & -18 & 0   & 0   \\
            0    & x_4  &-3   & -1  & 0   & [-3]& 1   & 0   \\
            -6   & x_2  &5    & 0   & 1   & 2   & 0   & -1  \\
            c_j - z_j       &&& -4  & 0   & -6  & 0   & -6  \\
            -18  & x_3  &1    & 1/3 & 0   & 1   & -1/3& 0   \\
            -6   & x_2  &3    & -2/3& 1   & 0   & 2/3 & -1  \\
            c_j - z_j       &&& -2  & 0   & 0   & -2  & -6  \\
        \end{simplex}
    \end{center}

    故$(0,3,1,0,0)$为最优解,最小值为$f_{min}=4\times0+6\times3+18\times1=36$.
\end{solution}
\begin{problem}{2.4}
    \begin{maxi*}|s|
        {}
        {-3x_1 - 2x_2 - 4x_3 - 8x_4}
        {}
        {}
        \addConstraint{-2x_1 + 5x_2 + 3x_3 - 5x_4}{\leq 3}
        \addConstraint{x_1 + 2x_2 + 5x_3 + 6x_4}{\geq 8}
        \addConstraint{x_j}{\geq 0}{,\ j=1,\ldots,4.}
    \end{maxi*}
\end{problem}
\begin{solution}
    \begin{center}
        \begin{simplex}{
                hline{5,6,8,9} = {0.08em},
                cell{5,8}{1} = {c=3,r=1}{c},
            }
            c_j \rightarrow &&& -3  & -2  & -4  & -8  & 0   & 0   \\
            C_B  & X_B  &b    & x_1 & x_2 & x_3 & x_4 & x_5 & x_6 \\
            0    & x_5  &3    & -2  & 5   & 3   & -5  & 1   & 0   \\
            0    & x_6  &-8   & -1  & -2  & [-5]& -6  & 0   & 1   \\
            c_j - z_j       &&& -3  & -2  & -4  & -8  & 0   & 0   \\
            0    & x_5  &-9/5 &-13/5& 19/5& 0 &[-43/5]& 1   & 3/5 \\
            -4   & x_3  &8/5  & 1/5 & 2/5 & 1   & 6/5 & 0   & -1/5\\
            c_j - z_j       &&&-11/5& -2/5& 0   &-16/5& 0   & -4/5\\
            -8   & x_4  &9/43 &13/43&-19/43& 0   & 1   &-5/43&-3/43\\
            -4   & x_3  &58/43&-7/43&40/43& 1   & 0   & 6/43&-5/43\\
            c_j - z_j       &&&-53/43&-78/43&0  & 0  &-16/43&-44/43\\
        \end{simplex}
    \end{center}

    故$(0,0,\frac{58}{43},\frac{9}{43},0,0)$为最优解,最大值为$f_{max}=-3\times0-2\times0-4\times\frac{58}{43}-8\times\frac{9}{43}=-\frac{304}{43}$.
\end{solution}
\begin{problem}{2.5}
    \begin{maxi*}|s|
        {}
        {x_1 + x_2}
        {}
        {}
        \addConstraint{x_1 - x_2 - x_3}{=1}
        \addConstraint{-x_1 + x_2 + 2x_3}{\geq 1}
        \addConstraint{x_1,x_2,x_3}{\geq 0}{.}
    \end{maxi*}
\end{problem}
\begin{solution}
    把第一个约束条件代入第二个约束条件,并加入松弛$x_4$,改造原问题为:
    \begin{maxi*}|s|
        {}
        {x_1 + x_2}
        {}
        {}
        \addConstraint{-x_1 + x_2 + x_3\ph}{=-1}
        \addConstraint{\ph\ph-x_3+x_4}{\leq -2}
        \addConstraint{x_1,x_2,x_3}{\geq 0}{.}
    \end{maxi*}
    \begin{center}
        \begin{simplex}{
                hline{5,6,8,9} = {0.08em},
                cell{5,8}{1} = {c=3,r=1}{c},
            }
            c_j \rightarrow &&& 1   & 1   & 0   & 0   \\
            C_B  & X_B  &b    & x_1 & x_2 & x_3 & x_4 \\
            1    & x_2  &-1   & -1  & 1   & 1   & 0   \\
            0    & x_4  &-2   & 0   & 0   & [-1]& 1   \\
            c_j - z_j       &&& 2   & 0   & -1  & 0   \\
            1    & x_2  &-3   & [-1]& 1   & 0   & 1   \\
            0    & x_3  &2    & 0   & 0   & 1   & -1  \\
            c_j - z_j       &&& 2   & 0   & 0   & -1  \\
            1    & x_1  &3    & 1   & -1  & 0   & -1  \\
            0    & x_3  &2    & 0   & 0   & 1   & -1  \\
            c_j - z_j       &&& 0   & 2   & 0   & 1   \\
        \end{simplex}
    \end{center}
    易见$x_1$从3往正无穷方向可无限增大,原问题无上界解。
\end{solution}
\begin{problem}{2.6}
    \begin{maxi*}|s|
        {}
        {-4x_1 + 3x_2}
        {}
        {}
        \addConstraint{4x_1 + 3x_2 + x_3 - x_4}{=32}
        \addConstraint{2x_1 + x_2 - x_3 - x_4}{=14}
        \addConstraint{x_j}{\geq 0}{,\ j=1,\ldots,4.}
    \end{maxi*}
\end{problem}
\begin{solution}
    利用等式约束条件,构造可行基,改造原问题为:
    \begin{maxi*}|s|
        {}
        {-4x_1 + 3x_2}
        {}
        {}
        \addConstraint{x_1 + x_2 + x_3\ph}{=9}
        \addConstraint{-3x_1 - 2x_2 \ph + x_4}{=-23}
        \addConstraint{x_1,x_2,x_3,x_4}{\geq 0}{.}
    \end{maxi*}

    \begin{center}
        \begin{simplex}{
                hline{5,6,8,9} = {0.08em},
                cell{5,8}{1} = {c=3,r=1}{c},
            }
            c_j \rightarrow &&& -4  & 3   & 0   & 0   \\
            C_B  & X_B  &b    & x_1 & x_2 & x_3 & x_4 \\
            0    & x_4  &9    & 1   & 1   & 1   & 0   \\
            0    & x_5  &-23  & [-3]& -2  & 0   & 1   \\
            c_j - z_j       &&& -4  & 3   & 0   & 0   \\
            0    & x_3  &4/3  & 0   & 1/3 & 1   & 1/3 \\
            -4   & x_1  &23/3 & 1   & 2/3 & 0   & -1/3\\
            c_j - z_j       &&& 0   & 17/3& 0   & -4/3\\
            3    & x_2  &4    & 0   & 1   & 3   & 1   \\
            -4   & x_1  &5    & 1   & 0   & -2  & -1  \\
            c_j - z_j       &&& 0   & 0   & -17 & -7  \\
        \end{simplex}
    \end{center}

    故$(5,4,0,0,0)$为最优解,最大值为$f_{max}=-4\times5+3\times4=-8$.
\end{solution}
\begin{problem}{2.7}
    \begin{mini*}|s|
        {}
        {4x_1 + 3x_2 + 5x_3 + x_4 + 2x_5}
        {}
        {}
        \addConstraint{-x_1 + 2x_2 - 2x_3 + 3x_4 -3x_5 + x_6 \ph + x_8}{=1}
        \addConstraint{x_1 + x_2 - 3x_3 + 2x_4 - 2x_5 \ph \ph + x_8}{=4}
        \addConstraint{\ph \ph -2x_3 + 3x_4 - 3x_5 \ph + x_7 + x_8}{=2}
        \addConstraint{x_j}{\geq 0}{,\ j=1,\ldots,8.}
    \end{mini*}
\end{problem}
\begin{solution}
    利用等式约束条件,构造可行基,改造原问题为:
    \begin{maxi*}|s|
        {}
        {-4x_1 - 3x_2 - 5x_3 - x_4 - 2x_5}
        {}
        {}
        \addConstraint{-2x_1 + x_2 + x_3 + x_4 - x_5 + x_6 \ph \ph}{=-3}
        \addConstraint{x_1 + x_2 - 3x_3 + 2x_4 - 2x_5 \ph \ph + x_8}{=4}
        \addConstraint{-x_1 - x_2 + x_3 + x_4 - x_5 + \ph x_7 \ph}{=2}
        \addConstraint{x_j}{\geq 0}{,\ j=1,\ldots,8.}
    \end{maxi*}
    \begin{center}
        \begin{simplex}{
                hline{6,7,10,11} = {0.08em},
                cell{6,10}{1} = {c=3,r=1}{c},
            }
            c_j \rightarrow &&& -4  & -3  & -5  & -1   & -2   & 0   & 0   & 0   \\
            C_B  & X_B  &b    & x_1 & x_2 & x_3 & x_4 & x_5 & x_6 & x_7 & x_8 \\
            0    & x_6  &-3   & -2  & 1   & 1   & 1   & [-1]& 1   & 0   & 0   \\
            0    & x_8  &4    & 1   & 1   & -3  & 2   & -2  & 0   & 0   & 1   \\
            0    & x_7  &-2   & -1  & -1  & 1   & 1   & -1  & 0   & 1   & 0   \\
            c_j - z_j       &&& -4  & -3  & -5  & -1  & -2  & 0   & 0   & 0   \\
            -2   & x_5  &3    & 2   & -1  & -1  & -1  & 1   & -1  & 0   & 0   \\
            0    & x_8  &10   & 5   & -1  & -5  & 0   & 0   & -2  & 0   & 1   \\
            0    & x_7  &1    & 1   & -2  & 0   & 0   & 0   & -1  & 1   & 0   \\
            c_j - z_j       &&& 0   & -5  & -7  & -3   & 0   & -2  & 0   & 0   \\
        \end{simplex}
    \end{center}

    故$(0,0,0,0,3,0,1,10)$为最优解,最大值为$f_{min}=2\times3=6$.
\end{solution}



    \section{整数规划}


\subsubsection{割平面法求解}

\begin{problem}{1.1$\bigstar$}
    \begin{maxi*}|s|
        {}
        {x_1 + x_2}
        {}
        {}
        \addConstraint{2x_1 + x_2}{\leq 6}
        \addConstraint{4x_1 + 5x_2}{\leq 20}
        \addConstraint{x_1,x_2}{\geq 0}{,\text{且为整数}.}
    \end{maxi*}
\end{problem}
\begin{problem}{1.2$\bigstar$}
    \begin{mini*}|s|
        {}
        {5x_1 + x_2}
        {}
        {}
        \addConstraint{3x_1 + x_2}{\geq 9}
        \addConstraint{x_1 + x_2}{\geq 5}
        \addConstraint{x_1 + 8x_2}{\geq 8}
        \addConstraint{x_1,x_2}{\geq 0}{,\text{且为整数}.}
    \end{mini*}
\end{problem}

\subsubsection{分支定界法求解}

\begin{problem}{2.1$\bigstar$}
    \begin{maxi*}|s|
        {}
        {2x_1 + x_2}
        {}
        {}
        \addConstraint{x_1 + x_2}{\leq 5}
        \addConstraint{-x_1 + x_2}{\leq 0}
        \addConstraint{6x_1 + 2x_2}{\leq 21}
        \addConstraint{x_1,x_2}{\geq 0}{,\text{且为整数}.}
    \end{maxi*}
\end{problem}
\begin{solution}
    第一步,单纯形法求解原问题
    \begin{center}
        \begin{tblr}{
                hline{6,7,10,11} = {0.08em},
                cell{6,10}{1} = {c=3,r=1}{c},
            }
            c_j \rightarrow &&& 2   & 1   & 0   & 0   & 0   \\
            C_B  & X_B  &b    & x_1 & x_2 & x_3 & x_4 & x_5 \\
            0    & x_3  &5    & 1   & 1   & 1   & 0   & 0   \\
            0    & x_4  &0    & -1  & 1   & 0   & 1   & 0   \\
            0    & x_5  &21   & [6] & 2   & 0   & 0   & 1   \\
            c_j - z_j       &&& 2   & 1   & 0   & 0   & 0   \\
            0    & x_3  &3/2  & 0   & 2/3 & 1   & 0   & -1/6\\
            0    & x_4  &7/2  & 0   & 4/3 & 0   & 1   & 1/6 \\
            2    & x_1  &7/2  & 1   & 1/3 & 0   & 0   & 1/6 \\
            c_j - z_j       &&& 0   & -2/3& 0   & 0   & -1/3\\
        \end{tblr}
    \end{center}
    得点$A(\frac{7}{2},0)$,设下界为7,对原问题进行分支
    \begin{equation}
        \begin{maxi*}|s|
            {}
            {2x_1 + x_2}
            {}
            {}
            \addConstraint{x_1 + x_2}{\leq 5}
            \addConstraint{-x_1 + x_2}{\leq 0}
            \addConstraint{6x_1 + 2x_2}{\leq 21}
            \addConstraint{x_1}{\geq 4}
            \addConstraint{x_1,x_2}{\geq 0}{,\text{且为整数}.}
        \end{maxi*}
        \tag{1}
    \end{equation}

    \begin{equation}
        \begin{maxi*}|s|
            {}
            {2x_1 + x_2}
            {}
            {}
            \addConstraint{x_1 + x_2}{\leq 5}
            \addConstraint{-x_1 + x_2}{\leq 0}
            \addConstraint{6x_1 + 2x_2}{\leq 21}
            \addConstraint{x_1}{\leq 3}
            \addConstraint{x_1,x_2}{\geq 0}{,\text{且为整数}.}
        \end{maxi*}
        \tag{2}
    \end{equation}


    第二步,解问题(1),易见$6x_1\geq24$与$\leq21$矛盾,无可行解;解问题(2)
    \begin{center}
        \begin{tblr}{
                hline{7,8,12,13,17,18} = {0.08em},
                cell{7,12,17}{1} = {c=3,r=1}{c},
            }
            c_j \rightarrow &&& 2   & 1   & 0   & 0   & 0   & 0   \\
            C_B  & X_B  &b    & x_1 & x_2 & x_3 & x_4 & x_5 & x_6 \\
            0    & x_3  &5    & 1   & 1   & 1   & 0   & 0   & 0   \\
            0    & x_4  &0    & -1  & 1   & 0   & 1   & 0   & 0   \\
            0    & x_5  &21   & 6   & 2   & 0   & 0   & 1   & 0   \\
            0    & x_6  &3    & [1] & 0   & 0   & 0   & 0   & 1   \\
            c_j - z_j       &&& 2   & 1   & 0   & 0   & 0   & 0   \\
            0    & x_3  &2    & 0   & 1   & 1   & 0   & 0   & -1  \\
            0    & x_4  &3    & 0   & 1   & 0   & 1   & 0   & 1   \\
            0    & x_5  &3    & 0   & [2] & 0   & 0   & 1   & -6  \\
            2    & x_1  &3    & 1   & 0   & 0   & 0   & 0   & 1   \\
            c_j - z_j       &&& 0   & 1   & 0   & 0   & 0   & -2  \\
            0    & x_3  &1/2  & 0   & 0   & 1   & 0   & -1/2& [2] \\
            0    & x_4  &3/2  & 0   & 0   & 0   & 1   & -1/2& 4   \\
            1    & x_2  &3/2  & 0   & 1   & 0   & 0   & 1/2 & -3  \\
            2    & x_1  &3    & 1   & 0   & 0   & 0   & 0   & 1   \\
            c_j - z_j       &&& 0   & 1   & 0   & 0   & -1/2& 1   \\
            0    & x_6  &1/4  & 0   & 0   & 1/2 & 0   & -1/4& 1   \\
            0    & x_4  &1/2  & 0   & 0   & -2  & 1   & 1/2 & 0   \\
            1    & x_2  &9/4  & 0   & 1   & 3/2 & 0   & -1/4& 0  \\
            2    & x_1  &11/4 & 1   & 0   & -1/2& 0   & 1/4 & 0   \\
            c_j - z_j       &&& 0   & 0   & -5/4& 0   & -1/4& 0   \\
        \end{tblr}
    \end{center}
    得点$(\frac{11}{4},\frac{9}{4})$,非整数解,继续分支
    \begin{equation}
    \begin{maxi*}|s|
        {}
        {2x_1 + x_2}
        {}
        {}
        \addConstraint{x_1 + x_2}{\leq 5}
        \addConstraint{-x_1 + x_2}{\leq 0}
        \addConstraint{6x_1 + 2x_2}{\leq 21}
        \addConstraint{x_1}{\leq 3}
        \addConstraint{x_2}{\geq 3}
        \addConstraint{x_1,x_2}{\geq 0}{,\text{且为整数}.}
    \end{maxi*}
    \tag{3}
    \end{equation}
    \begin{equation}
    \begin{maxi*}|s|
        {}
        {2x_1 + x_2}
        {}
        {}
        \addConstraint{x_1 + x_2}{\leq 5}
        \addConstraint{-x_1 + x_2}{\leq 0}
        \addConstraint{6x_1 + 2x_2}{\leq 21}
        \addConstraint{x_1}{\leq 3}
        \addConstraint{x_2}{\leq 2}
        \addConstraint{x_1,x_2}{\geq 0}{,\text{且为整数}.}
    \end{maxi*}
    \tag{4}
    \end{equation}
    第三步,解问题(3),易见$x_2\leq x_1$与$x_1\leq3\leq x_2$交集仅有(3,3),但3+3>5,无可行解;\\
    解问题(4)
        \begin{center}
        \begin{tblr}{
                hline{8,9,14,15,20,21} = {0.08em},
                cell{8,14,20}{1} = {c=3,r=1}{c},
            }
            c_j \rightarrow &&& 2   & 1   & 0   & 0   & 0   & 0   & 0   \\
            C_B  & X_B  &b    & x_1 & x_2 & x_3 & x_4 & x_5 & x_6 & x_7 \\
            0    & x_3  &5    & 1   & 1   & 1   & 0   & 0   & 0   & 0   \\
            0    & x_4  &0    & -1  & 1   & 0   & 1   & 0   & 0   & 0   \\
            0    & x_5  &21   & 6   & 2   & 0   & 0   & 1   & 0   & 0   \\
            0    & x_6  &3    & [1] & 0   & 0   & 0   & 0   & 1   & 0   \\
            0    & x_7  &2    & 0   & 1   & 0   & 0   & 0   & 0   & 1   \\
            c_j - z_j       &&& 2   & 1   & 0   & 0   & 0   & 0   & 0   \\
            0    & x_3  &2    & 0   & 1   & 1   & 0   & 0   & -1  & 0   \\
            0    & x_4  &3    & 0   & 1   & 0   & 1   & 0   & 1   & 0   \\
            0    & x_5  &3    & 0   & [2] & 0   & 0   & 1   & -6  & 0   \\
            2    & x_1  &3    & 1   & 0   & 0   & 0   & 0   & 1   & 0   \\
            0    & x_7  &2    & 0   & 1   & 0   & 0   & 0   & 0   & 1   \\
            c_j - z_j       &&& 0   & 1   & 0   & 0   & 0   & -2  & 0  \\
            0    & x_3  &1/2  & 0   & 0   & 1   & 0   & -1/2& 2   & 0   \\
            0    & x_4  &3/2  & 0   & 0   & 0   & 1   & -1/2& 4   & 0   \\
            1    & x_2  &3/2  & 0   & 1   & 0   & 0   & 1/2 & -3  & 0   \\
            2    & x_1  &3    & 1   & 0   & 0   & 0   & 0   & 1   & 0   \\
            0    & x_7  &1/2  & 1   & 0   & 0   & 0   & -1/2& [3] & 1   \\
            c_j - z_j       &&& 0   & 0   & 0   & 0   & -1/2& 1   & 0   \\
            0    & x_3  &1/6  & 0   & 0   & 1   & 0   & -1/6& 0   & -2/3\\
            0    & x_4  &5/6  & 0   & 0   & 0   & 1   & 1/6 & 0   & -4/3\\
            1    & x_2  &2    & 0   & 1   & 0   & 0   & 0   & 0   & 1   \\
            2    & x_1  &17/6 & 1   & 0   & 0   & 0   & 1/6 & 0   & -1/3\\
            0    & x_6  &1/6  & 0   & 0   & 0   & 0   & -1/6& 1   & 1/3 \\
            c_j - z_j       &&& 0   & 0   & 0   & 0   & -1/3& 0   & -1/3\\
        \end{tblr}
    \end{center}
    得点$(\frac{17}{6},2)$,非整数解,继续分支
    \begin{equation}
        \begin{maxi*}|s|
            {}
            {2x_1 + x_2}
            {}
            {}
            \addConstraint{x_1 + x_2}{\leq 5}
            \addConstraint{-x_1 + x_2}{\leq 0}
            \addConstraint{6x_1 + 2x_2}{\leq 21}
            \addConstraint{x_1}{= 3}
            \addConstraint{x_2}{\leq 2}
            \addConstraint{x_1,x_2}{\geq 0}{,\text{且为整数}.}
        \end{maxi*}
        \tag{5}
    \end{equation}
    \begin{equation}
        \begin{maxi*}|s|
            {}
            {2x_1 + x_2}
            {}
            {}
            \addConstraint{x_1 + x_2}{\leq 5}
            \addConstraint{-x_1 + x_2}{\leq 0}
            \addConstraint{6x_1 + 2x_2}{\leq 21}
            \addConstraint{x_1}{\leq 2}
            \addConstraint{x_2}{\leq 2}
            \addConstraint{x_1,x_2}{\geq 0}{,\text{且为整数}.}
        \end{maxi*}
        \tag{6}
    \end{equation}
    第四步,解问题(5),代入$x_1=3$,则易得最优解(3,1),目标函数值为7;\\
    解问题(6),易见问题(6)上界为6,该分支需舍弃。\\
    综上所述,原问题最优解为(3,1),最大值为7
\end{solution}

\begin{problem}{2.2$\bigstar$}
    \begin{mini*}|s|
        {}
        {5x_1 - x_2}
        {}
        {}
        \addConstraint{3x_1 + 10x_2}{\leq 50}
        \addConstraint{7x_1 - 2x_2}{\leq 28}
        \addConstraint{x_1,x_2}{\geq 0}{,x_2\text{为整数}.}
    \end{mini*}
\end{problem}
\begin{solution}
    第一步,单纯形法求解原问题
    \begin{center}
        \begin{tblr}{
                hline{5,6} = {0.08em},
                cell{5}{1} = {c=3,r=1}{c},
            }
            c_j \rightarrow &&& -5  & 1   & 0   & 0   \\
            C_B  & X_B  &b    & x_1 & x_2 & x_3 & x_4 \\
            0    & x_3  &50   & 3   & [10]& 1   & 0   \\
            0    & x_4  &28   & 7   & -2  & 0   & 1   \\
            c_j - z_j       &&& -5  & 1   & 0   & 0   \\
            1    & x_2  &5    & 3/10& 1   & 1/10& 0   \\
            0    & x_4  &38   & 38/5& 0   & 1/5 & 1   \\
            c_j - z_j       &&&-53/10& 0  &-1/10& 0   \\
        \end{tblr}
    \end{center}
    得点$(0,5)$,整数,原问题得解,目标函数值$f_{min}=5\times0-5=-5$

\end{solution}

\begin{problem}{2.3}
    \begin{mini*}|s|
        {}
        {2x_1 + x_2 - 3x_3}
        {}
        {}
        \addConstraint{x_1 + x_2 + 2x3}{\leq 5}
        \addConstraint{2x_1 + 2x_2 - x_3}{\leq 1}
        \addConstraint{x_1,x_2,x_3}{\geq 0}{,\text{且为整数}.}
    \end{mini*}
\end{problem}
\begin{solution}

\end{solution}

\begin{problem}{2.4}
    \begin{mini*}|s|
        {}
        {4x_1 + 7x_2 + 3x_3}
        {}
        {}
        \addConstraint{x_1 + 3x_2 + x_3}{\geq 5}
        \addConstraint{3x_1 + x_2 + 2x_3}{\geq 8}
        \addConstraint{x_1,x_2,x_3}{\geq 0}{,\text{且为整数}.}
    \end{mini*}
\end{problem}
\begin{solution}

\end{solution}

\subsubsection{最小指派求解}

\begin{problem}{3.1$\bigstar$}
        $$\begin{bmatrix}
        10 & 11 & 4 & 2 & 8\\
        7 & 11 & 10 & 14 & 12\\
        5 & 6 & 9 & 12 & 14\\
        13 & 15 & 11 & 10 & 7
    \end{bmatrix}$$
\end{problem}

\begin{problem}{3.2$\bigstar$}
    $$\begin{bmatrix}
        15 & 18 & 21 & 24\\
        19 & 23 & 22 & 18\\
        26 & 17 & 16 & 19\\
        19 & 21 & 23 & 17
    \end{bmatrix}$$
\end{problem}
%    \section{对偶原理及灵敏度分析}


    %    \include{Chapter15/chapter15}
    %    \include{Chapter6/chapter6}
    %    \include{Chapter7/chapter7}
    %    \include{Chapter8/chapter8}
    %    \include{Chapter9/chapter9}

\end{document}
