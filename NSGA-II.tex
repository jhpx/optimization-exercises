%!TEX program = xelatex
\documentclass[cn,hazy,black,normal]{elegantnote}
\title{NSGA-II算法}

\author{姜孟冯}

\date{\zhtoday}

%%%%%%%%%%%%%%%%%%%%%%%%%%%%%%%%%%%%%%%%%%%%%%%%%%%%%%%%%%%%%%%%%%%%%
% PACKAGES                                                          %
%%%%%%%%%%%%%%%%%%%%%%%%%%%%%%%%%%%%%%%%%%%%%%%%%%%%%%%%%%%%%%%%%%%%%
\usepackage{amssymb}
\usepackage{optidef}
\usepackage{pgfplots}
\usepackage{tabularray}

%%%%%%%%%%%%%%%%%%%%%%%%%%%%%%%%%%%%%%%%%%%%%%%%%%%%%%%%%%%%%%%%%%%%%%
% STYLE ENVIRONMENT                                                %
%%%%%%%%%%%%%%%%%%%%%%%%%%%%%%%%%%%%%%%%%%%%%%%%%%%%%%%%%%%%%%%%%%%%%%
\usepgfplotslibrary{fillbetween}
\pgfplotsset{
    my axis style/.style={
        axis lines=middle, % 将坐标轴置于图形中心
        unit vector ratio=1 1 1, % 设置 x 轴和 y 轴的单位长度比例
        %        xmin=0,  % x 轴范围
        %        ymin=0,  % y 轴范围
        axis on top,
        xlabel={$x_1$},
        ylabel={$x_2$},
        legend pos=outer north east,
    }
}

\NewTblrEnviron{simplex}
\SetTblrInner[simplex]{
    cells  = {c, m},
    row{2} = {mode = math},
    column{1,2} = {mode = math},
    hline{1,Z} = {0.15em},
    hline{2,3,Y} = {0.08em},
    vline{4} = {0.08em},
    cell{1}{1} = {c=3,r=1}{c},
    cell{Z}{1} = {c=3,r=1}{c},
}
\UseTblrLibrary{diagbox}
%%%%%%%%%%%%%%%%%%%%%%%%%%%%%%%%%%%%%%%%%%%%%%%%%%%%%%%%%%%%%%%%%%%%%%
% PROBLEM ENVIRONMENT                                                %
%%%%%%%%%%%%%%%%%%%%%%%%%%%%%%%%%%%%%%%%%%%%%%%%%%%%%%%%%%%%%%%%%%%%%%
\usepackage{tcolorbox}
\tcbuselibrary{theorems, breakable, skins}

\newtcbtheorem[number within=subsection]{prob}% environment name
{问题}% Title text
{enhanced, % tcolorbox styles
    attach boxed title to top left={xshift = 4mm, yshift=-2mm},
    colback=blue!5, colframe=black, colbacktitle=blue!3, coltitle=black,
    boxed title style={size=small,colframe=gray},
    fonttitle=\bfseries,
    separator sign none
}%
{}

\newenvironment{problem}[1]{\begin{prob*}{#1}{}}{\end{prob*}}

\newtheorem*{solution}{解答.}

%%%%%%%%%%%%%%%%%%%%%%%%%%%%%%%%%%%%%%%%%%%%%%%%%%%%%%%%%%%%%%%%%%%%%%
% MY COMMANDS                                                        %
%%%%%%%%%%%%%%%%%%%%%%%%%%%%%%%%%%%%%%%%%%%%%%%%%%%%%%%%%%%%%%%%%%%%%%
\newcommand{\R}{\mathbf{R}}
\newcommand{\C}{\mathbf{C}}
\newcommand{\F}{\mathbf{F}}
\newcommand{\U}{\mathit{U}}
\newcommand{\V}{\mathit{V}}
\newcommand{\W}{\mathit{W}}
\newcommand{\poly}{\mathcal{P}}
\newcommand{\espace}{\mathcal{L}}
\newcommand{\expect}{\mathcal{E}}
\newcommand{\mat}{\mathcal{M}}
\newcommand{\mtxA}{\mathcal{A}}
\DeclareMathOperator{\Span}{span}
\DeclareMathOperator{\Real}{Re}
\DeclareMathOperator{\Imag}{Im}
\DeclareMathOperator{\Null}{null}
\DeclareMathOperator{\Range}{range}

\newcommand{\ph}{\phantom{+x_0}}
%\newcommand{\bigO}{\mathcal{O}}
%\newcommand{\mat}{\mathcal{M}}
%\newcommand{\defeq}{\vcentcolon=}
%\newcommand{\restr}[1]{|_{#1}}

%%%%%%%%%%%%%%%%%%%%%%%%%%%%%%%%%%%%%%%%%%%%%%%%%%%%%%%%%%%%%%%%%%%%%%
% SECTION NUMBERING                                                  %
%%%%%%%%%%%%%%%%%%%%%%%%%%%%%%%%%%%%%%%%%%%%%%%%%%%%%%%%%%%%%%%%%%%%%%
%\renewcommand{\thesubsection}{\thesection\Alph{subsection}}
%\renewcommand{\thesubsubsection}{\arabic{subsubsection}}

%%%%%%%%%%%%%%%%%%%%%%%%%%%%%%%%%%%%%%%%%%%%%%%%%%%%%%%%%%%%%%%%%%%%%%
% DOCUMENT                                                           %
%%%%%%%%%%%%%%%%%%%%%%%%%%%%%%%%%%%%%%%%%%%%%%%%%%%%%%%%%%%%%%%%%%%%%%
\begin{document}

    \maketitle

\section{算法背景}
\subsection{多目标优化问题}
\textbf{多目标优化问题(MOP)}是涉及多个目标函数同时优化的数学问题。
需要在两个或多个相互冲突的目标之间进行权衡的情况下作出\textbf{最优决策}。
一组各个目标值所折衷的解集,称之为Pareto最优集。

多目标优化问题的一般形式表示为:

\begin{Definition}
多目标优化问题(multi-objective optimization problem(MOP))
F ( x ) = ( f 1 ( x ) , . . . , f m ( x ) ) s . t .    x ∈ Ω F(x)=(f_1(x),...,f_m(x))\\ s.t. \; x\in \OmegaF(x)=(f
1
​
(x),...,f
m
​
(x))
s.t.x∈Ω
\end{Definition}

\begin{Definition}
Pareto支配(Pareto Dominance)
x支配y,记为 x ≺ \prec≺ y ,当且仅当∀ i ∈ { 1 , 2 , . . . , m } \forall {i} \in \{1,2,...,m\}∀i∈{1,2,...,m},f i ( x ) ≤ f i f_i(x) \leq f_if
i
​
(x)≤f
i
​
(y), 且∃ j ∈ { 1 , 2 , . . . , m } \exists {j} \in \{1,2,...,m\}∃j∈{1,2,...,m}, s.t.    f j ( x ) < f j ( y ) \; f_j(x)<f_j(y)f
j
​
(x)<f
j
​
(y) 。
\end{Definition}


\begin{Definition}
Definition 3: Pareto最优解(Pareto Optimal Solution)
如果一个解x ∗ x^*x
∗
被称之为Pareto optimal solution, 当且仅当 x ∗ x^*x
∗
不被其他的解支配。
\end{Definition}

\begin{Definition}
Definition 4: Pareto 集(Pareto Set)
一个MOP,对于一组给定的最优解集,如果这个集合中的解是相互非支配的,也即两两不是支配关系,那么则称这个解集为Pareto Set 。
\end{Definition}

\begin{Definition}
Definition 5: Pareto 前沿(Pareto Front)
Pareto Set 中每个解对应的目标值向量组成的集合称之为Pareto Front, 简称为PF
\end{Definition}

上式中,
表示目标函数的数量,
表示不等式约束的数量,
表示等式约束的数量。变量
表示决策矢量(通常也被称为设计矢量/优化矢量),其中
表示独立变量
的个数。
表示目标函数
的矢量。
也被称为目标、标准、效用函数或损失函数等。约束中的不等式符号和等式符号将作用在对应的向量分量上。
\subsection{多目标算法分类}

\begin{enumerate}
    \item 基于Pareto支配关系的方法:NSGA-II
    \item 基于分解的方法:MOEA/D
    \item 基于indicator的方法:IBEA、HypE
\end{enumerate}}


\section{算法原理及框架}
\subsection{算法框架}
NSGA-II算法是一种多目标进化算法,其基本框架如下:

Syntax error in graph
mermaid version 8.14.0

\subsection{适应度分配}
NSGA-II算法使用非支配排序和拥挤度计算来分配适应度。非支配排序将个体分为不同的等级,等级越低,个体的适应度越高。拥挤度计算则衡量个体在目标空间中的拥挤程度,拥挤度越低,个体的适应度越高。

\subsection{拥挤度计算}
拥挤度计算使用以下公式:

crowding_distance(i) = sum(d_i^j / (f_j^{max} - f_j^{min}))
其中:

i 是个体的索引
d_i^j 是个体i在目标j上的拥挤度
f_j^{max} 和 f_j^{min} 是目标j的最大值和最小值

\subsection{选择机制}
NSGA-II算法使用二进制锦标赛选择机制。在每次选择中,算法从种群中随机选择两个个体,并根据它们的非支配等级和拥挤度进行比较。适应度较高的个体将被选中。

\section{算法步骤及实现}
\subsection{参数设置}
NSGA-II算法的参数设置对算法的性能有重要影响。主要参数包括:

**种群规模(N):**控制种群中个体的数量。较大的种群规模可以提供更广泛的搜索空间,但计算成本更高。
**最大进化代数(MaxGen):**控制算法运行的代数。较大的最大进化代数可以提高算法的收敛性,但计算成本也更高。
**交叉概率(Pc):**控制交叉算子应用的概率。较高的交叉概率可以促进种群多样性,但可能导致过早收敛。
**变异概率(Pm):**控制变异算子应用的概率。较高的变异概率可以引入新的个体,但可能破坏种群的收敛性。

\subsection{基本步骤}

NSGA-II算法的实现步骤如下:

**初始化种群:**随机生成一个N个个体的初始种群。
**评估种群:**计算每个个体的目标函数值和约束条件值。
**非支配排序:**将种群中的个体根据非支配关系进行排序。
**拥挤度计算:**计算每个个体的拥挤度,衡量个体在目标空间中的密度。
**选择:**根据非支配等级和拥挤度,选择下一代的个体。
**交叉:**对选定的个体进行交叉操作,产生新的个体。
**变异:**对新的个体进行变异操作,引入新的基因。
**合并:**将新的个体与当前种群合并,形成新的种群。
**重复步骤2-8:**重复上述步骤,直到达到最大进化代数或满足终止条件。

\section{算法应用}
NSGA-II算法应用非常广泛,

\subsection{工程设计优化}

工程设计优化是NSGA-II算法最常见的应用领域之一。在工程设计中,往往需要同时考虑多个目标,如成本、性能、可靠性等。NSGA-II算法可以有效地处理多目标优化问题,找到满足所有目标约束的最佳解决方案。

案例:飞机机翼设计优化

飞机机翼设计是一个典型的多目标优化问题,需要同时考虑机翼的升力、阻力、重量等多个目标。使用NSGA-II算法对飞机机翼进行优化,可以找到满足升力、阻力、重量约束的最佳机翼形状。
逻辑分析:

定义目标函数:计算飞机机翼的升力和阻力。
定义NSGA-II算法:包含初始化种群、迭代优化、选择操作、交叉和变异操作等步骤。
拥挤度计算:计算种群中个体的拥挤度,用于选择操作。
交叉操作:对两个父个体进行交叉,生成两个子个体。
变异操作:对个体进行变异,引入多样性。
运行NSGA-II算法:迭代优化,找到最优解。
输出最优解:打印飞机机翼的最佳形状。

\subsection{资源分配优化}

资源分配优化是NSGA-II算法的另一个重要应用领域。在资源分配问题中,需要在有限的资源约束下,将资源分配给多个任务或项目,以最大化整体收益或最小化成本。NSGA-II算法可以有效地解决此类问题,找到满足资源约束的最佳分配方案。

案例:项目组合优化

项目组合优化是一个典型的资源分配优化问题,需要在有限的预算约束下,选择一组项目进行投资,以最大化投资回报率。使用NSGA-II算法对项目组合进行优化,可以找到满足预算约束的最佳项目组合。

NSGA-II算法在风力发电机组优化中得到了广泛的应用。风力发电机组优化涉及到多目标,如发电量最大化、成本最小化和环境影响最小化。

应用步骤:

定义目标函数:

发电量:P = 0.5 * ρ * A * V^3 * C_p
成本:C = C_i + C_o * P
环境影响:E = N_b * P
其中,ρ为空气密度,A为叶片面积,V为风速,C_p为功率系数,C_i为初始成本,C_o为运行成本,N_b为噪声水平。

\subsection{数据挖掘优化}
数据挖掘优化是NSGA-II算法的又一应用领域。在数据挖掘中,需要从大量数据中提取有价值的信息,如模式、规则或分类器。NSGA-II算法可以优化数据挖掘算法的参数,以提高挖掘效率和准确性。

案例:特征选择优化

特征选择是数据挖掘中的一个重要步骤,需要从原始数据集中选择最具区分性的特征。使用NSGA-II算法对特征选择进行优化,可以找到满足分类或回归任务要求的最佳特征子集。
\end{document}
