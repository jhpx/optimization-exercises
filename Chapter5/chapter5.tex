\section{无约束最优化方法}

\subsection{最速下降法}

\subsubsection{最速下降法求解}

\begin{problem}{1.1$\bigstar$}
    $\min f(x)=x_1^2+2x_2^2-2x_1x_2-4x_1$\\
    初始迭代点${x^{(0)}}=[1,1]^T$
\end{problem}
\begin{solution}
    $\nabla f(x)=\begin{bmatrix}
        2x_1-2x_2-4  \\
        4x_2-2x_1  \\
    \end{bmatrix}, H=\nabla^2 f(x)=\begin{bmatrix}
        2   & -2  \\
        -2  & 4  \\
    \end{bmatrix}$,
    设$\varepsilon$为0.1\\
    第1次迭代:\\
    $p^{(0)}=-\nabla f(x^{(0)})=\begin{bmatrix} 4\\-2\\\end{bmatrix},||p^{(0)}||=2\sqrt{5}>\varepsilon$\\
    $\lambda_0=\dfrac{\nabla f(x^{(0)})^T\nabla f(x^{(0)})}{\nabla f(x^{(0)})^TH\nabla f(x^{(0)})}=\frac{1}{4},x^{(1)}=x^{(0)}+\lambda_0p^{(0)}=\begin{bmatrix} 2\\\frac{1}{2}\\\end{bmatrix}$\\
    第2次迭代:\\
    $p^{(1)}=-\nabla f(x^{(1)})=\begin{bmatrix} 1\\2\\\end{bmatrix},||p^{(1)}||=\sqrt{5}>\varepsilon$\\
    $\lambda_1=\dfrac{\nabla f(x^{(1)})^T\nabla f(x^{(1)})}{\nabla f(x^{(1)})^TH\nabla f(x^{(1)})}=\frac{1}{2},x^{(2)}=x^{(1)}+\lambda_1p^{(1)}=\begin{bmatrix} \frac{5}{2}\\\frac{3}{2}\\\end{bmatrix}$\\
    第3次迭代:\\
    $p^{(2)}=-\nabla f(x^{(2)})=\begin{bmatrix} 2\\-1\\\end{bmatrix},||p^{(2)}||=\sqrt{5}>\varepsilon$\\
    $\lambda_2=\dfrac{\nabla f(x^{(2)})^T\nabla f(x^{(2)})}{\nabla f(x^{(2)})^TH\nabla f(x^{(2)})}=\frac{1}{4},x^{(3)}=x^{(2)}+\lambda_2p^{(2)}=\begin{bmatrix} 3\\\frac{5}{4}\\\end{bmatrix}$\\
    第4次迭代:\\
    $p^{(3)}=-\nabla f(x^{(3)})=\begin{bmatrix} \frac{1}{2}\\1\\\end{bmatrix},||p^{(3)}||=\frac{\sqrt{5}}{2}>\varepsilon$\\
    $\lambda_3=\dfrac{\nabla f(x^{(3)})^T\nabla f(x^{(3)})}{\nabla f(x^{(3)})^TH\nabla f(x^{(3)})}=\frac{1}{2},x^{(4)}=x^{(3)}+\lambda_3p^{(3)}=\begin{bmatrix} \frac{13}{4}\\\frac{7}{4}\\\end{bmatrix}$\\
    第5次迭代:\\
    $p^{(4)}=-\nabla f(x^{(4)})=\begin{bmatrix} 1\\-\frac{1}{2}\\\end{bmatrix},||p^{(4)}||=\frac{\sqrt{5}}{2}>\varepsilon$\\
    $\lambda_4=\dfrac{\nabla f(x^{(4)})^T\nabla f(x^{(4)})}{\nabla f(x^{(4)})^TH\nabla f(x^{(4)})}=\frac{1}{4},x^{(5)}=x^{(4)}+\lambda_4p^{(4)}=\begin{bmatrix} \frac{7}{2}\\\frac{13}{8}\\\end{bmatrix}$\\
    第6次迭代:\\
    $p^{(5)}=-\nabla f(x^{(5)})=\begin{bmatrix} \frac{1}{4}\\\frac{1}{2}\\\end{bmatrix},||p^{(5)}||=\frac{\sqrt{5}}{4}>\varepsilon$\\
    $\lambda_5=\dfrac{\nabla f(x^{(5)})^T\nabla f(x^{(5)})}{\nabla f(x^{(5)})^TH\nabla f(x^{(5)})}=\frac{1}{2},x^{(6)}=x^{(5)}+\lambda_5p^{(5)}=\begin{bmatrix} \frac{29}{8}\\\frac{15}{8}\\\end{bmatrix}$\\
    第7次迭代:\\
    $p^{(6)}=-\nabla f(x^{(6)})=\begin{bmatrix} \frac{1}{2}\\-\frac{1}{4}\\\end{bmatrix},||p^{(6)}||=\frac{\sqrt{5}}{4}>\varepsilon$\\
    $\lambda_6=\dfrac{\nabla f(x^{(6)})^T\nabla f(x^{(6)})}{\nabla f(x^{(6)})^TH\nabla f(x^{(6)})}=\frac{1}{4},x^{(7)}=x^{(6)}+\lambda_6p^{(6)}=\begin{bmatrix} \frac{15}{4}\\\frac{29}{16}\\\end{bmatrix}$\\
    第8次迭代:\\
    $p^{(7)}=-\nabla f(x^{(7)})=\begin{bmatrix} \frac{1}{8}\\\frac{1}{4}\\\end{bmatrix},||p^{(7)}||=\frac{\sqrt{5}}{8}>\varepsilon$\\
    $\lambda_7=\dfrac{\nabla f(x^{(7)})^T\nabla f(x^{(7)})}{\nabla f(x^{(7)})^TH\nabla f(x^{(7)})}=\frac{1}{2},x^{(8)}=x^{(7)}+\lambda_7p^{(7)}=\begin{bmatrix} \frac{61}{16}\\\frac{31}{16}\\\end{bmatrix}$\\
    第9次迭代:\\
    $p^{(8)}=-\nabla f(x^{(8)})=\begin{bmatrix} \frac{1}{4}\\-\frac{1}{8}\\\end{bmatrix},||p^{(8)}||=\frac{\sqrt{5}}{8}>\varepsilon$\\
    $\lambda_8=\dfrac{\nabla f(x^{(8)})^T\nabla f(x^{(8)})}{\nabla f(x^{(8)})^TH\nabla f(x^{(8)})}=\frac{1}{4},x^{(9)}=x^{(8)}+\lambda_8p^{(8)}=\begin{bmatrix} \frac{31}{8}\\\frac{61}{32}\\\end{bmatrix}$\\
    第10次迭代:\\
    $p^{(9)}=-\nabla f(x^{(9)})=\begin{bmatrix} \frac{1}{16}\\\frac{1}{8}\\\end{bmatrix},||p^{(9)}||=\frac{\sqrt{5}}{16}>\varepsilon$\\
    $\lambda_9=\dfrac{\nabla f(x^{(9)})^T\nabla f(x^{(9)})}{\nabla f(x^{(9)})^TH\nabla f(x^{(9)})}=\frac{1}{2},x^{(10)}=x^{(9)}+\lambda_9p^{(9)}=\begin{bmatrix} \frac{125}{32}\\\frac{63}{32}\\\end{bmatrix}$\\
    第11次迭代:\\
    $p^{(10)}=-\nabla f(x^{(10)})=\begin{bmatrix} \frac{1}{8}\\-\frac{1}{16}\\\end{bmatrix},||p^{(10)}||=\frac{\sqrt{5}}{16}>\varepsilon$\\
    $\lambda_{10}=\dfrac{\nabla f(x^{(10)})^T\nabla f(x^{(10)})}{\nabla f(x^{(10)})^TH\nabla f(x^{(10)})}=\frac{1}{4},x^{(11)}=x^{(10)}+\lambda_{10}p^{(10)}=\begin{bmatrix} \frac{63}{16}\\\frac{125}{64}\\\end{bmatrix}$\\
    第12次迭代:\\
    $p^{(11)}=-\nabla f(x^{(11)})=\begin{bmatrix} \frac{1}{16}\\\frac{1}{8}\\\end{bmatrix},||p^{(11)}||=\frac{\sqrt{5}}{32}<\varepsilon,\text{结束计算}$\\
    故最优点为$\begin{bmatrix} \frac{63}{16}\\\frac{125}{64}\\\end{bmatrix}$,目标函数最小值$f_{min}=(\frac{63}{16})^2+2\times(\frac{125}{64})^2-2\times\frac{63}{16}\times\frac{125}{64}-4\times\frac{63}{16}=-7.998$
\end{solution}

\begin{problem}{1.2$\bigstar$}
    $\min f(x)=\frac{3}{2}x_1^2+\frac{1}{2}x_2^2-x_1x_2-2x_1$\\
    初始迭代点${x^{(0)}}=[-2,4]^T$
\end{problem}
\begin{solution}
    $\nabla f(x)=\begin{bmatrix}
        3x_1-x_2-2  \\
        x_2-x_1  \\
    \end{bmatrix}, H=\nabla^2 f(x)=\begin{bmatrix}
        3   & -1  \\
        -1  & 1  \\
    \end{bmatrix}$,
    设$\varepsilon$为0.1\\
    第1次迭代:\\
    $p^{(0)}=-\nabla f(x^{(0)})=\begin{bmatrix} 12\\-6\\\end{bmatrix},||p^{(0)}||=6\sqrt{5}>\varepsilon$\\
    $\lambda_0=\dfrac{\nabla f(x^{(0)})^T\nabla f(x^{(0)})}{\nabla f(x^{(0)})^TH\nabla f(x^{(0)})}=\frac{5}{17},x^{(1)}=x^{(0)}+\lambda_0p^{(0)}=\begin{bmatrix} \frac{26}{17}\\\frac{38}{17}\\\end{bmatrix}$\\
    第2次迭代:\\
    $p^{(1)}=-\nabla f(x^{(1)})=\begin{bmatrix} -\frac{6}{17}\\-\frac{12}{17}\\\end{bmatrix},||p^{(1)}||=\frac{6}{17}\sqrt{5}>\varepsilon$\\
    $\lambda_1=\dfrac{\nabla f(x^{(1)})^T\nabla f(x^{(1)})}{\nabla f(x^{(1)})^TH\nabla f(x^{(1)})}=\frac{5}{3},x^{(2)}=x^{(1)}+\lambda_1p^{(1)}=\begin{bmatrix} \frac{16}{17}\\\frac{18}{17}\\\end{bmatrix}$\\
    第3次迭代:\\
    $p^{(2)}=-\nabla f(x^{(2)})=\begin{bmatrix} \frac{4}{17}\\-\frac{2}{17}\\\end{bmatrix},||p^{(2)}||=\frac{2}{17}\sqrt{5}>\varepsilon$\\
    $\lambda_2=\dfrac{\nabla f(x^{(2)})^T\nabla f(x^{(2)})}{\nabla f(x^{(2)})^TH\nabla f(x^{(2)})}=\frac{5}{17},x^{(3)}=x^{(2)}+\lambda_2p^{(2)}=\begin{bmatrix} \frac{292}{289}\\\frac{296}{289}\\\end{bmatrix}$\\
    第4次迭代:\\
    $p^{(3)}=-\nabla f(x^{(3)})=\begin{bmatrix} -\frac{2}{289}\\-\frac{4}{289}\\\end{bmatrix},||p^{(3)}||=\frac{2\sqrt{5}}{289}<\varepsilon,\text{结束计算}$\\
    故最优点为$\begin{bmatrix} \frac{292}{289}\\\frac{296}{289}\\\end{bmatrix}$,目标函数最小值$f_{min}=\frac{3}{2}\times(\frac{292}{289})^2+\frac{1}{2}\times(\frac{296}{289})^2-\frac{292}{289}\times\frac{296}{289}-2\times\frac{292}{289}=-0.999$
\end{solution}

\subsection{共轭梯度法}

\subsubsection{共轭梯度法求解}

\begin{problem}{1.1$\bigstar$}
    $\min f(x)=x_1^2+4x_2^2-2x_1x_2+x_1-3x_2$\\
    初始迭代点${x^{(0)}}=[1,1]^T$
\end{problem}
\begin{solution}
    $\nabla f(x)=\begin{bmatrix}
        2x_1-2x_2+1  \\
        8x_2-2x_1-3  \\
    \end{bmatrix}, A=\nabla^2 f(x)=\begin{bmatrix}
        2   & -2  \\
        -2  & 8  \\
    \end{bmatrix}$,
    设$\varepsilon$为0.1,则\\
    $g_0=\nabla f(x^{(0)})=\begin{bmatrix} 1\\3\\\end{bmatrix},||g_0||=\sqrt{10}>\varepsilon$\\
    第1次迭代:\\
    $p^{(0)}=-g_0=\begin{bmatrix} -1\\-3\\\end{bmatrix}$\\
    $\lambda_0=-\dfrac{g_0^Tp^{(0)}}{(p^{(0)})^TAp^{(0)}}=\frac{5}{31}$\\
    $x^{(1)}=x^{(0)}+\lambda_0p^{(0)}=\begin{bmatrix} \frac{26}{31}\\\frac{16}{31}\\\end{bmatrix}$\\
    $g_1=\nabla f(x^{(1)})=\begin{bmatrix} \frac{51}{31}\\-\frac{17}{31}\\\end{bmatrix},||g_1||=\frac{\sqrt{2890}}{31}=1.7>\varepsilon$\\
    第2次迭代:\\
    $\beta_0=\dfrac{g_1^Tg_1}{g_0^Tg_0}=\frac{289}{961}$\\
    $p^{(1)}=-g_1+\beta_0p^{(0)}=\begin{bmatrix} -\frac{1870}{961}\\-\frac{340}{961}\\\end{bmatrix}$\\
    $\lambda_1=-\dfrac{g_1^Tp^{(1)}}{(p^{(1)})^TAp^{(1)}}=\frac{31}{60}$\\
    $x^{(2)}=x^{(1)}+\lambda_1p^{(1)}=\begin{bmatrix} -\frac{1}{6}\\\frac{1}{3}\\\end{bmatrix}$\\
    $g_2=\nabla f(x^{(2)})=0,||g_2||=0<\varepsilon,\text{结束计算}$\\
    故最优点为$\begin{bmatrix} -\frac{1}{6}\\\frac{1}{3}\\\end{bmatrix}$,目标函数最小值$f_{min}=(-\frac{1}{6})^2+4\times(\frac{1}{3})^2-2\times(-\frac{1}{6})\times\frac{1}{3}+(-\frac{1}{6})-3\times\frac{1}{3}=-\frac{7}{12}$
\end{solution}

\begin{problem}{1.2$\bigstar$}
    $\min f(x)=(x_1-2)^2+2(x_2-1)^2$\\
    初始迭代点${x^{(0)}}=[1,3]^T$
\end{problem}
\begin{solution}
    $\nabla f(x)=\begin{bmatrix}
        2x_1-4  \\
        4x_2-4  \\
    \end{bmatrix}, A=\nabla^2 f(x)=\begin{bmatrix}
        2  & 0  \\
        0  & 4  \\
    \end{bmatrix}$,
    设$\varepsilon$为0.1,则\\
    $g_0=\nabla f(x^{(0)})=\begin{bmatrix} -2\\8\\\end{bmatrix},||g_0||=2\sqrt{17}>\varepsilon$\\
    第1次迭代:\\
    $p^{(0)}=-g_0=\begin{bmatrix} 2\\-8\\\end{bmatrix}$\\
    $\lambda_0=-\dfrac{g_0^Tp^{(0)}}{(p^{(0)})^TAp^{(0)}}=\frac{17}{66}$\\
    $x^{(1)}=x^{(0)}+\lambda_0p^{(0)}=\begin{bmatrix} \frac{50}{33}\\\frac{31}{33}\\\end{bmatrix}$\\
    $g_1=\nabla f(x^{(1)})=\begin{bmatrix} -\frac{32}{33}\\-\frac{8}{33}\\\end{bmatrix},||g_1||=\frac{8}{33}\sqrt{17}>\varepsilon$\\
    第2次迭代:\\
    $\beta_0=\dfrac{g_1^Tg_1}{g_0^Tg_0}=\frac{16}{1089}$\\
    $p^{(1)}=-g_1+\beta_0p^{(0)}=\begin{bmatrix} \frac{1088}{1089}\\\frac{136}{1089}\\\end{bmatrix}$\\
    $\lambda_1=-\dfrac{g_1^Tp^{(1)}}{(p^{(1)})^TAp^{(1)}}=\frac{33}{68}$\\
    $x^{(2)}=x^{(1)}+\lambda_1p^{(1)}=\begin{bmatrix} 2\\1\\\end{bmatrix}$\\
    $g_2=\nabla f(x^{(2)})=0,||g_2||=0<\varepsilon,\text{结束计算}$\\
    故最优点为$\begin{bmatrix} 2\\1\\\end{bmatrix}$,目标函数最小值$f_{min}=0$
\end{solution}

\begin{problem}{1.3$\bigstar$}
    $\min f(x)=(x_1+10x_2)^2+5(x_3-x_4)^2+(x_2-2x_3)^4+10(x_1-x_4)^4$\\
    初始迭代点${x^{(0)}}=[3,-1,0,1]^T$
\end{problem}
\begin{solution}
    $\nabla f(x)=\begin{bmatrix}
        2x_1-4  \\
        4x_2-4  \\
    \end{bmatrix}, A=\nabla^2 f(x)=\begin{bmatrix}
        2  & 0  \\
        0  & 4  \\
    \end{bmatrix}$,
    设$\varepsilon$为0.1,则\\
    $g_0=\nabla f(x^{(0)})=\begin{bmatrix} -2\\8\\\end{bmatrix},||g_0||=2\sqrt{17}>\varepsilon$\\
    第1次迭代:\\
    $p^{(0)}=-g_0=\begin{bmatrix} 2\\-8\\\end{bmatrix}$\\
    $\lambda_0=-\dfrac{g_0^Tp^{(0)}}{(p^{(0)})^TAp^{(0)}}=\frac{17}{66}$\\
    $x^{(1)}=x^{(0)}+\lambda_0p^{(0)}=\begin{bmatrix} \frac{50}{33}\\\frac{31}{33}\\\end{bmatrix}$\\
    $g_1=\nabla f(x^{(1)})=\begin{bmatrix} -\frac{32}{33}\\-\frac{8}{33}\\\end{bmatrix},||g_1||=\frac{8}{33}\sqrt{17}>\varepsilon$\\
    第2次迭代:\\
    $\beta_0=\dfrac{g_1^Tg_1}{g_0^Tg_0}=\frac{16}{1089}$\\
    $p^{(1)}=-g_1+\beta_0p^{(0)}=\begin{bmatrix} \frac{1088}{1089}\\\frac{136}{1089}\\\end{bmatrix}$\\
    $\lambda_1=-\dfrac{g_1^Tp^{(1)}}{(p^{(1)})^TAp^{(1)}}=\frac{33}{68}$\\
    $x^{(2)}=x^{(1)}+\lambda_1p^{(1)}=\begin{bmatrix} 2\\1\\\end{bmatrix}$\\
    $g_2=\nabla f(x^{(2)})=0,||g_2||=0<\varepsilon,\text{结束计算}$\\
    故最优点为$\begin{bmatrix} 2\\1\\\end{bmatrix}$,目标函数最小值$f_{min}=0$
\end{solution}

\subsection{牛顿法}

\subsubsection{牛顿法求解}

\begin{problem}{1.1$\bigstar$}
    $\min f(x)=x_1^2+2x_2^2-2x_1x_2-4x_1$\\
    初始迭代点${x^{(0)}}=[1,1]^T$
\end{problem}
\begin{solution}
    $\nabla f(x)=\begin{bmatrix}
        2x_1-2x_2-4  \\
        4x_2-2x_1  \\
    \end{bmatrix}, G=\nabla^2 f(x)=\begin{bmatrix}
        2   & -2  \\
        -2  & 4  \\
    \end{bmatrix}$,\\
    $g_0=\nabla f(x^{(0)})=\begin{bmatrix} -4\\2\\\end{bmatrix},$\\
    $G^{-1}=\begin{bmatrix}
        1   & 1/2\\
        1/2 & 1/2\\
    \end{bmatrix},$\\
    $x^{(1)}=x^{(0)}-G_0^{-1}g_0=\begin{bmatrix} 1\\1\\\end{bmatrix}-\begin{bmatrix}
        1   & 1/2\\
        1/2 & 1/2\\
    \end{bmatrix}\begin{bmatrix} -4\\2\\\end{bmatrix}=\begin{bmatrix} 4\\2\\\end{bmatrix}$\\
    故最优点为$\begin{bmatrix} 4\\2\\\end{bmatrix}$,目标函数最小值$f_{min}=4^2+2\times2^2-2\times4\times2-4\times4=-8$
\end{solution}

\begin{problem}{1.2$\bigstar$}
    $\min f(x)=\frac{3}{2}x_1^2+\frac{1}{2}x_2^2-x_1x_2-2x_1$\\
    初始迭代点${x^{(0)}}=[-2,4]^T$
\end{problem}
\begin{solution}
    $\nabla f(x)=\begin{bmatrix}
        3x_1-x_2-2  \\
        x_2-x_1  \\
    \end{bmatrix}, G=\nabla^2 f(x)=\begin{bmatrix}
        3   & -1  \\
        -1  & 1  \\
    \end{bmatrix}$,\\
    $g_0=\nabla f(x^{(0)})=\begin{bmatrix} -12\\6\\\end{bmatrix},$\\
    $G^{-1}=\begin{bmatrix}
        1/2 & 1/2\\
        1/2 & 3/2\\
    \end{bmatrix},$\\
    $x^{(1)}=x^{(0)}-G_0^{-1}g_0=\begin{bmatrix} -2\\4\\\end{bmatrix}-\begin{bmatrix}
        1/2 & 1/2\\
        1/2 & 3/2\\
    \end{bmatrix}\begin{bmatrix} -12\\6\\\end{bmatrix}=\begin{bmatrix} 1\\1\\\end{bmatrix}$\\
    故最优点为$\begin{bmatrix} 1\\1\\\end{bmatrix}$,目标函数最小值$f_{min}=\frac{3}{2}+\frac{1}{2}-1-2=-1$
\end{solution}

\begin{problem}{1.3$\bigstar$}
    $\min f(x)=100(x_2-x_1^2)^2+(1-x_1)^2$\\
    初始迭代点${x^{(0)}}=[-1.2,1]^T$
\end{problem}
\begin{solution}
    $\nabla f(x)=\begin{bmatrix}
        -400x_1(x_2-x_1^2)+2x_1-2  \\
        200(x_2-x_1^2)  \\
    \end{bmatrix}$, \\
    $\nabla^2 f(x)=\begin{bmatrix}
        1200x_1^2-400x_2+2 & -400x_1  \\
        -400x_1  & 200  \\
    \end{bmatrix}$,\\
    第一次迭代:\\
    $g_0=\nabla f(x^{(0)})=\begin{bmatrix} -215.6\\-88\\\end{bmatrix},||g_0||>\varepsilon$\\
    $G_0=\nabla^2 f(x^{(0)})=\begin{bmatrix}
        1330 & 480\\
        480 & 200\\
    \end{bmatrix},$\\
    $G_0^{-1}=\begin{bmatrix}
        1/178 & -6/445\\
        -6/445 & 133/3560\\
    \end{bmatrix},$\\
    $x^{(1)}=x^{(0)}-G_0^{-1}g_0=\begin{bmatrix} -1.2\\1\\\end{bmatrix}-\begin{bmatrix}
        1/178 & -6/445\\
        -6/445 & 133/3560\\
    \end{bmatrix}\begin{bmatrix} -215.6\\-88\\\end{bmatrix}=\begin{bmatrix} -523/445\\943/683\\\end{bmatrix}$\\
    第二次迭代:\\
    $g_1=\nabla f(x^{(1)})=\begin{bmatrix} -2871/619\\-316/2583\\\end{bmatrix},||g_1||>\varepsilon$\\
    $G_1=\nabla^2 f(x^{(1)})=\begin{bmatrix}
        12180/11 & 41840/89\\
        41840/89 & 200\\
    \end{bmatrix},$\\
    $G_1^{-1}=\begin{bmatrix}
        1/178 & -6/445\\
        -6/445 & 133/3560\\
    \end{bmatrix},$\\
    $x^{(2)}=x^{(1)}-G_1^{-1}g_1=\begin{bmatrix} -1.2\\1\\\end{bmatrix}-\begin{bmatrix}
        1/178 & -6/445\\
        -6/445 & 133/3560\\
    \end{bmatrix}\begin{bmatrix} -215.6\\-88\\\end{bmatrix}=\begin{bmatrix} -\frac{523}{445}\\\frac{943}{683}\\\end{bmatrix}$\\
    第三次迭代:\\
    $g_2=\nabla f(x^{(2)})=\begin{bmatrix} -215.6\\-88\\\end{bmatrix},||g_2||=2\sqrt{17}>\varepsilon$\\
    $G_2=\nabla^2 f(x^{(2)})=\begin{bmatrix}
        1330 & 480\\
        480 & 200\\
    \end{bmatrix},$\\
    $G_2^{-1}=\begin{bmatrix}
        1/178 & -6/445\\
        -6/445 & 133/3560\\
    \end{bmatrix},$\\
    $x^{(3)}=x^{(2)}-G_2^{-1}g_2=\begin{bmatrix} -1.2\\1\\\end{bmatrix}-\begin{bmatrix}
        1/178 & -6/445\\
        -6/445 & 133/3560\\
    \end{bmatrix}\begin{bmatrix} -215.6\\-88\\\end{bmatrix}=\begin{bmatrix} -\frac{523}{445}\\\frac{943}{683}\\\end{bmatrix}$\\
    故最优点为$\begin{bmatrix} 1\\1\\\end{bmatrix}$,目标函数最小值$f_{min}=\frac{3}{2}+\frac{1}{2}-1-2=-1$
\end{solution}