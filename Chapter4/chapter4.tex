\section{非线性规划问题}

\subsection{凸函数与凸规划}

\subsubsection{凸规划判定}

\begin{problem}{1.1$\bigstar$}
    \begin{mini*}|s|
        {}
        {f(x) = 2x_1^2 + x_2^2 + x_3^2}
        {}
        {}
        \addConstraint{-x_1^2 - x_2^2 + 4}{\geq 0}
        \addConstraint{5x_1 - 4x_2 \ph}{=8}
        \addConstraint{x_1,x_2,x_3}{\geq 0}{.}
    \end{mini*}
\end{problem}
\begin{solution}
    第二、三条件为线性函数,把它们看成凹函数,\\
    第一约束条件的海赛矩阵是$\nabla^2g_1(x)=\begin{bmatrix}
        -2  & 0  \\
        0  & -2  \\
    \end{bmatrix}$,负定,凹函数,\\
    目标函数的海赛矩阵是$\nabla^2g_1(x)=\begin{bmatrix}
        4  & 0 & 0  \\
        0  & 2 & 0  \\
        0  & 0 & 2  \\
    \end{bmatrix}$,正定,为凸函数,\\
    从而可知该问题是凸规划,有唯一最小值解。
\end{solution}
\begin{problem}{1.2$\bigstar$}
    \begin{mini*}|s|
        {}
        {f(x) = x_1 + 3x_2}
        {}
        {}
        \addConstraint{x_1^2 + x_2^2}{\leq 9}
        \addConstraint{x_2}{\geq 0}{.}
    \end{mini*}
\end{problem}
\begin{solution}

    目标函数是线性看书,把其看为凸函数;\\
    第一约束条件两边同乘-1,改为大于等于号后的海赛矩阵是
    $\nabla^2g_1(x)=\begin{bmatrix}
        -2  & 0  \\
        0  & -2  \\
    \end{bmatrix}$,负定,凹函数,\\
    第二约束条件为线性函数,把其看成凹函数\\
    从而可知该问题是凸规划,有唯一最小值解。
\end{solution}

\subsection{最优性条件}


\subsubsection{Kuhn-Tucker点判定}

\begin{problem}{1.1$\bigstar$}
    \begin{mini*}|s|
        {}
        {f(x) = (x_1 - 3)^2 + (x_2 - 2)^2}
        {}
        {}
        \addConstraint{x_1^2 + x_2^2}{\leq 5}
        \addConstraint{x_1 + 2x_2}{=4}
        \addConstraint{x_1,x_2}{\geq 0}{.}
    \end{mini*}
    $$\bar{x}=[2\ 1]^T$$
\end{problem}
\begin{solution}
    令
    \begin{align*}
        g_1(x)&=-x_1^2 - x_2^2 + 5\geq0,\\
        g_2(x)&=x_1\geq0\\
        g_3(x)&=x_2\geq0\\
        h(x)&=x_1 + 2x_2-4=0
    \end{align*}
    将$\bar{x}=[2\ 1]^T$代入$g(x)$,易见$g_1(x)$等式成立,$g_2(x)、g_3(x)$不等式成立,则有$I(\bar{x})=\{1\}$,\\
    $\nabla f(\bar{x})=[2x_1-6\ 2x_2-4]^T=[-2\ -2]^T$,\\
    $\nabla g_1(\bar{x})=[-2x_1\ -2x_2]^T=[-4\ -2]^T$,\\
    $\nabla g_2(\bar{x})=[1\ 0]^T$,\\
    $\nabla g_3(\bar{x})=[0\ 1]^T$,\\
    $\nabla h(\bar{x})=[1\ 2]^T$,\\
    根据Kuhn-Tucker一阶必要条件,则有
    $$\left\{
    \begin{aligned}
        -2 - (-4)\lambda_1 - \mu_1 &=0\\
        -2 - (-2)\lambda_1 - 2\mu_1 &=0\\
        \lambda_1&\geq0\\
        \lambda_2=\lambda_3&=0
    \end{aligned}\right.$$
    解得$\lambda_1=\frac{1}{3},\mu_1=-\frac{2}{3}$,故点$[2\ 1]^T$是原问题Kuhn-Tucker点。
\end{solution}

\begin{problem}{1.2$\bigstar$}
    \begin{mini*}|s|
        {}
        {f(x) = (x_1 - 2)^2 + x_2^2}
        {}
        {}
        \addConstraint{x_1 - x_2^2 }{\geq 0}
        \addConstraint{-x_1 + x_2}{\geq 0}
    \end{mini*}
    $$\bar{x}^{(1)}=[0\ ,0]^T,\bar{x}^{(2)}=[1\ 1]^T$$
\end{problem}
\begin{solution}
    令
    \begin{align*}
        g_1(x)&=x_1 - x_2^2 \geq 0\\
        g_2(x)&=-x_1 + x_2 \geq 0\\
    \end{align*}
    (1)先验证$\bar{x}^{(1)}=[0\ ,0]^T$\\
    将$\bar{x}^{(1)}=[0\ 0]^T$代入$g(x)$,易见$g_1(x),g_2(x)$等式成立,则有$I(\bar{x}^{(1)})=\{1,2\}$,\\
    $\nabla f(\bar{x}^{(1)})=[2x_1-4\ 2x_2]^T=[-4\ 0]^T$,\\
    $\nabla g_1(\bar{x}^{(1)})=[1\ -2x_2]^T=[1\ 0]^T$,\\
    $\nabla g_2(\bar{x}^{(1)})=[-1\ 1]^T$,\\
    根据Kuhn-Tucker一阶必要条件,则有
    $$\left\{
    \begin{aligned}
        -4 - \lambda_1 - (-1)\lambda_2 &=0\\
        0 \ph - \lambda_2 &=0\\
        \lambda_1&\geq0\\
        \lambda_2&\geq0
    \end{aligned}\right.$$
    解得$\lambda_1=4,\lambda_2=0$,故点$[0\ 0]^T$是原问题Kuhn-Tucker点。\\
    (2)再验证$\bar{x}^{(2)}=[1\ 1]^T$\\
    将$\bar{x}^{(2)}=[1\ 1]^T$代入$g(x)$,易见$g_1(x),g_2(x)$等式成立,则有$I(\bar{x}^{(2)})=\{1,2\}$,\\
    $\nabla f(\bar{x}^{(2)})=[2x_1-4\ 2x_2]^T=[-2\ 2]^T$,\\
    $\nabla g_1(\bar{x}^{(2)})=[1\ -2x_2]^T=[1\ -2]^T$,\\
    $\nabla g_2(\bar{x}^{(2)})=[-1\ 1]^T$,\\
    根据Kuhn-Tucker一阶必要条件,则有
    $$\left\{
    \begin{aligned}
        -2 - \lambda_1 - (-1)\lambda_2 &=0\\
        2 - (-2)\lambda_1 - \lambda_2 &=0\\
        \lambda_1&\geq0\\
        \lambda_2&\geq0
    \end{aligned}\right.$$
    解得$\lambda_1=0,\lambda_2=2$,故点$[1\ 1]^T$是原问题Kuhn-Tucker点。
\end{solution}

\subsubsection{Kuhn-Tucker点求解,并验证是否最优解}

\begin{problem}{2.1$\bigstar$}
    \begin{mini*}|s|
        {}
        {f(x) = (x_1 - 1)^2 + (x_1 - 2)^2}
        {}
        {}
        \addConstraint{-x_1 + x_2}{= 1}
        \addConstraint{x_1 + x_2}{= 2}
        \addConstraint{x_1,x_2}{\geq 0}{.}
    \end{mini*}
\end{problem}
\begin{solution}
    令
    \begin{align*}
        g_1(x)&=x_1\geq0\\
        g_2(x)&=x_2\geq0\\
        h_1(x)&=-x_1 + x_2 - 1 =0\\
        h_2(x)&=x_1 + x_2 - 2 = 0\\
    \end{align*}
    则有
    \begin{align*}
        \nabla f(x)&=[2x_1-2\ 2x_2-4]^T\\
        \nabla g_1(x)&=[1\ 0]^T\\
        \nabla g_2(x)&=[0\ 1]^T\\
        \nabla h_1(x)&=[-1\ 1]^T\\
        \nabla h_2(x)&=[1\ 1]^T\\
    \end{align*}
    根据Kuhn-Tucker一阶必要条件,则有
    $$\left\{
    \begin{aligned}
        2x_1 - 2 - \lambda_1 - (-1)\mu_1 - \mu_2 &=0\\
        2x_2 - 4 - \lambda_2 - \mu_1 - \mu_2 &=0\\
        \lambda_1(x_1)&=0\\
        \lambda_2(x_2)&=0\\
        \lambda_1&\geq0\\
        \lambda_2&\geq0
    \end{aligned}\right.$$
    解得$\lambda_1=4,\lambda_2=0$,故点$[0\ 0]^T$是原问题Kuhn-Tucker点。
\end{solution}

\begin{problem}{2.2$\bigstar$}
    \begin{mini*}|s|
        {}
        {f(x) = x_1^2 - x_2 - 3x_3}
        {}
        {}
        \addConstraint{-x_1 - x_2 - x_3}{\geq 0}
        \addConstraint{x_1^2 + 2x_2 - x_3}{=0}
    \end{mini*}
\end{problem}
\begin{solution}
    令
    \begin{align*}
        g(x)&=-x_1 - x_2 - x_3\geq0\\
        h(x)&=x_1^2 + 2x_2 - x_3 =0\\
    \end{align*}
    则有
    \begin{align*}
        \nabla f(x)&=[2x_1\ -1\ -3]^T\\
        \nabla g(x)&=[-1\ -1\ -1]^T\\
        \nabla h(x)&=[2x_1\ 2\ -1]^T\\
    \end{align*}
    根据Kuhn-Tucker一阶必要条件,则有
    $$\left\{
    \begin{aligned}
        2x_1 - (-1)\lambda - (2x_1)\mu &=0\\
        -1 - (-1)\lambda - 2\mu  &=0\\
        -3 - (-1)\lambda - (-1)\mu  &=0\\
        \lambda(-x_1 - x_2 - x_3)&=0\\
        \lambda&\geq0
    \end{aligned}\right.$$
    解得$\lambda=\frac{7}{3},\mu=\frac{2}{3},x_1=-\frac{7}{2}$。
    再将$x_1$代入下式求$x_2$与$x_3$
    $$\left\{
    \begin{aligned}
        -x_1 - x_2 - x_3 &=0\\
        x_1^2 + 2x_2 - x_3&=0\\
        \lambda_1&\geq0
    \end{aligned}\right.$$
    解得$[x_1\ x_2\ x_3]^T=[-\frac{7}{2}\  -\frac{35}{12}\ \frac{77}{12}]^T$
    考虑拉格朗日函数
    \begin{align*}
        L(x,\lambda,\mu)&=f(x)-\lambda g(x)-\mu h(x)\\
        &=x_1^2 - x_2 - 3x_3 - \frac{7}{3}(-x_1 - x_2 - x_3) - \frac{2}{3}(x_1^2 + 2x_2 - x_3)\\
        &=\frac{1}{3}x_1^2+\frac{7}{3}x_1\\
        \nabla^2 L(x,\lambda,\mu)&=\begin{bmatrix}
            \frac{2}{3}  & 0 & 0  \\
            0  & 0 & 0  \\
            0  & 0 & 0  \\
        \end{bmatrix}\\
    \end{align*}
    海赛矩阵半正定,故点$[-\frac{7}{2}\  -\frac{35}{12}\ \frac{77}{12}]^T$是最优解

\end{solution}

\subsubsection{Fritz-John点判定}

\begin{problem}{3.1$\bigstar$}
    \begin{mini*}|s|
        {}
        {f(x) = -x_2}
        {}
        {}
        \addConstraint{-2x_1 + (2 - x_2)^3}{\geq 0}
        \addConstraint{x_1 \ph\ph}{\geq 0.}
    \end{mini*}
    $$\bar{x}=[0\ 2]^T$$
\end{problem}
\begin{solution}
    令
    \begin{align*}
        g_1(x)&=-2x_1 + (2 - x_2)^3\geq 0\\
        g_2(x)&=x_1\geq0\\
    \end{align*}
    将$\bar{x}=[0\ 2]^T$代入$g(x)$,易见$g_1(x),g_2(x)$等式成立,则有$I(\bar{x})=\{1,2\}$,\\
    $\nabla f(\bar{x})=[0\ -1]^T$,\\
    $\nabla g_1(\bar{x})=[-2\ -3x_2^2+12x_2]^T=[-2\ 12]^T$,\\
    $\nabla g_2(\bar{x})=[1\ 0]^T$,\\
    根据Fritz-John一阶必要条件,则有
    $$\left\{
    \begin{aligned}
        (0)\lambda_0 - (-2)\lambda_1 - \lambda_2 &=0\\
        (-1)\lambda_0 - 12\lambda_1 - (0)\lambda_2 &=0\\
    \end{aligned}\right.$$
    该方程组中$\lambda_0$与$\lambda_1$异号,找不到非负解,故点$[0\ 2]^T$不是原问题Fritz-John点。
\end{solution}
\begin{problem}{3.2$\bigstar$}
    \begin{mini*}|s|
        {}
        {f(x) = -x_1}
        {}
        {}
        \addConstraint{(1 - x_1)^2 + x_2}{\geq 0}
        \addConstraint{-x_2}{\geq 0}{.}
    \end{mini*}
    $$\bar{x}=[1\ 0]^T$$
\end{problem}
\begin{solution}
    令
    \begin{align*}
        g_1(x)&=(1 - x_1)^2 + x_2\geq 0\\
        g_2(x)&=x_2\geq0\\
    \end{align*}
    将$\bar{x}=[1\ 0]^T$代入$g(x)$,易见$g_1(x),g_2(x)$等式成立,则有$I(\bar{x})=\{1,2\}$,\\
    $\nabla f(\bar{x})=[-1\ 0]^T$,\\
    $\nabla g_1(\bar{x})=[2x_1-2\ 1]^T=[0\ 1]^T$,\\
    $\nabla g_2(\bar{x})=[0\ 1]^T$,\\
    根据Fritz-John一阶必要条件,则有
    $$\left\{
    \begin{aligned}
        (-1)\lambda_0 - (0)\lambda_1 - (0)\lambda_2 &=0\\
        (0)\lambda_0 - \lambda_1 - \lambda_2 &=0\\
    \end{aligned}\right.$$
    该方程组中有无穷非负解,如$[0\ 1\ 1]$,故点$[1\ 0]^T$是原问题Fritz-John点。
\end{solution}

\subsection{一维搜索算法}

\subsubsection{黄金分割法}

\begin{problem}{1.1$\bigstar$}
    $f(x)=x^3-3x+2$,区间为$[0,3]$
\end{problem}
\begin{solution}
    $f'(x)=3x^2-3=0$,解得x=1或-1,仅1在[0,3]区间内,故f(x)在[0,3]上仅一个驻点,是[0,3]上的单峰函数。
    采用黄金分割法,迭代计算
    \begin{flalign*}
        &\text{第1次迭代:}\\
        &a_1=0,b_1=3,\lambda_1=0.382\times3=1.146,\mu_1=0.618\times3=1.854\\
        &f(\lambda_1)=(1.146)^3-3\times1.146+2=0.067\\
        &f(\mu_1)=(1.854)^3-3\times1.854+2=2.811\\
        &f(\lambda_1)<f(\mu_1)\\
        &\text{第2次迭代:}\\
        &a_2=0,b_2=\mu_1=1.854,\lambda_2=0.382\times1.854=0.708,\mu_2=\lambda_1=1.146\\
        &f(\lambda_2)=(0.708)^3-3\times0.708+2=0.231\\
        &f(\mu_2)=f(\lambda_1)=0.067\\
        &f(\lambda_2)>f(\mu_2)\\
        &\text{第3次迭代:}\\
        &a_3=\lambda_2=0.708,b_3=\mu_1=1.854,\lambda_3=\mu_2=1.146,\mu_3=0.618\times(1.854-0.708)=1.416\\
        &f(\lambda_3)=f(\mu_2)=0.067\\
        &f(\mu_3)=(1.416)^3-3\times1.416+2=0.591\\
        &f(\lambda_3)<f(\mu_3)\\
        &\text{第4次迭代:}\\
        &a_4=0.708,b_4=\mu_3=1.416,\lambda_4=0.382\times(1.416-0.708)=0.978,\mu_4=\lambda_3=1.146\\
        &f(\lambda_4)=(0.978)^3-3\times0.978+2=0.001\\
        &f(\mu_4)=f(\lambda_3)=0.067\\
        &f(\lambda_4)<f(\mu_4)\\
        &\text{第5次迭代:}\\
        &a_5=0.708,b_5=\mu_4=1.146,\lambda_5=0.382\times(1.146-0.708)=0.875,\mu_5=\lambda_4=0.978\\
        &f(\lambda_5)=(0.875)^3-3\times0.875+2=0.045\\
        &f(\mu_5)=f(\lambda_4)=0.001\\
        &f(\lambda_5)>f(\mu_5)\\
        &\text{第6次迭代:}\\
        &a_6=\lambda_5=0.875,b_6=1.146,\lambda_6=\mu_5=0.978,\mu_6=0.618\times(1.146-0.875)=1.042\\
        &f(\lambda_6)=f(\mu_5)=0.001\\
        &f(\mu_6)=(1.042)^3-3\times1.042+2=0.005\\
        &f(\lambda_6)<f(\mu_6)\\
        &\text{第7次迭代:}\\
        &a_7=0.875,b_7=\mu_6=1.042,\lambda_7=0.382\times(1.042-0.875)=0.939,\mu_7=\lambda_6=0.978\\
        &f(\lambda_7)=(0.939)^3-3\times0.939+2=0.011\\
        &f(\mu_7)=f(\lambda_6)=0.001\\
        &f(\lambda_7)>f(\mu_7)\\
    \end{flalign*}
    \begin{flalign*}
        &\text{第8次迭代:}\\
        &a_8=\lambda_7=0.939,b_8=1.042,\lambda_8=\mu_7=0.978,\mu_8==0.618\times(1.042-0.939)=1.003\\
        &f(\lambda_8)=f(\mu_7)=0.001\\
        &f(\mu_8)=(1.003)^3-3\times1.003+2=0.000\\
        &f(\lambda_8)>f(\mu_8)\\
        &\text{第9次迭代:}\\
        &a_9=\lambda_8=0.978,b_9=1.042,b_9-a_9=0.064<0.1,\text{结束计算}
    \end{flalign*}
    极小值点为1/2*(0.978+1.042)=1.01\\
    列表结果如下
    \begin{center}
        \begin{tblr}{
                hlines,
                vlines,
                row{1} = {mode = math},
            }
            k  & a_k      & b_k    &\lambda_k&\mu_k&f(\lambda_k)&f(\mu_k)& \text{不等式方向} \\
            1  &  0       &  3       & 1.146    &  1.854 &  0.067     & 2.811    &   <    \\
            2  &  0       &  1.854   & 0.708    &  1.146 &  0.231     & 0.067    &   >     \\
            3  &  0.708   &  1.854   & 1.146    &  1.416 &  0.067     & 0.591    &   <     \\
            4  &  0.708   &  1.416   & 0.978    &  1.146 &  0.001     & 0.067    &   <     \\
            5  &  0.708   &  1.146   & 0.875    &  0.978 &  0.045     & 0.001    &   >     \\
            6  &  0.875   &  1.146   & 0.978    &  1.042 &  0.001     & 0.005    &   <     \\
            7  &  0.875   &  1.042   & 0.939    &  0.978 &  0.011     & 0.001    &   >     \\
            8  &  0.939   &  1.042   & 0.978    &  1.003 &  0.001     & 0.000    &   >     \\
            9  &  0.978   &  1.042   &          &        &            &          &       \\
        \end{tblr}
    \end{center}
\end{solution}

\begin{problem}{1.2$\bigstar$}
    $f(x)=e^2-e^{-x}$,区间为$[0,1]$
\end{problem}
\begin{solution}
    在[0,1]区间内,易见f(x)单调,故f(x)是[0,1]上的单峰函数。
    采用黄金分割法,迭代计算
    \begin{flalign*}
        &\text{第1次迭代:}\\
        &a_1=0,b_1=1,\lambda_1=0.382,\mu_1=0.618\\
        &f(\lambda_1)=e^2-e^{(-0.382)}=6.707\\
        &f(\mu_1)=e^2-e^{(-0.618)}=6.850\\
        &f(\lambda_1)<f(\mu_1)\\
        &\text{第2次迭代:}\\
        &a_2=0,b_2=\mu_1=0.618,\lambda_2=0.382\times0.618=0.236,\mu_2=\lambda_1=0.382\\
        &f(\lambda_2)=e^2-e^{(-0.236)}=6.599\\
        &f(\mu_2)=f(\lambda_1)=6.707\\
        &f(\lambda_2)<f(\mu_2)\\
        &\text{第3次迭代:}\\
        &a_3=0,b_3=\mu_2=0.382,\lambda_3=0.382\times0.382=0.146,\mu_3=\lambda_2=0.236\\
        &f(\lambda_3)=e^2-e^{(-0.146)}=6.525\\
        &f(\mu_3)=f(\lambda_2)=6.599\\
        &f(\lambda_3)<f(\mu_3)\\
        &\text{第4次迭代:}\\
        &a_4=0,b_4=\mu_3=0.236,\lambda_4=0.382\times0.236=0.090,\mu_4=\lambda_3=0.146\\
        &f(\lambda_4)=e^2-e^{(-0.090)}=6.475\\
        &f(\mu_4)=f(\lambda_3)=6.525\\
        &f(\lambda_4)<f(\mu_4)\\
        &\text{第5次迭代:}\\
        &a_5=0,b_5=\mu_4=0.146,\lambda_5=0.382\times0.146=0.056,\mu_5=\lambda_4=0.090\\
        &f(\lambda_5)=e^2-e^{(-0.056)}=6.444\\
        &f(\mu_5)=f(\lambda_4)=6.475\\
        &f(\lambda_5)<f(\mu_5)\\
        &\text{第6次迭代:}\\
        &a_6=0,b_6=\mu_5=0.090,b_6-a_6=0.090<0.1,\text{结束计算}
    \end{flalign*}
    极小值点为1/2*(0+0.090)=0.045\\
    列表结果如下
    \begin{center}
        \begin{tblr}{
                hlines,
                vlines,
                row{1} = {mode = math},
            }
            k  & a_k      & b_k    &\lambda_k&\mu_k&f(\lambda_k)&f(\mu_k)& \text{不等式方向} \\
            1  &  0       &  1       & 0.382    &  0.618 &  6.707     & 6.850    &     <    \\
            2  &  0       &  0.618   & 0.236    &  0.382 &  6.599     & 6.707    &     <    \\
            3  &  0       &  0.382   & 0.146    &  0.236 &  6.525     & 6.599    &     <    \\
            4  &  0       &  0.236   & 0.090    &  0.146 &  6.475     & 6.525    &     <    \\
            5  &  0       &  0.146   & 0.056    &  0.090 &  6.444     & 6.475    &     <    \\
            6  &  0       &  0.090   &          &        &            &          &          \\
        \end{tblr}
    \end{center}
\end{solution}

\begin{problem}{1.3$\bigstar$}
    $f(x)=1-xe^{-x^2}$,区间为$[0,1]$
\end{problem}
\begin{solution}
    $f'(x)=e^{-x^2}(2x^2-1)=0$,解得$x=\frac{\sqrt{2}}{2}$或$-\frac{\sqrt{2}}{2}$,仅前者在[0,1]区间内,故f(x)在[0,1]上仅一个驻点,是[0,1]上的单峰函数。
    采用黄金分割法,迭代计算
    \begin{flalign*}
        &\text{第1次迭代:}\\
        &a_1=0,b_1=1,\lambda_1=0.382,\mu_1=0.618\\
        &f(\lambda_1)=1-0.382e^{-0.382^2}=0.670\\
        &f(\mu_1)=1-0.618e^{-0.618^2}=0.578\\
        &f(\lambda_1)>f(\mu_1)\\
        &\text{第2次迭代:}\\
        &a_2=\lambda_1=0.382,b_2=1,\lambda_2=\mu_1=0.618,\mu_2=0.382+0.618(1-0.382)=0.764\\
        &f(\lambda_2)=f(\mu_1)=0.578\\
        &f(\mu_2)=1-0.764e^{-0.764^2}=0.574\\
        &f(\lambda_2)>f(\mu_2)\\
        &\text{第3次迭代:}\\
        &a_3=\lambda_2=0.618,b_3=1,\lambda_3=\mu_2=0.764,\mu_3=0.618+0.618\times(1-0.618)=0.854\\
        &f(\lambda_3)=f(mu_2)=0.574\\
        &f(\mu_3)=1-0.854e^{-0.854^2}=0.588\\
        &f(\lambda_3)<f(\mu_3)\\
        \end{flalign*}
        \begin{flalign*}
        &\text{第4次迭代:}\\
        &a_4=0.618,b_4=\mu_3=0.854,\lambda_4=0.618+0.382\times(0.854-0.618)=0.708,\mu_4=\lambda_3=0.764\\
        &f(\lambda_4)=1-0.708e^{-0.708^2}=0.571\\
        &f(\mu_4)=f(\lambda_3)=0.574\\
        &f(\lambda_4)<f(\mu_4)\\
        &\text{第5次迭代:}\\
        &a_5=0.618,b_5=\mu_4=0.764,\lambda_5=0.618+0.382\times(0.764-0.618)=0.674,\mu_5=\lambda_4=0.708\\
        &f(\lambda_5)=1-0.674e^{-0.674^2}=0.572\\
        &f(\mu_5)=f(\lambda_4)=0.571\\
        &f(\lambda_5)>f(\mu_5)\\
        &\text{第6次迭代:}\\
        &a_6=\lambda_5=0.674,b_6=0.764,b_6-a_6=0.090<0.1,\text{结束计算}
    \end{flalign*}
    极小值点为1/2*(0.674+0.764)=0.719\\
    列表结果如下
    \begin{center}
        \begin{tblr}{
                hlines,
                vlines,
                row{1} = {mode = math},
            }
            k  & a_k      & b_k    &\lambda_k&\mu_k&f(\lambda_k)&f(\mu_k)& \text{不等式方向} \\
            1  &  0       &  1       & 0.382    &  0.618 &  0.670     & 0.578    &     >    \\
            2  &  0.382   &  1       & 0.618    &  0.764 &  0.578     & 0.574    &     >    \\
            3  &  0.618   &  1       & 0.764    &  0.854 &  0.574     & 0.588    &     <    \\
            4  &  0.618   &  0.854   & 0.708    &  0.764 &  0.571     & 0.574    &     <    \\
            5  &  0.618   &  0.764   & 0.674    &  0.708 &  0.572     & 0.571    &     >    \\
            6  &  0.674   &  0.764   &          &        &            &          &          \\
        \end{tblr}
    \end{center}
\end{solution}

\subsubsection{分数法}

\begin{problem}{2.1$\bigstar$}
    $f(x)=x^3-3x+2$,区间为$[0,3]$
\end{problem}
\begin{solution}
    $f'(x)=3x^2-3=0$,解得x=1或-1,仅1在[0,3]区间内,故f(x)在[0,3]上仅一个驻点,是[0,3]上的单峰函数。
    采用分数法,$\varepsilon=0.1$,则$F_n\geq3/0.1=30$,n=8,迭代计算
    \begin{flalign*}
        &\text{第1次迭代:}\\
        &a_1=0,b_1=3,\lambda_1=0+\frac{13}{34}(3-0)=1.147,\mu_1=0+\frac{21}{34}(3-0)=1.853\\
        &f(\lambda_1)=(1.147)^3-3\times1.147+2=0.068\\
        &f(\mu_1)=(1.853)^3-3\times1.853+2=2.803\\
        &f(\lambda_1)<f(\mu_1)\\
        &\text{第2次迭代:}\\
        &a_2=0,b_2=\mu_1=1.853,\lambda_2=0+\frac{8}{21}(1.853-0)=0.706,\mu_2=\lambda_1=1.147\\
        &f(\lambda_2)=(0.706)^3-3\times0.706+2=0.234\\
        &f(\mu_2)=f(\lambda_1)=0.068\\
        &f(\lambda_2)>f(\mu_2)\\
        &\text{第3次迭代:}\\
        &a_3=\lambda_2=0.706,b_3=\mu_1=1.853,\lambda_3=\mu_2=1.147,\mu_3=0.706+\frac{8}{13}(1.853-0.706)=1.412\\
        &f(\lambda_3)=f(\mu_2)=0.068\\
        &f(\mu_3)=(1.412)^3-3\times1.412+2=0.579\\
        &f(\lambda_3)<f(\mu_3)\\
        &\text{第4次迭代:}\\
        &a_4=0.706,b_4=\mu_3=1.412,\lambda_4=0.706+\frac{3}{8}(1.412-0.706)=0.971,\mu_4=\lambda_3=1.147\\
        &f(\lambda_4)=(0.971)^3-3\times0.971+2=0.002\\
        &f(\mu_4)=f(\lambda_3)=0.068\\
        &f(\lambda_4)<f(\mu_4)\\
        &\text{第5次迭代:}\\
        &a_5=0.706,b_5=\mu_4=1.147,\lambda_5=0.706+\frac{2}{5}(1.147-0.706)=0.882,\mu_5=\lambda_4=0.971\\
        &f(\lambda_5)=(0.882)^3-3\times0.882+2=0.040\\
        &f(\mu_5)=f(\lambda_4)=0.002\\
        &f(\lambda_5)>f(\mu_5)\\
        &\text{第6次迭代:}\\
        &a_6=\lambda_5=0.882,b_6=1.147,\lambda_6=\mu_5=0.971,\mu_6=0.882+\frac{2}{3}(1.147-0.882)=1.059\\
        &f(\lambda_6)=f(\mu_5)=0.002\\
        &f(\mu_6)=(1.059)^3-3\times1.059+2=0.011\\
        &f(\lambda_6)<f(\mu_6)\\
        &\text{第7次迭代:}\\
        &a_7=0.882,b_7=\mu_6=1.059,\lambda_7=0.882+\frac{1}{2}(1.059-0.882)=0.971,\mu_7=\lambda_6+\delta=0.971+0.01=0.981\\
        &f(\lambda_7)=(0.971)^3-3\times0.971+2=0.002\\
        &f(\mu_7)=(0.981)^3-3\times0.981+2=0.001\\
        &f(\lambda_7)>f(\mu_7)\\
    \end{flalign*}
    \begin{flalign*}
        &\text{第8次迭代:}\\
        &a_8=\lambda_7=0.971,b_8=1.059,b_6-a_6=0.088<0.1,\text{结束计算}
    \end{flalign*}
    极小值点为1/2*(0.971+1.059)=1.015\\
    列表结果如下
    \begin{center}
        \begin{tblr}{
                hlines,
                vlines,
                row{1} = {mode = math},
            }
            k  & a_k      & b_k    &\lambda_k&\mu_k&f(\lambda_k)&f(\mu_k)& \text{不等式方向} \\
            1  &  0       &  3       & 1.147    &  1.853 &  0.068     & 2.803    &   <    \\
            2  &  0       &  1.853   & 0.706    &  1.147 &  0.234     & 0.068    &   >     \\
            3  &  0.706   &  1.853   & 1.147    &  1.412 &  0.068     & 0.579    &   <     \\
            4  &  0.706   &  1.412   & 0.971    &  1.147 &  0.002     & 0.068    &   <     \\
            5  &  0.706   &  1.147   & 0.882    &  0.971 &  0.040     & 0.002    &   >     \\
            6  &  0.882   &  1.147   & 0.971    &  1.059 &  0.002     & 0.011    &   <     \\
            7  &  0.882   &  1.059   & 0.971    &  0.981 &  0.002     & 0.001    &   >     \\
            8  &  0.971   &  1.059   &          &        &            &          &       \\
        \end{tblr}
    \end{center}
\end{solution}

\begin{problem}{2.2$\bigstar$}
    $f(x)=e^2-e^{-x}$,区间为$[0,1]$
\end{problem}
\begin{solution}
    在[0,1]区间内,易见f(x)单调,故f(x)是[0,1]上的单峰函数。
    采用分数法,$\varepsilon=0.1$,则$F_n\geq1/0.1=10$,n=6,迭代计算
    \begin{flalign*}
        &\text{第1次迭代:}\\
        &a_1=0,b_1=1,\lambda_1=0+\frac{5}{13}(1-0)=0.385,\mu_1=0+\frac{8}{13}(1-0)=0.615\\
        &f(\lambda_1)=e^2-e^{(-0.385)}=6.709\\
        &f(\mu_1)=e^2-e^{(-0.615)}=6.848\\
        &f(\lambda_1)<f(\mu_1)\\
        &\text{第2次迭代:}\\
        &a_2=0,b_2=\mu_1=0.615,\lambda_2=0+\frac{3}{8}(0.615-0)=0.231,\mu_2=\lambda_1=0.385\\
        &f(\lambda_2)=e^2-e^{(-0.231)}=6.595\\
        &f(\mu_2)=f(\lambda_1)=6.709\\
        &f(\lambda_2)<f(\mu_2)\\
        &\text{第3次迭代:}\\
        &a_3=0,b_3=\mu_2=0.385,\lambda_3=0+\frac{2}{5}(0.385-0)=0.154,\mu_3=\lambda_2=0.231\\
        &f(\lambda_3)=e^2-e^{(-0.154)}=6.532\\
        &f(\mu_3)=f(\lambda_2)=6.595\\
        &f(\lambda_3)<f(\mu_3)\\
        &\text{第4次迭代:}\\
        &a_4=0,b_4=\mu_3=0.231,\lambda_4=0+\frac{1}{3}(0.231-0)=0.077,\mu_4=\lambda_3=0.154\\
        &f(\lambda_4)=e^2-e^{(-0.077)}=6.463\\
        &f(\mu_4)=f(\lambda_3)=6.532\\
        &f(\lambda_4)<f(\mu_4)\\
        &\text{第5次迭代:}\\
        &a_5=0,b_5=\mu_4=0.154,\lambda_5=0+\frac{1}{2}(0.154-0)=0.077,\mu_5=\lambda_4+\delta=0.077+0.01=0.087\\
        &f(\lambda_5)=e^2-e^{(-0.077)}=6.463\\
        &f(\mu_5)=e^2-e^{(-0.087)}=6.472\\
        &f(\lambda_5)<f(\mu_5)\\
        &\text{第6次迭代:}\\
        &a_6=0,b_6=\mu_5=0.077,b_6-a_6=0.077<0.1,\text{结束计算}
    \end{flalign*}
    极小值点为1/2*(0+0.077)=0.0385\\
    列表结果如下
    \begin{center}
        \begin{tblr}{
                hlines,
                vlines,
                row{1} = {mode = math},
            }
            k  & a_k      & b_k    &\lambda_k&\mu_k&f(\lambda_k)&f(\mu_k)& \text{不等式方向} \\
            1  &  0       &  1       & 0.385    &  0.615 &  6.709     & 6.848    &     <    \\
            2  &  0       &  0.615   & 0.231    &  0.385 &  6.595     & 6.709    &     <    \\
            3  &  0       &  0.385   & 0.154    &  0.231 &  6.532     & 6.595    &     <    \\
            4  &  0       &  0.231   & 0.077    &  0.154 &  6.463     & 6.532    &     <    \\
            5  &  0       &  0.154   & 0.077    &  0.087 &  6.463     & 6.472    &     <    \\
            6  &  0       &  0.077   &          &        &            &          &          \\
        \end{tblr}
    \end{center}
\end{solution}

\begin{problem}{2.3$\bigstar$}
    $f(x)=1-xe^{-x^2}$,区间为$[0,1]$
\end{problem}
\begin{solution}
    $f'(x)=e^{-x^2}(2x^2-1)=0$,解得$x=\frac{\sqrt{2}}{2}$或$-\frac{\sqrt{2}}{2}$,仅前者在[0,1]区间内,故f(x)在[0,1]上仅一个驻点,是[0,1]上的单峰函数。
    采用分数法,$\varepsilon=0.1$,则$F_n\geq1/0.1=10$,n=6,迭代计算
    \begin{flalign*}
        &\text{第1次迭代:}\\
        &a_1=0,b_1=1,\lambda_1=0+\frac{5}{13}(1-0)=0.385,\mu_1=0+\frac{8}{13}(1-0)=0.615\\
        &f(\lambda_1)=1-0.385e^{-0.385^2}=0.668\\
        &f(\mu_1)=1-0.615e^{-0.615^2}=0.579\\
        &f(\lambda_1)>f(\mu_1)\\
        &\text{第2次迭代:}\\
        &a_2=\lambda_1=0.385,b_2=1,\lambda_2=\mu_1=0.615,\mu_2=0.385+\frac{5}{8}(1-0.385)=0.769\\
        &f(\lambda_2)=f(\mu_1)=0.579\\
        &f(\mu_2)=1-0.769e^{-0.769^2}=0.574\\
        &f(\lambda_2)>f(\mu_2)\\
        &\text{第3次迭代:}\\
        &a_3=\lambda_2=0.615,b_3=1,\lambda_3=\mu_2=0.769,\mu_3=0.615+\frac{3}{5}(1-0.615)=0.846\\
        &f(\lambda_3)=f(mu_2)=0.574\\
        &f(\mu_3)=1-0.846e^{-0.846^2}=0.586\\
        &f(\lambda_3)<f(\mu_3)\\
    \end{flalign*}
    \begin{flalign*}
        &\text{第4次迭代:}\\
        &a_4=0.615,b_4=\mu_3=0.846,\lambda_4=0.615+\frac{1}{3}(0.846-0.615)=0.692,\mu_4=\lambda_3=0.769\\
        &f(\lambda_4)=1-0.692e^{-0.692^2}=0.571\\
        &f(\mu_4)=1-0.769e^{-0.769^2}=0.574\\
        &f(\lambda_4)<f(\mu_4)\\
        &\text{第5次迭代:}\\
        &a_5=0.615,b_5=\mu_4=0.769,\lambda_5=0.615+\frac{1}{2}(0.769-0.615)=0.692,\mu_5=\lambda_4+\delta=0.692+0.01=0.702\\
        &f(\lambda_5)=1-0.692e^{-0.692^2}=0.5713\\
        &f(\mu_5)=1-0.702e^{-0.702^2}=0.5711\\
        &f(\lambda_5)>f(\mu_5)\\
        &\text{第6次迭代:}\\
        &a_6=\lambda_5=0.692,b_6=0.769,b_6-a_6=0.070<0.1,\text{结束计算}
    \end{flalign*}
    极小值点为1/2*(0.692+0.769)=0.719\\
    列表结果如下
    \begin{center}
        \begin{tblr}{
                hlines,
                vlines,
                row{1} = {mode = math},
            }
            k  & a_k      & b_k    &\lambda_k&\mu_k&f(\lambda_k)&f(\mu_k)& \text{不等式方向} \\
            1  &  0       &  1       & 0.385    &  0.615 &  0.668     & 0.579    &     >    \\
            2  &  0.385   &  1       & 0.615    &  0.769 &  0.579     & 0.574    &     >    \\
            3  &  0.615   &  1       & 0.769    &  0.846 &  0.574     & 0.586    &     <    \\
            4  &  0.615   &  0.846   & 0.692    &  0.769 &  0.571     & 0.574    &     <    \\
            5  &  0.615   &  0.769   & 0.692    &  0.702 &  0.5713    & 0.5711   &     >    \\
            6  &  0.692   &  0.769   &          &        &            &          &          \\
        \end{tblr}
    \end{center}
\end{solution}
