\section{非线性规划问题}

\subsection{凸函数与凸规划}

\subsubsection{凸规划判定}

\begin{problem}{1.1$\bigstar$}
    \begin{mini*}|s|
        {}
        {f(x) = 2x_1^2 + x_2^2 + x_3^2}
        {}
        {}
        \addConstraint{-x_1^2 - x_2^2 + 4}{\geq 0}
        \addConstraint{5x_1 - 4x_2 \ph}{=8}
        \addConstraint{x_1,x_2,x_3}{\geq 0}{.}
    \end{mini*}
\end{problem}
\begin{solution}
    第二、三条件为线性函数,把它们看成凹函数,\\
    第一约束条件的海赛矩阵是$\nabla^2g_1(x)=\begin{bmatrix}
        -2  & 0  \\
        0  & -2  \\
    \end{bmatrix}$,负定,凹函数,\\
    目标函数的海赛矩阵是$\nabla^2g_1(x)=\begin{bmatrix}
        4  & 0 & 0  \\
        0  & 2 & 0  \\
        0  & 0 & 2  \\
    \end{bmatrix}$,正定,为凸函数,\\
    从而可知该问题是凸规划,有唯一最小值解。
\end{solution}
\begin{problem}{1.2$\bigstar$}
    \begin{mini*}|s|
        {}
        {f(x) = x_1 + 3x_2}
        {}
        {}
        \addConstraint{x_1^2 + x_2^2}{\leq 9}
        \addConstraint{x_2}{\geq 0}{.}
    \end{mini*}
\end{problem}
\begin{solution}

    目标函数是线性看书,把其看为凸函数;\\
    第一约束条件两边同乘-1,改为大于等于号后的海赛矩阵是
    $\nabla^2g_1(x)=\begin{bmatrix}
        -2  & 0  \\
        0  & -2  \\
    \end{bmatrix}$,负定,凹函数,\\
    第二约束条件为线性函数,把其看成凹函数\\
    从而可知该问题是凸规划,有唯一最小值解。
\end{solution}

\subsection{最优性条件}


\subsubsection{Kuhn-Tucker点判定}

\begin{problem}{1.1$\bigstar$}
    \begin{mini*}|s|
        {}
        {f(x) = (x_1 - 3)^2 + (x_2 - 2)^2}
        {}
        {}
        \addConstraint{x_1^2 + x_2^2}{\leq 5}
        \addConstraint{x_1 + 2x_2}{=4}
        \addConstraint{x_1,x_2}{\geq 0}{.}
    \end{mini*}
    $$\bar{x}=[2\ 1]^T$$
\end{problem}
\begin{solution}
    令
    \begin{align*}
        g_1(x)&=-x_1^2 - x_2^2 + 5\geq0,\\
        g_2(x)&=x_1\geq0\\
        g_3(x)&=x_2\geq0\\
        h(x)&=x_1 + 2x_2-4=0
    \end{align*}
    将$\bar{x}=[2\ 1]^T$代入$g(x)$,易见$g_1(x)$等式成立,$g_2(x)、g_3(x)$不等式成立,则有$I(\bar{x})=\{1\}$,\\
    $\nabla f(\bar{x})=[2x_1-6\ 2x_2-4]^T=[-2\ -2]^T$,\\
    $\nabla g_1(\bar{x})=[-2x_1\ -2x_2]^T=[-4\ -2]^T$,\\
    $\nabla g_2(\bar{x})=[1\ 0]^T$,\\
    $\nabla g_3(\bar{x})=[0\ 1]^T$,\\
    $\nabla h(\bar{x})=[1\ 2]^T$,\\
    根据KT一阶必要条件,则有
    $$\left\{
    \begin{aligned}
        -2 - (-4)\lambda_1 - \mu_1 &=0\\
        -2 - (-2)\lambda_1 - 2\mu_1 &=0\\
        \lambda_1&\geq0\\
        \lambda_2=\lambda_3&=0
    \end{aligned}\right.$$
    解得$\lambda_1=\frac{1}{3},\mu_1=-\frac{2}{3}$,故点$[2\ 1]^T$是原问题KT点。
\end{solution}

\begin{problem}{1.2$\bigstar$}
    \begin{mini*}|s|
        {}
        {f(x) = (x_1 - 2)^2 + x_2^2}
        {}
        {}
        \addConstraint{x_1 - x_2^2 }{\geq 0}
        \addConstraint{-x_1 + x_2}{\geq 0}
    \end{mini*}
    $$\bar{x}^{(1)}=[0\ ,0]^T,\bar{x}^{(2)}=[1\ 1]^T$$
\end{problem}
\begin{solution}
    令
    \begin{align*}
        g_1(x)&=x_1 - x_2^2 \geq 0\\
        g_2(x)&=-x_1 + x_2 \geq 0\\
    \end{align*}
    (1)先验证$\bar{x}^{(1)}=[0\ ,0]^T$\\
    将$\bar{x}^{(1)}=[0\ 0]^T$代入$g(x)$,易见$g_1(x),g_2(x)$等式成立,则有$I(\bar{x}^{(1)})=\{1,2\}$,\\
    $\nabla f(\bar{x}^{(1)})=[2x_1-4\ 2x_2]^T=[-4\ 0]^T$,\\
    $\nabla g_1(\bar{x}^{(1)})=[1\ -2x_2]^T=[1\ 0]^T$,\\
    $\nabla g_2(\bar{x}^{(1)})=[-1\ 1]^T$,\\
    根据KT一阶必要条件,则有
    $$\left\{
    \begin{aligned}
        -4 - \lambda_1 - (-1)\lambda_2 &=0\\
        0 \ph - \lambda_2 &=0\\
        \lambda_1&\geq0\\
        \lambda_2&\geq0
    \end{aligned}\right.$$
    解得$\lambda_1=4,\lambda_2=0$,故点$[0\ 0]^T$是原问题KT点。\\
    (2)再验证$\bar{x}^{(2)}=[1\ 1]^T$\\
    将$\bar{x}^{(2)}=[1\ 1]^T$代入$g(x)$,易见$g_1(x),g_2(x)$等式成立,则有$I(\bar{x}^{(2)})=\{1,2\}$,\\
    $\nabla f(\bar{x}^{(2)})=[2x_1-4\ 2x_2]^T=[-2\ 2]^T$,\\
    $\nabla g_1(\bar{x}^{(2)})=[1\ -2x_2]^T=[1\ -2]^T$,\\
    $\nabla g_2(\bar{x}^{(2)})=[-1\ 1]^T$,\\
    根据KT一阶必要条件,则有
    $$\left\{
    \begin{aligned}
        -2 - \lambda_1 - (-1)\lambda_2 &=0\\
        2 - (-2)\lambda_1 - \lambda_2 &=0\\
        \lambda_1&\geq0\\
        \lambda_2&\geq0
    \end{aligned}\right.$$
    解得$\lambda_1=0,\lambda_2=2$,故点$[1\ 1]^T$是原问题KT点。
\end{solution}

\subsubsection{Kuhn-Tucker点求解,并验证是否最优解}

\begin{problem}{2.1$\bigstar$}
    \begin{mini*}|s|
        {}
        {f(x) = (x_1 - 1)^2 + (x_1 - 2)^2}
        {}
        {}
        \addConstraint{-x_1 + x_2}{= 1}
        \addConstraint{x_1 + x_2}{= 2}
        \addConstraint{x_1,x_2}{\geq 0}{.}
    \end{mini*}
\end{problem}
\begin{solution}
    令
    \begin{align*}
        g_1(x)&=x_1\geq0\\
        g_2(x)&=x_2\geq0\\
        h_1(x)&=-x_1 + x_2 - 1 =0\\
        h_2(x)&=x_1 + x_2 - 2 = 0\\
    \end{align*}
    则有
    \begin{align*}
        \nabla f(x)&=[2x_1-2\ 2x_2-4]^T\\
        \nabla g_1(x)&=[1\ 0]^T\\
        \nabla g_2(x)&=[0\ 1]^T\\
        \nabla h_1(x)&=[-1\ 1]^T\\
        \nabla h_2(x)&=[1\ 1]^T\\
    \end{align*}
    根据KT一阶必要条件,则有
    $$\left\{
    \begin{aligned}
        2x_1 - 2 - \lambda_1 - (-1)\mu_1 - \mu_2 &=0\\
        2x_2 - 4 - \lambda_2 - \mu_1 - \mu_2 &=0\\
        \lambda_1(x_1)&=0\\
        \lambda_2(x_2)&=0\\
        \lambda_1&\geq0\\
        \lambda_2&\geq0
    \end{aligned}\right.$$
    解得$\lambda_1=4,\lambda_2=0$,故点$[0\ 0]^T$是原问题KT点。
\end{solution}

\begin{problem}{2.2$\bigstar$}
    \begin{mini*}|s|
        {}
        {f(x) = x_1^2 - x_2 - 3x_3}
        {}
        {}
        \addConstraint{-x_1 - x_2 - x_3}{\geq 0}
        \addConstraint{x_1^2 + 2x_2 - x_3}{=0}
    \end{mini*}
\end{problem}
\begin{solution}
    令
    \begin{align*}
        g(x)&=-x_1 - x_2 - x_3\geq0\\
        h(x)&=x_1^2 + 2x_2 - x_3 =0\\
    \end{align*}
    则有
    \begin{align*}
    \nabla f(x)&=[2x_1\ -1\ -3]^T\\
    \nabla g(x)&=[-1\ -1\ -1]^T\\
    \nabla h(x)&=[2x_1\ 2\ -1]^T\\
    \end{align*}
    根据KT一阶必要条件,则有
    $$\left\{
    \begin{aligned}
        2x_1 - (-1)\lambda - (2x_1)\mu &=0\\
        -1 - (-1)\lambda - 2\mu  &=0\\
        -3 - (-1)\lambda - (-1)\mu  &=0\\
        \lambda(-x_1 - x_2 - x_3)&=0\\
        \lambda&\geq0
    \end{aligned}\right.$$
    解得$\lambda=\frac{7}{3},\mu=\frac{2}{3},x_1=-\frac{7}{2}$。
    再将$x_1$代入下式求$x_2$与$x_3$
        $$\left\{
    \begin{aligned}
        -x_1 - x_2 - x_3 &=0\\
        x_1^2 + 2x_2 - x_3&=0\\
        \lambda_1&\geq0
    \end{aligned}\right.$$
    解得$[x_1\ x_2\ x_3]^T=[-\frac{7}{2}\  -\frac{35}{12}\ \frac{77}{12}]^T$
    考虑拉格朗日函数
    \begin{align*}
    L(x,\lambda,\mu)&=f(x)-\lambda g(x)-\mu h(x)\\
    &=x_1^2 - x_2 - 3x_3 - \frac{7}{3}(-x_1 - x_2 - x_3) - \frac{2}{3}(x_1^2 + 2x_2 - x_3)\\
    &=\frac{1}{3}x_1^2+\frac{7}{3}x_1\\
    \nabla^2 L(x,\lambda,\mu)&=\begin{bmatrix}
        \frac{2}{3}  & 0 & 0  \\
        0  & 0 & 0  \\
        0  & 0 & 0  \\
    \end{bmatrix}\\
    \end{align*}
    海赛矩阵半正定,故点$[-\frac{7}{2}\  -\frac{35}{12}\ \frac{77}{12}]^T$是局部最优解

\end{solution}

\subsubsection{Fritz-John点判定}

\begin{problem}{3.1$\bigstar$}
    \begin{mini*}|s|
        {}
        {f(x) = -x_2}
        {}
        {}
        \addConstraint{-2x_1 + (2 - x_2)^3}{\geq 0}
        \addConstraint{x_1 \ph\ph}{\geq 0.}
    \end{mini*}
    $$\bar{x}=[0\ 2]^T$$
\end{problem}
\begin{solution}
    令
    \begin{align*}
        g_1(x)&=-2x_1 + (2 - x_2)^3\geq 0\\
        g_2(x)&=x_1\geq0\\
    \end{align*}
    将$\bar{x}=[0\ 2]^T$代入$g(x)$,易见$g_1(x),g_2(x)$等式成立,则有$I(\bar{x})=\{1,2\}$,\\
    $\nabla f(\bar{x})=[0\ -1]^T$,\\
    $\nabla g_1(\bar{x})=[-2\ -3x_2^2+12x_2]^T=[-2\ 12]^T$,\\
    $\nabla g_2(\bar{x})=[1\ 0]^T$,\\
    根据FJ一阶必要条件,则有
    $$\left\{
    \begin{aligned}
        (0)\lambda_0 - (-2)\lambda_1 - \lambda_2 &=0\\
        (-1)\lambda_0 - 12\lambda_1 - (0)\lambda_2 &=0\\
    \end{aligned}\right.$$
    该方程组中$\lambda_0$与$\lambda_1$异号,找不到非负解,故点$[0\ 2]^T$不是原问题FJ点。
\end{solution}
\begin{problem}{3.2$\bigstar$}
    \begin{mini*}|s|
        {}
        {f(x) = -x_1}
        {}
        {}
        \addConstraint{(1 - x_1)^2 + x_2}{\geq 0}
        \addConstraint{-x_2}{\geq 0}{.}
    \end{mini*}
    $$\bar{x}=[1\ 0]^T$$
\end{problem}
\begin{solution}
    令
    \begin{align*}
        g_1(x)&=(1 - x_1)^2 + x_2\geq 0\\
        g_2(x)&=x_2\geq0\\
    \end{align*}
    将$\bar{x}=[1\ 0]^T$代入$g(x)$,易见$g_1(x),g_2(x)$等式成立,则有$I(\bar{x})=\{1,2\}$,\\
    $\nabla f(\bar{x})=[-1\ 0]^T$,\\
    $\nabla g_1(\bar{x})=[2x_1-2\ 1]^T=[0\ 1]^T$,\\
    $\nabla g_2(\bar{x})=[0\ 1]^T$,\\
    根据FJ一阶必要条件,则有
    $$\left\{
    \begin{aligned}
        (-1)\lambda_0 - (0)\lambda_1 - (0)\lambda_2 &=0\\
        (0)\lambda_0 - \lambda_1 - \lambda_2 &=0\\
    \end{aligned}\right.$$
    该方程组中有无穷非负解,如$[0\ 1\ 1]$,故点$[1\ 0]^T$是原问题FJ点。
\end{solution}